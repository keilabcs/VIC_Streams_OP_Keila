% Options for packages loaded elsewhere
\PassOptionsToPackage{unicode}{hyperref}
\PassOptionsToPackage{hyphens}{url}
%
\documentclass[
]{article}
\usepackage{amsmath,amssymb}
\usepackage{iftex}
\ifPDFTeX
  \usepackage[T1]{fontenc}
  \usepackage[utf8]{inputenc}
  \usepackage{textcomp} % provide euro and other symbols
\else % if luatex or xetex
  \usepackage{unicode-math} % this also loads fontspec
  \defaultfontfeatures{Scale=MatchLowercase}
  \defaultfontfeatures[\rmfamily]{Ligatures=TeX,Scale=1}
\fi
\usepackage{lmodern}
\ifPDFTeX\else
  % xetex/luatex font selection
\fi
% Use upquote if available, for straight quotes in verbatim environments
\IfFileExists{upquote.sty}{\usepackage{upquote}}{}
\IfFileExists{microtype.sty}{% use microtype if available
  \usepackage[]{microtype}
  \UseMicrotypeSet[protrusion]{basicmath} % disable protrusion for tt fonts
}{}
\makeatletter
\@ifundefined{KOMAClassName}{% if non-KOMA class
  \IfFileExists{parskip.sty}{%
    \usepackage{parskip}
  }{% else
    \setlength{\parindent}{0pt}
    \setlength{\parskip}{6pt plus 2pt minus 1pt}}
}{% if KOMA class
  \KOMAoptions{parskip=half}}
\makeatother
\usepackage{xcolor}
\usepackage[margin=1in]{geometry}
\usepackage{color}
\usepackage{fancyvrb}
\newcommand{\VerbBar}{|}
\newcommand{\VERB}{\Verb[commandchars=\\\{\}]}
\DefineVerbatimEnvironment{Highlighting}{Verbatim}{commandchars=\\\{\}}
% Add ',fontsize=\small' for more characters per line
\usepackage{framed}
\definecolor{shadecolor}{RGB}{248,248,248}
\newenvironment{Shaded}{\begin{snugshade}}{\end{snugshade}}
\newcommand{\AlertTok}[1]{\textcolor[rgb]{0.94,0.16,0.16}{#1}}
\newcommand{\AnnotationTok}[1]{\textcolor[rgb]{0.56,0.35,0.01}{\textbf{\textit{#1}}}}
\newcommand{\AttributeTok}[1]{\textcolor[rgb]{0.13,0.29,0.53}{#1}}
\newcommand{\BaseNTok}[1]{\textcolor[rgb]{0.00,0.00,0.81}{#1}}
\newcommand{\BuiltInTok}[1]{#1}
\newcommand{\CharTok}[1]{\textcolor[rgb]{0.31,0.60,0.02}{#1}}
\newcommand{\CommentTok}[1]{\textcolor[rgb]{0.56,0.35,0.01}{\textit{#1}}}
\newcommand{\CommentVarTok}[1]{\textcolor[rgb]{0.56,0.35,0.01}{\textbf{\textit{#1}}}}
\newcommand{\ConstantTok}[1]{\textcolor[rgb]{0.56,0.35,0.01}{#1}}
\newcommand{\ControlFlowTok}[1]{\textcolor[rgb]{0.13,0.29,0.53}{\textbf{#1}}}
\newcommand{\DataTypeTok}[1]{\textcolor[rgb]{0.13,0.29,0.53}{#1}}
\newcommand{\DecValTok}[1]{\textcolor[rgb]{0.00,0.00,0.81}{#1}}
\newcommand{\DocumentationTok}[1]{\textcolor[rgb]{0.56,0.35,0.01}{\textbf{\textit{#1}}}}
\newcommand{\ErrorTok}[1]{\textcolor[rgb]{0.64,0.00,0.00}{\textbf{#1}}}
\newcommand{\ExtensionTok}[1]{#1}
\newcommand{\FloatTok}[1]{\textcolor[rgb]{0.00,0.00,0.81}{#1}}
\newcommand{\FunctionTok}[1]{\textcolor[rgb]{0.13,0.29,0.53}{\textbf{#1}}}
\newcommand{\ImportTok}[1]{#1}
\newcommand{\InformationTok}[1]{\textcolor[rgb]{0.56,0.35,0.01}{\textbf{\textit{#1}}}}
\newcommand{\KeywordTok}[1]{\textcolor[rgb]{0.13,0.29,0.53}{\textbf{#1}}}
\newcommand{\NormalTok}[1]{#1}
\newcommand{\OperatorTok}[1]{\textcolor[rgb]{0.81,0.36,0.00}{\textbf{#1}}}
\newcommand{\OtherTok}[1]{\textcolor[rgb]{0.56,0.35,0.01}{#1}}
\newcommand{\PreprocessorTok}[1]{\textcolor[rgb]{0.56,0.35,0.01}{\textit{#1}}}
\newcommand{\RegionMarkerTok}[1]{#1}
\newcommand{\SpecialCharTok}[1]{\textcolor[rgb]{0.81,0.36,0.00}{\textbf{#1}}}
\newcommand{\SpecialStringTok}[1]{\textcolor[rgb]{0.31,0.60,0.02}{#1}}
\newcommand{\StringTok}[1]{\textcolor[rgb]{0.31,0.60,0.02}{#1}}
\newcommand{\VariableTok}[1]{\textcolor[rgb]{0.00,0.00,0.00}{#1}}
\newcommand{\VerbatimStringTok}[1]{\textcolor[rgb]{0.31,0.60,0.02}{#1}}
\newcommand{\WarningTok}[1]{\textcolor[rgb]{0.56,0.35,0.01}{\textbf{\textit{#1}}}}
\usepackage{graphicx}
\makeatletter
\def\maxwidth{\ifdim\Gin@nat@width>\linewidth\linewidth\else\Gin@nat@width\fi}
\def\maxheight{\ifdim\Gin@nat@height>\textheight\textheight\else\Gin@nat@height\fi}
\makeatother
% Scale images if necessary, so that they will not overflow the page
% margins by default, and it is still possible to overwrite the defaults
% using explicit options in \includegraphics[width, height, ...]{}
\setkeys{Gin}{width=\maxwidth,height=\maxheight,keepaspectratio}
% Set default figure placement to htbp
\makeatletter
\def\fps@figure{htbp}
\makeatother
\setlength{\emergencystretch}{3em} % prevent overfull lines
\providecommand{\tightlist}{%
  \setlength{\itemsep}{0pt}\setlength{\parskip}{0pt}}
\setcounter{secnumdepth}{-\maxdimen} % remove section numbering
\ifLuaTeX
  \usepackage{selnolig}  % disable illegal ligatures
\fi
\IfFileExists{bookmark.sty}{\usepackage{bookmark}}{\usepackage{hyperref}}
\IfFileExists{xurl.sty}{\usepackage{xurl}}{} % add URL line breaks if available
\urlstyle{same}
\hypersetup{
  pdftitle={Tricky Time Series},
  pdfauthor={Alejandro C. Frery and Keila Barbosa},
  hidelinks,
  pdfcreator={LaTeX via pandoc}}

\title{Tricky Time Series}
\author{Alejandro C. Frery and Keila Barbosa}
\date{2024-07-26}

\begin{document}
\maketitle

\hypertarget{change-detected-vs.-drift-exemplos-reais}{%
\subsubsection{Change Detected vs.~Drift: Exemplos
Reais}\label{change-detected-vs.-drift-exemplos-reais}}

\hypertarget{change-detected-mudanuxe7a-detectada}{%
\paragraph{Change Detected (Mudança
Detectada)}\label{change-detected-mudanuxe7a-detectada}}

Se uma mudança é detectada (retorna TRUE), isso significa que houve uma
alteração significativa nos dados, de acordo com o critério estatístico
definido.

\textbf{Exemplo:}

Imagine que estamos monitorando a temperatura de uma sala de servidores.
A temperatura é normalmente estável em torno de 22°C, mas de repente o
sistema de ar condicionado falha, fazendo com que a temperatura suba
rapidamente para 30°C.

\begin{Shaded}
\begin{Highlighting}[]
\NormalTok{knitr}\SpecialCharTok{::}\NormalTok{opts\_chunk}\SpecialCharTok{$}\FunctionTok{set}\NormalTok{(}\AttributeTok{echo =} \ConstantTok{TRUE}\NormalTok{)}
\FunctionTok{library}\NormalTok{(pracma)}
\FunctionTok{library}\NormalTok{(ggplot2)}
\FunctionTok{library}\NormalTok{(ggthemes)}
\FunctionTok{library}\NormalTok{(statcomp)}

\CommentTok{\# Função para calcular padrões ordinais usando statcomp}
\NormalTok{ordinal\_patterns\_statcomp }\OtherTok{\textless{}{-}} \ControlFlowTok{function}\NormalTok{(series, emb\_dim) \{}
  \ControlFlowTok{if}\NormalTok{ (}\FunctionTok{length}\NormalTok{(series) }\SpecialCharTok{\textless{}}\NormalTok{ emb\_dim) \{}
    \FunctionTok{stop}\NormalTok{(}\StringTok{"A série temporal é muito curta para a dimensão de embedding especificada."}\NormalTok{)}
\NormalTok{  \}}
  
  \CommentTok{\# Utilizando a função from \textasciigrave{}statcomp\textasciigrave{} para calcular os padrões ordinais}
\NormalTok{  patterns }\OtherTok{\textless{}{-}} \FunctionTok{ordinal\_pattern}\NormalTok{(series, emb\_dim)}
  
  \FunctionTok{return}\NormalTok{(patterns)}
\NormalTok{\}}

\CommentTok{\# Função ADWIN ajustada para padrões ordinais}
\NormalTok{ADWIN }\OtherTok{\textless{}{-}} \ControlFlowTok{function}\NormalTok{(}\AttributeTok{delta =} \FloatTok{0.002}\NormalTok{) \{}
\NormalTok{  width }\OtherTok{\textless{}{-}} \DecValTok{0}
\NormalTok{  total }\OtherTok{\textless{}{-}} \DecValTok{0}
\NormalTok{  variance }\OtherTok{\textless{}{-}} \DecValTok{0}
\NormalTok{  window }\OtherTok{\textless{}{-}} \FunctionTok{numeric}\NormalTok{(}\DecValTok{0}\NormalTok{)}
  
\NormalTok{  update }\OtherTok{\textless{}{-}} \ControlFlowTok{function}\NormalTok{(value) \{}
\NormalTok{    width }\OtherTok{\textless{}\textless{}{-}}\NormalTok{ width }\SpecialCharTok{+} \DecValTok{1}
\NormalTok{    window }\OtherTok{\textless{}\textless{}{-}} \FunctionTok{c}\NormalTok{(window, value)}
\NormalTok{    total }\OtherTok{\textless{}\textless{}{-}}\NormalTok{ total }\SpecialCharTok{+}\NormalTok{ value}
    \ControlFlowTok{if}\NormalTok{ (width }\SpecialCharTok{\textgreater{}} \DecValTok{1}\NormalTok{) \{}
\NormalTok{      variance }\OtherTok{\textless{}\textless{}{-}} \FunctionTok{var}\NormalTok{(window, }\AttributeTok{na.rm =} \ConstantTok{TRUE}\NormalTok{) }\CommentTok{\# Calcula a variância, removendo NAs}
\NormalTok{    \}}
    \ControlFlowTok{if}\NormalTok{ (width }\SpecialCharTok{\textgreater{}} \DecValTok{1} \SpecialCharTok{\&\&} \FunctionTok{detect\_change}\NormalTok{()) \{}
      \FunctionTok{return}\NormalTok{(}\ConstantTok{TRUE}\NormalTok{)}
\NormalTok{    \}}
    \FunctionTok{return}\NormalTok{(}\ConstantTok{FALSE}\NormalTok{)}
\NormalTok{  \}}
  
\NormalTok{  detect\_change }\OtherTok{\textless{}{-}} \ControlFlowTok{function}\NormalTok{() \{}
\NormalTok{    mean\_val }\OtherTok{\textless{}{-}} \FunctionTok{mean}\NormalTok{(window, }\AttributeTok{na.rm =} \ConstantTok{TRUE}\NormalTok{) }\CommentTok{\# Calcula a média, removendo NAs}
    \ControlFlowTok{for}\NormalTok{ (n }\ControlFlowTok{in} \DecValTok{1}\SpecialCharTok{:}\NormalTok{(width }\SpecialCharTok{{-}} \DecValTok{1}\NormalTok{)) \{}
\NormalTok{      mean0 }\OtherTok{\textless{}{-}} \FunctionTok{mean}\NormalTok{(window[}\DecValTok{1}\SpecialCharTok{:}\NormalTok{n], }\AttributeTok{na.rm =} \ConstantTok{TRUE}\NormalTok{)}
\NormalTok{      mean1 }\OtherTok{\textless{}{-}} \FunctionTok{mean}\NormalTok{(window[(n }\SpecialCharTok{+} \DecValTok{1}\NormalTok{)}\SpecialCharTok{:}\NormalTok{width], }\AttributeTok{na.rm =} \ConstantTok{TRUE}\NormalTok{)}
      \ControlFlowTok{if}\NormalTok{ (}\FunctionTok{is.na}\NormalTok{(mean0) }\SpecialCharTok{||} \FunctionTok{is.na}\NormalTok{(mean1)) }\ControlFlowTok{next} \CommentTok{\# Pula iteração se a média for NA}
      \ControlFlowTok{if}\NormalTok{ (}\FunctionTok{abs}\NormalTok{(mean0 }\SpecialCharTok{{-}}\NormalTok{ mean1) }\SpecialCharTok{\textgreater{}} \FunctionTok{sqrt}\NormalTok{((variance }\SpecialCharTok{/}\NormalTok{ n) }\SpecialCharTok{+}\NormalTok{ (variance }\SpecialCharTok{/}\NormalTok{ (width }\SpecialCharTok{{-}}\NormalTok{ n))) }\SpecialCharTok{*} \FunctionTok{qnorm}\NormalTok{(}\DecValTok{1} \SpecialCharTok{{-}}\NormalTok{ delta)) \{}
\NormalTok{        window }\OtherTok{\textless{}\textless{}{-}}\NormalTok{ window[(n }\SpecialCharTok{+} \DecValTok{1}\NormalTok{)}\SpecialCharTok{:}\NormalTok{width]}
\NormalTok{        width }\OtherTok{\textless{}\textless{}{-}} \FunctionTok{length}\NormalTok{(window)}
\NormalTok{        total }\OtherTok{\textless{}\textless{}{-}} \FunctionTok{sum}\NormalTok{(window, }\AttributeTok{na.rm =} \ConstantTok{TRUE}\NormalTok{)}
\NormalTok{        variance }\OtherTok{\textless{}\textless{}{-}} \FunctionTok{var}\NormalTok{(window, }\AttributeTok{na.rm =} \ConstantTok{TRUE}\NormalTok{)}
        \FunctionTok{return}\NormalTok{(}\ConstantTok{TRUE}\NormalTok{)}
\NormalTok{      \}}
\NormalTok{    \}}
    \FunctionTok{return}\NormalTok{(}\ConstantTok{FALSE}\NormalTok{)}
\NormalTok{  \}}
  
  \FunctionTok{list}\NormalTok{(}\AttributeTok{update =}\NormalTok{ update)}
\NormalTok{\}}

\CommentTok{\# Função para criar a série temporal}
\NormalTok{TrickyTimeSeries }\OtherTok{\textless{}{-}} \ControlFlowTok{function}\NormalTok{(n, k)\{}
\NormalTok{  y1 }\OtherTok{\textless{}{-}} \FunctionTok{rnorm}\NormalTok{(n)}
\NormalTok{  y2 }\OtherTok{\textless{}{-}} \FunctionTok{sqrt}\NormalTok{(}\DecValTok{12}\NormalTok{)}\SpecialCharTok{*}\FunctionTok{runif}\NormalTok{(n, }\AttributeTok{min=}\SpecialCharTok{{-}}\DecValTok{1}\SpecialCharTok{/}\DecValTok{2}\NormalTok{, }\AttributeTok{max=}\DecValTok{1}\SpecialCharTok{/}\DecValTok{2}\NormalTok{)}
\NormalTok{  y3 }\OtherTok{\textless{}{-}} \FunctionTok{rexp}\NormalTok{(n)}\SpecialCharTok{{-}}\DecValTok{1}
\NormalTok{  fk }\OtherTok{\textless{}{-}} \FunctionTok{TK95}\NormalTok{(}\AttributeTok{N=}\DecValTok{3}\SpecialCharTok{*}\NormalTok{n, }\AttributeTok{alpha=}\NormalTok{k)}
\NormalTok{  fk }\OtherTok{\textless{}{-}}\NormalTok{ (fk}\SpecialCharTok{{-}}\FunctionTok{mean}\NormalTok{(fk))}\SpecialCharTok{/}\FunctionTok{sd}\NormalTok{(fk)}
\NormalTok{  output }\OtherTok{\textless{}{-}} \FunctionTok{c}\NormalTok{(y1, y2, y3, fk)}
  \FunctionTok{return}\NormalTok{(output)}
\NormalTok{\}}

\CommentTok{\# Função para gerar ruído colorido}
\NormalTok{TK95 }\OtherTok{\textless{}{-}} \ControlFlowTok{function}\NormalTok{(N, }\AttributeTok{alpha =} \DecValTok{1}\NormalTok{)\{ }
\NormalTok{    f }\OtherTok{\textless{}{-}} \FunctionTok{seq}\NormalTok{(}\AttributeTok{from=}\DecValTok{0}\NormalTok{, }\AttributeTok{to=}\NormalTok{pi, }\AttributeTok{length.out=}\NormalTok{(N}\SpecialCharTok{/}\DecValTok{2}\SpecialCharTok{+}\DecValTok{1}\NormalTok{))[}\SpecialCharTok{{-}}\FunctionTok{c}\NormalTok{(}\DecValTok{1}\NormalTok{,(N}\SpecialCharTok{/}\DecValTok{2}\SpecialCharTok{+}\DecValTok{1}\NormalTok{))] }\CommentTok{\# Frequências de Fourier}
\NormalTok{    f\_ }\OtherTok{\textless{}{-}} \DecValTok{1} \SpecialCharTok{/}\NormalTok{ f}\SpecialCharTok{\^{}}\NormalTok{alpha }\CommentTok{\# Lei de potência}
\NormalTok{    RW }\OtherTok{\textless{}{-}} \FunctionTok{sqrt}\NormalTok{(}\FloatTok{0.5}\SpecialCharTok{*}\NormalTok{f\_) }\SpecialCharTok{*} \FunctionTok{rnorm}\NormalTok{(N}\SpecialCharTok{/}\DecValTok{2{-}1}\NormalTok{) }\CommentTok{\# Parte real}
\NormalTok{    IW }\OtherTok{\textless{}{-}} \FunctionTok{sqrt}\NormalTok{(}\FloatTok{0.5}\SpecialCharTok{*}\NormalTok{f\_) }\SpecialCharTok{*} \FunctionTok{rnorm}\NormalTok{(N}\SpecialCharTok{/}\DecValTok{2{-}1}\NormalTok{) }\CommentTok{\# Parte imaginária}
\NormalTok{    fR }\OtherTok{\textless{}{-}} \FunctionTok{complex}\NormalTok{(}\AttributeTok{real =} \FunctionTok{c}\NormalTok{(}\FunctionTok{rnorm}\NormalTok{(}\DecValTok{1}\NormalTok{), RW, }\FunctionTok{rnorm}\NormalTok{(}\DecValTok{1}\NormalTok{), RW[(N}\SpecialCharTok{/}\DecValTok{2{-}1}\NormalTok{)}\SpecialCharTok{:}\DecValTok{1}\NormalTok{]), }
                  \AttributeTok{imaginary =} \FunctionTok{c}\NormalTok{(}\DecValTok{0}\NormalTok{, IW, }\DecValTok{0}\NormalTok{, }\SpecialCharTok{{-}}\NormalTok{IW[(N}\SpecialCharTok{/}\DecValTok{2{-}1}\NormalTok{)}\SpecialCharTok{:}\DecValTok{1}\NormalTok{]), }\AttributeTok{length.out=}\NormalTok{N)}
\NormalTok{    reihe }\OtherTok{\textless{}{-}} \FunctionTok{fft}\NormalTok{(fR, }\AttributeTok{inverse=}\ConstantTok{TRUE}\NormalTok{) }\CommentTok{\# Retornar ao domínio do tempo com transformada inversa de Fourier}
    \FunctionTok{return}\NormalTok{(}\FunctionTok{Re}\NormalTok{(reihe)) }\CommentTok{\# Retorna apenas a parte real}
\NormalTok{\}}


\CommentTok{\# Função para calcular padrões ordinais}
\CommentTok{\# Esta função calcula padrões ordinais para uma série temporal dada uma dimensão de embedding.}
\CommentTok{\# Patterns: Vetor com os padrões ordinais calculados}
\CommentTok{\# series: A série temporal de entrada}
\CommentTok{\# emb\_dim: Dimensão de embedding (por exemplo, 3)}

\NormalTok{ordinal\_patterns }\OtherTok{\textless{}{-}} \ControlFlowTok{function}\NormalTok{(series, emb\_dim) \{}
\NormalTok{  n }\OtherTok{\textless{}{-}} \FunctionTok{length}\NormalTok{(series)}
  \ControlFlowTok{if}\NormalTok{ (n }\SpecialCharTok{\textless{}}\NormalTok{ emb\_dim) \{}
    \FunctionTok{stop}\NormalTok{(}\StringTok{"A série temporal é muito curta para a dimensão de embedding especificada."}\NormalTok{)}
\NormalTok{  \}}

\NormalTok{  patterns }\OtherTok{\textless{}{-}} \FunctionTok{numeric}\NormalTok{(n }\SpecialCharTok{{-}}\NormalTok{ emb\_dim }\SpecialCharTok{+} \DecValTok{1}\NormalTok{)}
  \ControlFlowTok{for}\NormalTok{ (i }\ControlFlowTok{in} \DecValTok{1}\SpecialCharTok{:}\NormalTok{(n }\SpecialCharTok{{-}}\NormalTok{ emb\_dim }\SpecialCharTok{+} \DecValTok{1}\NormalTok{)) \{}
\NormalTok{    subseq }\OtherTok{\textless{}{-}}\NormalTok{ series[i}\SpecialCharTok{:}\NormalTok{(i }\SpecialCharTok{+}\NormalTok{ emb\_dim }\SpecialCharTok{{-}} \DecValTok{1}\NormalTok{)]}
\NormalTok{    ranks }\OtherTok{\textless{}{-}} \FunctionTok{rank}\NormalTok{(subseq, }\AttributeTok{ties.method =} \StringTok{"first"}\NormalTok{)}
\NormalTok{    pattern }\OtherTok{\textless{}{-}} \FunctionTok{sum}\NormalTok{((ranks }\SpecialCharTok{{-}} \DecValTok{1}\NormalTok{) }\SpecialCharTok{*}\NormalTok{ (emb\_dim }\SpecialCharTok{\^{}}\NormalTok{ (}\DecValTok{0}\SpecialCharTok{:}\NormalTok{(emb\_dim }\SpecialCharTok{{-}} \DecValTok{1}\NormalTok{))))}
\NormalTok{    patterns[i] }\OtherTok{\textless{}{-}}\NormalTok{ pattern}
\NormalTok{  \}}
  \FunctionTok{return}\NormalTok{(patterns)}
\NormalTok{\}}
\end{Highlighting}
\end{Shaded}

\begin{Shaded}
\begin{Highlighting}[]
\FunctionTok{set.seed}\NormalTok{(}\DecValTok{123}\NormalTok{)}
\CommentTok{\# Dados simulados de temperatura}
\NormalTok{temperature\_stream }\OtherTok{\textless{}{-}} \FunctionTok{c}\NormalTok{(}\FunctionTok{rnorm}\NormalTok{(}\DecValTok{100}\NormalTok{, }\AttributeTok{mean =} \DecValTok{22}\NormalTok{, }\AttributeTok{sd =} \FloatTok{0.5}\NormalTok{), }\FunctionTok{rnorm}\NormalTok{(}\DecValTok{100}\NormalTok{, }\AttributeTok{mean =} \DecValTok{30}\NormalTok{, }\AttributeTok{sd =} \FloatTok{0.5}\NormalTok{))}

\CommentTok{\# Inicializa o detector ADWIN}
\NormalTok{adwin\_temp }\OtherTok{\textless{}{-}} \FunctionTok{ADWIN}\NormalTok{(}\AttributeTok{delta =} \FloatTok{0.002}\NormalTok{)}

\CommentTok{\# Vetor para armazenar os pontos de detecção de mudança}
\NormalTok{change\_points\_temp }\OtherTok{\textless{}{-}} \FunctionTok{numeric}\NormalTok{(}\DecValTok{0}\NormalTok{)}

\CommentTok{\# Processa o fluxo de dados}
\ControlFlowTok{for}\NormalTok{ (i }\ControlFlowTok{in} \DecValTok{1}\SpecialCharTok{:}\FunctionTok{length}\NormalTok{(temperature\_stream)) \{}
  \ControlFlowTok{if}\NormalTok{ (adwin\_temp}\SpecialCharTok{$}\FunctionTok{update}\NormalTok{(temperature\_stream[i])) \{}
\NormalTok{    change\_points\_temp }\OtherTok{\textless{}{-}} \FunctionTok{c}\NormalTok{(change\_points\_temp, i)}
\NormalTok{  \}}
\NormalTok{\}}

\CommentTok{\# Plotar os dados e os pontos de mudança}
\NormalTok{df\_temp }\OtherTok{\textless{}{-}} \FunctionTok{data.frame}\NormalTok{(}
  \AttributeTok{index =} \DecValTok{1}\SpecialCharTok{:}\FunctionTok{length}\NormalTok{(temperature\_stream),}
  \AttributeTok{value =}\NormalTok{ temperature\_stream,}
  \AttributeTok{change =} \FunctionTok{ifelse}\NormalTok{(}\DecValTok{1}\SpecialCharTok{:}\FunctionTok{length}\NormalTok{(temperature\_stream) }\SpecialCharTok{\%in\%}\NormalTok{ change\_points\_temp, }\StringTok{"Change Detected"}\NormalTok{, }\StringTok{"No Change"}\NormalTok{)}
\NormalTok{)}

\FunctionTok{ggplot}\NormalTok{(df\_temp, }\FunctionTok{aes}\NormalTok{(}\AttributeTok{x =}\NormalTok{ index, }\AttributeTok{y =}\NormalTok{ value)) }\SpecialCharTok{+}
  \FunctionTok{geom\_line}\NormalTok{() }\SpecialCharTok{+}
  \FunctionTok{geom\_point}\NormalTok{(}\AttributeTok{data =} \FunctionTok{subset}\NormalTok{(df\_temp, change }\SpecialCharTok{==} \StringTok{"Change Detected"}\NormalTok{), }\FunctionTok{aes}\NormalTok{(}\AttributeTok{x =}\NormalTok{ index, }\AttributeTok{y =}\NormalTok{ value), }\AttributeTok{color =} \StringTok{"red"}\NormalTok{, }\AttributeTok{size =} \DecValTok{2}\NormalTok{) }\SpecialCharTok{+}
  \FunctionTok{labs}\NormalTok{(}\AttributeTok{title =} \StringTok{"Detecção de Mudança na Temperatura da Sala de Servidores"}\NormalTok{, }\AttributeTok{x =} \StringTok{"Índice"}\NormalTok{, }\AttributeTok{y =} \StringTok{"Temperatura (°C)"}\NormalTok{) }\SpecialCharTok{+}
  \FunctionTok{theme\_minimal}\NormalTok{()}
\end{Highlighting}
\end{Shaded}

\includegraphics{TrickyTimeSeries_files/figure-latex/unnamed-chunk-2-1.pdf}
\emph{Resultado:}

Neste caso, o ADWIN detecta uma mudança significativa quando a
temperatura sobe de 22°C para 30°C devido à falha no sistema de ar
condicionado.

\hypertarget{drift-deriva}{%
\paragraph{Drift (Deriva)}\label{drift-deriva}}

Embora o \emph{ADWIN possa detectar mudanças abruptas, ele não é
especificamente projetado para detectar drift gradual}. No entanto, se
um drift gradual acumular uma mudança significativa que ultrapassa o
limiar, ele pode ser detectado como uma mudança.

Drift gradual é um tipo de mudança nos dados em que as propriedades
estatísticas dos dados mudam lentamente ao longo do tempo. Isso
contrasta com mudanças abruptas, onde há uma mudança repentina e
significativa nas propriedades dos dados.

\emph{Características do Drift Gradual}

\begin{enumerate}
\def\labelenumi{\arabic{enumi}.}
\item
  \emph{Lento e Contínuo}: O drift gradual ocorre de forma lenta e
  contínua, em vez de uma mudança súbita.
\item
  \emph{Dificuldade de Detecção}: Pode ser mais difícil de detectar
  porque as mudanças são pequenas em cada passo, mas acumulam ao longo
  do tempo.
\item
  \emph{Acúmulo de Efeito}: Embora cada pequena mudança possa não ser
  significativa por si só, o efeito acumulado ao longo do tempo pode
  levar a uma mudança significativa que pode ser detectada.
\end{enumerate}

\begin{itemize}
\item
  Embora o \emph{ADWIN} seja eficaz na detecção de mudanças abruptas,
  ele pode não ser tão eficaz na detecção de drift gradual devido à
  natureza do algoritmo, que é projetado para detectar mudanças
  estatisticamente significativas em janelas de dados. No entanto, se o
  drift gradual acumular uma mudança significativa que ultrapassa o
  limiar estatístico definido pelo delta, ele pode ser detectado como
  uma mudança.
\item
  Mudanças Abruptas: ADWIN é bom para detectar mudanças abruptas porque
  essas mudanças causam uma diferença estatisticamente significativa na
  janela de dados, o que aciona a detecção.
\item
  Drift Gradual: ADWIN não é projetado especificamente para drift
  gradual porque pequenas mudanças em cada passo podem não causar uma
  diferença significativa na janela de dados. No entanto, se essas
  pequenas mudanças se acumularem ao longo do tempo a ponto de causar
  uma diferença estatisticamente significativa, ADWIN ainda pode
  detectar essa mudança acumulada como uma mudança.
\end{itemize}

\emph{Exemplo de Drift:}

Digamos que estamos monitorando a demanda de energia elétrica em um
bairro ao longo de um ano. No verão, a demanda de energia aumenta
gradualmente devido ao maior uso de ar condicionado.

\begin{Shaded}
\begin{Highlighting}[]
\FunctionTok{set.seed}\NormalTok{(}\DecValTok{123}\NormalTok{)}
\CommentTok{\# Dados simulados de demanda de energia}
\NormalTok{energy\_demand\_stream }\OtherTok{\textless{}{-}} \FunctionTok{c}\NormalTok{(}\FunctionTok{rnorm}\NormalTok{(}\DecValTok{100}\NormalTok{, }\AttributeTok{mean =} \DecValTok{50}\NormalTok{, }\AttributeTok{sd =} \DecValTok{2}\NormalTok{), }\FunctionTok{rnorm}\NormalTok{(}\DecValTok{100}\NormalTok{, }\AttributeTok{mean =} \DecValTok{60}\NormalTok{, }\AttributeTok{sd =} \DecValTok{2}\NormalTok{))}

\CommentTok{\# Inicializa o detector ADWIN}
\NormalTok{adwin\_energy }\OtherTok{\textless{}{-}} \FunctionTok{ADWIN}\NormalTok{(}\AttributeTok{delta =} \FloatTok{0.002}\NormalTok{)}

\CommentTok{\# Vetor para armazenar os pontos de detecção de mudança}
\NormalTok{change\_points\_energy }\OtherTok{\textless{}{-}} \FunctionTok{numeric}\NormalTok{(}\DecValTok{0}\NormalTok{)}

\CommentTok{\# Processa o fluxo de dados}
\ControlFlowTok{for}\NormalTok{ (i }\ControlFlowTok{in} \DecValTok{1}\SpecialCharTok{:}\FunctionTok{length}\NormalTok{(energy\_demand\_stream)) \{}
  \ControlFlowTok{if}\NormalTok{ (adwin\_energy}\SpecialCharTok{$}\FunctionTok{update}\NormalTok{(energy\_demand\_stream[i])) \{}
\NormalTok{    change\_points\_energy }\OtherTok{\textless{}{-}} \FunctionTok{c}\NormalTok{(change\_points\_energy, i)}
\NormalTok{  \}}
\NormalTok{\}}

\CommentTok{\# Plotar os dados e os pontos de mudança}
\NormalTok{df\_energy }\OtherTok{\textless{}{-}} \FunctionTok{data.frame}\NormalTok{(}
  \AttributeTok{index =} \DecValTok{1}\SpecialCharTok{:}\FunctionTok{length}\NormalTok{(energy\_demand\_stream),}
  \AttributeTok{value =}\NormalTok{ energy\_demand\_stream,}
  \AttributeTok{change =} \FunctionTok{ifelse}\NormalTok{(}\DecValTok{1}\SpecialCharTok{:}\FunctionTok{length}\NormalTok{(energy\_demand\_stream) }\SpecialCharTok{\%in\%}\NormalTok{ change\_points\_energy, }\StringTok{"Change Detected"}\NormalTok{, }\StringTok{"No Change"}\NormalTok{)}
\NormalTok{)}

\FunctionTok{ggplot}\NormalTok{(df\_energy, }\FunctionTok{aes}\NormalTok{(}\AttributeTok{x =}\NormalTok{ index, }\AttributeTok{y =}\NormalTok{ value)) }\SpecialCharTok{+}
  \FunctionTok{geom\_line}\NormalTok{() }\SpecialCharTok{+}
  \FunctionTok{geom\_point}\NormalTok{(}\AttributeTok{data =} \FunctionTok{subset}\NormalTok{(df\_energy, change }\SpecialCharTok{==} \StringTok{"Change Detected"}\NormalTok{), }\FunctionTok{aes}\NormalTok{(}\AttributeTok{x =}\NormalTok{ index, }\AttributeTok{y =}\NormalTok{ value), }\AttributeTok{color =} \StringTok{"red"}\NormalTok{, }\AttributeTok{size =} \DecValTok{2}\NormalTok{) }\SpecialCharTok{+}
  \FunctionTok{labs}\NormalTok{(}\AttributeTok{title =} \StringTok{"Energy Demand Change Detection"}\NormalTok{, }\AttributeTok{x =} \StringTok{"Índice"}\NormalTok{, }\AttributeTok{y =} \StringTok{"Demanda de Energia (kW)"}\NormalTok{) }\SpecialCharTok{+}
  \FunctionTok{theme\_minimal}\NormalTok{()}
\end{Highlighting}
\end{Shaded}

\includegraphics{TrickyTimeSeries_files/figure-latex/unnamed-chunk-3-1.pdf}
\emph{Resultado:}

Neste caso, a demanda de energia aumenta gradualmente. O ADWIN pode
detectar uma mudança significativa quando a demanda acumulada ao longo
do tempo ultrapassa o limiar, indicando um drift.

\begin{itemize}
\item
  \textbf{Change Detected}: Exemplifica mudanças abruptas e
  significativas nos dados, como uma falha no sistema de ar condicionado
  que causa um aumento súbito na temperatura.
\item
  \textbf{Drift}: Representa mudanças graduais nos dados, que podem
  eventualmente ser detectadas como uma mudança significativa, como o
  aumento gradual da demanda de energia durante o verão.
\end{itemize}

Ambos os exemplos ilustram como o ADWIN pode ser utilizado para detectar
diferentes tipos de mudanças em fluxos de dados, oferecendo uma
abordagem robusta para monitoramento e análise. Além disso, o uso de
padrões ordinais de Bandt e Pompe em conjunto com o ADWIN melhorou a
capacidade de detecção de mudanças, tornando o algoritmo mais sensível a
variações estruturais. Isso é particularmente útil em cenários onde as
mudanças não são apenas de amplitude, mas também de comportamento.

\begin{center}\rule{0.5\linewidth}{0.5pt}\end{center}

\hypertarget{algoritmo-adwin-por-bifet}{%
\subsubsection{Algoritmo ADWIN por
Bifet}\label{algoritmo-adwin-por-bifet}}

O ADWIN é um algoritmo que ajusta dinamicamente o tamanho de uma janela
deslizante de dados e verifica se houve uma mudança significativa na
distribuição dos dados.

\hypertarget{funcionamento-do-adwin}{%
\paragraph{Funcionamento do ADWIN}\label{funcionamento-do-adwin}}

\begin{enumerate}
\def\labelenumi{\arabic{enumi}.}
\item
  \textbf{Janela Deslizante Dinâmica:} ADWIN mantém uma janela de dados
  que pode aumentar ou diminuir de tamanho conforme necessário para
  detectar mudanças.
\item
  \textbf{Divisão da Janela:} A janela é dividida em duas sub-janelas:
  \(W_0\) e \(W_1\). A média das duas sub-janelas é comparada para
  verificar mudanças.
\item
  \textbf{Teste de Mudança:} Se a diferença entre as médias das
  sub-janelas exceder um limite estatístico baseado na variância e no
  tamanho das janelas, ADWIN conclui que houve uma mudança.
\item
  \textbf{Ajuste da Janela:} Quando uma mudança é detectada, ADWIN reduz
  a janela para excluir os dados antigos e manter os novos dados que
  representam o novo conceito.
\end{enumerate}

\hypertarget{cenuxe1rio-1-detecuxe7uxe3o-de-estabilidade}{%
\paragraph{Cenário 1: Detecção de
Estabilidade}\label{cenuxe1rio-1-detecuxe7uxe3o-de-estabilidade}}

Suponha que estamos monitorando um fluxo de dados de temperatura.

\begin{enumerate}
\def\labelenumi{\arabic{enumi}.}
\tightlist
\item
  \textbf{Inicialização:} Os primeiros dados de temperatura são
  inseridos na janela: \[
  W = [20, 21, 19, 20, 21, 20, 21, 20]
  \]
\item
  \textbf{Divisão da Janela:} A janela é dividida em duas sub-janelas:
  \[
  W_0 = [20, 21, 19, 20] \quad \text{e} \quad W_1 = [21, 20, 21, 20]
  \]
\item
  \textbf{Cálculo das Médias:} \[
  \text{Média}(W_0) = 20 \quad \text{e} \quad \text{Média}(W_1) = 20.5
  \]
\item
  \textbf{Teste de Mudança:} A diferença entre as médias é pequena,
  então nenhuma mudança é detectada.
\end{enumerate}

\hypertarget{cenuxe1rio-2-detecuxe7uxe3o-de-mudanuxe7a}{%
\subparagraph{Cenário 2: Detecção de
Mudança}\label{cenuxe1rio-2-detecuxe7uxe3o-de-mudanuxe7a}}

Agora, os dados mudam devido a uma alteração no ambiente:

\begin{enumerate}
\def\labelenumi{\arabic{enumi}.}
\tightlist
\item
  \textbf{Novos Dados:} Dados adicionais mostram um aumento de
  temperatura: \[
  W = [20, 21, 19, 20, 21, 20, 21, 20, 25, 26, 25, 27]
  \]
\item
  \textbf{Divisão da Janela:} A janela é dividida em duas sub-janelas:
  \[
  W_0 = [20, 21, 19, 20, 21, 20] \quad \text{e} \quad W_1 = [21, 20, 25, 26, 25, 27]
  \]
\item
  \textbf{Cálculo das Médias:} \[
  \text{Média}(W_0) = 20.1667 \quad \text{e} \quad \text{Média}(W_1) = 24
  \]
\item
  \textbf{Teste de Mudança:} A diferença entre as médias é
  significativa, indicando uma mudança de conceito. ADWIN ajusta a
  janela para: \[
  W = [25, 26, 25, 27]
  \]
\end{enumerate}

\hypertarget{implementauxe7uxe3o-em-r}{%
\paragraph{Implementação em R}\label{implementauxe7uxe3o-em-r}}

\begin{Shaded}
\begin{Highlighting}[]
\DocumentationTok{\#\# O pacote stream oferece funcionalidades para a detecção de concept drift, incluindo uma implementação do ADWIN.}

\CommentTok{\# A função ADWIN é projetada para detectar mudanças de conceito (concept drift) em fluxos de dados (data streams). Ela ajusta dinamicamente o tamanho de uma janela deslizante de dados e verifica se houve uma mudança significativa na distribuição dos dados.}

\CommentTok{\# Neste exemplo, o ADWIN detectará a mudança quando os dados passarem de 20 para 25 e ajustará a janela para refletir o novo conceito}

\CommentTok{\# ADWIN (Adaptive Windowing): Este algoritmo monitora uma janela de dados em constante mudança. Quando os dados mais recentes na janela diferem estatisticamente dos dados mais antigos, uma mudança é detectada. O parâmetro delta controla a sensibilidade do detector.}
\NormalTok{ADWIN }\OtherTok{\textless{}{-}} \ControlFlowTok{function}\NormalTok{(}\AttributeTok{delta =} \FloatTok{0.002}\NormalTok{) \{}
  \CommentTok{\# Inicializa as variáveis}
\NormalTok{  width }\OtherTok{\textless{}{-}} \DecValTok{0} \CommentTok{\# Tamanho da janela}
\NormalTok{  total }\OtherTok{\textless{}{-}} \DecValTok{0} \CommentTok{\# Soma dos valores na janela}
\NormalTok{  variance }\OtherTok{\textless{}{-}} \DecValTok{0} \CommentTok{\# Variância dos valores na janela}
\NormalTok{  window }\OtherTok{\textless{}{-}} \FunctionTok{numeric}\NormalTok{(}\DecValTok{0}\NormalTok{) }\CommentTok{\# Vetor que armazena os valores na janela}
  
  \CommentTok{\# Função para atualizar o ADWIN com um novo valor}
  \CommentTok{\# Função update: Adiciona um novo valor à janela e verifica se houve uma mudança significativa nos dados.}
\NormalTok{  update }\OtherTok{\textless{}{-}} \ControlFlowTok{function}\NormalTok{(value) \{}
\NormalTok{    width }\OtherTok{\textless{}\textless{}{-}}\NormalTok{ width }\SpecialCharTok{+} \DecValTok{1} \CommentTok{\# Incrementa o tamanho da janela}
\NormalTok{    window }\OtherTok{\textless{}\textless{}{-}} \FunctionTok{c}\NormalTok{(window, value) }\CommentTok{\# Adiciona o novo valor à janela}
\NormalTok{    total }\OtherTok{\textless{}\textless{}{-}}\NormalTok{ total }\SpecialCharTok{+}\NormalTok{ value }\CommentTok{\# Atualiza a soma total}
    \ControlFlowTok{if}\NormalTok{ (width }\SpecialCharTok{\textgreater{}} \DecValTok{1}\NormalTok{) \{}
\NormalTok{      variance }\OtherTok{\textless{}\textless{}{-}} \FunctionTok{var}\NormalTok{(window) }\CommentTok{\# Calcula a variância se a janela tiver mais de um valor}
\NormalTok{    \}}
    \CommentTok{\# Checa por concept drift}
    \CommentTok{\# Função detect\_change: Verifica se a média dos dados mais recentes na janela é significativamente diferente da média dos       dados antigos, ajustando a janela conforme necessário para manter a sensibilidade às mudanças.}
    \ControlFlowTok{if}\NormalTok{ (width }\SpecialCharTok{\textgreater{}} \DecValTok{1} \SpecialCharTok{\&\&} \FunctionTok{detect\_change}\NormalTok{()) \{}
      \FunctionTok{return}\NormalTok{(}\ConstantTok{TRUE}\NormalTok{) }\CommentTok{\# Retorna TRUE se uma mudança for detectada}
\NormalTok{    \}}
    \FunctionTok{return}\NormalTok{(}\ConstantTok{FALSE}\NormalTok{) }\CommentTok{\# Retorna FALSE se nenhuma mudança for detectada}
\NormalTok{  \}}
  
  \CommentTok{\# Função para detectar mudança}
\NormalTok{  detect\_change }\OtherTok{\textless{}{-}} \ControlFlowTok{function}\NormalTok{() \{}
\NormalTok{    mean\_val }\OtherTok{\textless{}{-}} \FunctionTok{mean}\NormalTok{(window) }\CommentTok{\# Calcula a média dos valores na janela}
    \ControlFlowTok{for}\NormalTok{ (n }\ControlFlowTok{in} \DecValTok{1}\SpecialCharTok{:}\NormalTok{(width }\SpecialCharTok{{-}} \DecValTok{1}\NormalTok{)) \{}
      \CommentTok{\# Divide a janela em duas sub{-}janelas e calcula as médias de cada sub{-}janela}
\NormalTok{      mean0 }\OtherTok{\textless{}{-}} \FunctionTok{mean}\NormalTok{(window[}\DecValTok{1}\SpecialCharTok{:}\NormalTok{n])}
\NormalTok{      mean1 }\OtherTok{\textless{}{-}} \FunctionTok{mean}\NormalTok{(window[(n }\SpecialCharTok{+} \DecValTok{1}\NormalTok{)}\SpecialCharTok{:}\NormalTok{width])}
      \CommentTok{\# Compara as médias das sub{-}janelas usando um teste estatístico}
      \ControlFlowTok{if}\NormalTok{ (}\FunctionTok{abs}\NormalTok{(mean0 }\SpecialCharTok{{-}}\NormalTok{ mean1) }\SpecialCharTok{\textgreater{}} \FunctionTok{sqrt}\NormalTok{((variance }\SpecialCharTok{/}\NormalTok{ n) }\SpecialCharTok{+}\NormalTok{ (variance }\SpecialCharTok{/}\NormalTok{ (width }\SpecialCharTok{{-}}\NormalTok{ n))) }\SpecialCharTok{*} \FunctionTok{qnorm}\NormalTok{(}\DecValTok{1} \SpecialCharTok{{-}}\NormalTok{ delta)) \{}
        \CommentTok{\# Se uma mudança for detectada, ajusta a janela para descartar os dados antigos}
\NormalTok{        window }\OtherTok{\textless{}\textless{}{-}}\NormalTok{ window[(n }\SpecialCharTok{+} \DecValTok{1}\NormalTok{)}\SpecialCharTok{:}\NormalTok{width]}
\NormalTok{        width }\OtherTok{\textless{}\textless{}{-}} \FunctionTok{length}\NormalTok{(window)}
\NormalTok{        total }\OtherTok{\textless{}\textless{}{-}} \FunctionTok{sum}\NormalTok{(window)}
\NormalTok{        variance }\OtherTok{\textless{}\textless{}{-}} \FunctionTok{var}\NormalTok{(window)}
        \FunctionTok{return}\NormalTok{(}\ConstantTok{TRUE}\NormalTok{) }\CommentTok{\# Retorna TRUE indicando que uma mudança foi detectada}
\NormalTok{      \}}
\NormalTok{    \}}
    \FunctionTok{return}\NormalTok{(}\ConstantTok{FALSE}\NormalTok{) }\CommentTok{\# Retorna FALSE se nenhuma mudança for detectada}
\NormalTok{  \}}
  
  \CommentTok{\# Retorna as funções de atualização e detecção}
  \FunctionTok{list}\NormalTok{(}\AttributeTok{update =}\NormalTok{ update)}
\NormalTok{\}}

\CommentTok{\# Exemplo de uso com visualização}
\FunctionTok{set.seed}\NormalTok{(}\DecValTok{123}\NormalTok{) }\CommentTok{\# Define a semente para reprodutibilidade}
\NormalTok{data\_stream }\OtherTok{\textless{}{-}} \FunctionTok{c}\NormalTok{(}\FunctionTok{rnorm}\NormalTok{(}\DecValTok{100}\NormalTok{, }\AttributeTok{mean =} \DecValTok{20}\NormalTok{), }\FunctionTok{rnorm}\NormalTok{(}\DecValTok{100}\NormalTok{, }\AttributeTok{mean =} \DecValTok{25}\NormalTok{)) }\CommentTok{\# Gera dados de exemplo com duas médias diferentes}
\NormalTok{adwin }\OtherTok{\textless{}{-}} \FunctionTok{ADWIN}\NormalTok{(}\AttributeTok{delta =} \FloatTok{0.002}\NormalTok{) }\CommentTok{\# Inicializa o detector ADWIN com delta = 0.002}

\CommentTok{\# Vetor para armazenar os pontos de detecção de mudança}
\NormalTok{change\_points }\OtherTok{\textless{}{-}} \FunctionTok{numeric}\NormalTok{(}\DecValTok{0}\NormalTok{)}

\CommentTok{\# Processa o fluxo de dados}
\ControlFlowTok{for}\NormalTok{ (i }\ControlFlowTok{in} \DecValTok{1}\SpecialCharTok{:}\FunctionTok{length}\NormalTok{(data\_stream)) \{}
  \ControlFlowTok{if}\NormalTok{ (adwin}\SpecialCharTok{$}\FunctionTok{update}\NormalTok{(data\_stream[i])) \{}
\NormalTok{    change\_points }\OtherTok{\textless{}{-}} \FunctionTok{c}\NormalTok{(change\_points, i) }\CommentTok{\# Armazena o índice onde a mudança foi detectada}
\NormalTok{  \}}
\NormalTok{\}}

\CommentTok{\# Plotar os dados e os pontos de mudança}
\FunctionTok{library}\NormalTok{(ggplot2)}

\CommentTok{\# Criar um data frame com os dados e os pontos de mudança}
\NormalTok{df }\OtherTok{\textless{}{-}} \FunctionTok{data.frame}\NormalTok{(}
  \AttributeTok{index =} \DecValTok{1}\SpecialCharTok{:}\FunctionTok{length}\NormalTok{(data\_stream),}
  \AttributeTok{value =}\NormalTok{ data\_stream,}
  \AttributeTok{change =} \FunctionTok{ifelse}\NormalTok{(}\DecValTok{1}\SpecialCharTok{:}\FunctionTok{length}\NormalTok{(data\_stream) }\SpecialCharTok{\%in\%}\NormalTok{ change\_points, }\StringTok{"Change Detected"}\NormalTok{, }\StringTok{"No Change"}\NormalTok{)}
\NormalTok{)}

\CommentTok{\# Plotar os dados}
\FunctionTok{ggplot}\NormalTok{(df, }\FunctionTok{aes}\NormalTok{(}\AttributeTok{x =}\NormalTok{ index, }\AttributeTok{y =}\NormalTok{ value)) }\SpecialCharTok{+}
  \FunctionTok{geom\_line}\NormalTok{() }\SpecialCharTok{+}
  \FunctionTok{geom\_point}\NormalTok{(}\AttributeTok{data =} \FunctionTok{subset}\NormalTok{(df, change }\SpecialCharTok{==} \StringTok{"Change Detected"}\NormalTok{), }\FunctionTok{aes}\NormalTok{(}\AttributeTok{x =}\NormalTok{ index, }\AttributeTok{y =}\NormalTok{ value), }\AttributeTok{color =} \StringTok{"red"}\NormalTok{, }\AttributeTok{size =} \DecValTok{2}\NormalTok{) }\SpecialCharTok{+}
  \FunctionTok{labs}\NormalTok{(}\AttributeTok{title =} \StringTok{"Change Detected with ADWIN"}\NormalTok{, }\AttributeTok{x =} \StringTok{"Índice"}\NormalTok{, }\AttributeTok{y =} \StringTok{"Valor"}\NormalTok{) }\SpecialCharTok{+}
  \FunctionTok{theme\_minimal}\NormalTok{()}
\end{Highlighting}
\end{Shaded}

\includegraphics{TrickyTimeSeries_files/figure-latex/unnamed-chunk-4-1.pdf}

\begin{Shaded}
\begin{Highlighting}[]
\CommentTok{\# Este processo detecta quando os dados passam de uma média de 20 para 25, ajustando a janela deslizante para refletir o novo conceito.}
\end{Highlighting}
\end{Shaded}

\hypertarget{teste-estatuxedstico-ajustado}{%
\paragraph{Teste Estatístico
Ajustado:}\label{teste-estatuxedstico-ajustado}}

\begin{itemize}
\tightlist
\item
  O teste estatístico foi ajustado para comparar as médias das
  sub-janelas usando a fórmula correta para a comparação de médias com
  variâncias desconhecidas.
\item
  A comparação agora usa a soma das variâncias divididas pelo tamanho
  das sub-janelas para calcular o valor crítico.
\end{itemize}

\hypertarget{ajuste-da-janela}{%
\paragraph{Ajuste da Janela:}\label{ajuste-da-janela}}

\begin{itemize}
\tightlist
\item
  Quando uma mudança é detectada, a janela é ajustada corretamente,
  descartando os dados antigos e recalculando a variância da nova
  janela.
\end{itemize}

\hypertarget{usando-op}{%
\subsubsection{Usando OP}\label{usando-op}}

\begin{Shaded}
\begin{Highlighting}[]
\CommentTok{\# Carregar bibliotecas necessárias}
\FunctionTok{library}\NormalTok{(pracma)}
\FunctionTok{library}\NormalTok{(ggplot2)}
\FunctionTok{library}\NormalTok{(ggthemes)}
\FunctionTok{library}\NormalTok{(statcomp)}

\CommentTok{\# Função para calcular padrões ordinais usando statcomp}
\NormalTok{ordinal\_patterns\_statcomp }\OtherTok{\textless{}{-}} \ControlFlowTok{function}\NormalTok{(series, emb\_dim) \{}
  \ControlFlowTok{if}\NormalTok{ (}\FunctionTok{length}\NormalTok{(series) }\SpecialCharTok{\textless{}}\NormalTok{ emb\_dim) \{}
    \FunctionTok{stop}\NormalTok{(}\StringTok{"A série temporal é muito curta para a dimensão de embedding especificada."}\NormalTok{)}
\NormalTok{  \}}
  
  \CommentTok{\# Utilizando a função from \textasciigrave{}statcomp\textasciigrave{} para calcular os padrões ordinais}
\NormalTok{  patterns }\OtherTok{\textless{}{-}} \FunctionTok{ordinal\_pattern}\NormalTok{(series, emb\_dim)}
  
  \FunctionTok{return}\NormalTok{(patterns)}
\NormalTok{\}}

\CommentTok{\# Função ADWIN ajustada para padrões ordinais}
\NormalTok{ADWIN }\OtherTok{\textless{}{-}} \ControlFlowTok{function}\NormalTok{(}\AttributeTok{delta =} \FloatTok{0.002}\NormalTok{) \{}
\NormalTok{  width }\OtherTok{\textless{}{-}} \DecValTok{0}
\NormalTok{  total }\OtherTok{\textless{}{-}} \DecValTok{0}
\NormalTok{  variance }\OtherTok{\textless{}{-}} \DecValTok{0}
\NormalTok{  window }\OtherTok{\textless{}{-}} \FunctionTok{numeric}\NormalTok{(}\DecValTok{0}\NormalTok{)}
  
\NormalTok{  update }\OtherTok{\textless{}{-}} \ControlFlowTok{function}\NormalTok{(value) \{}
\NormalTok{    width }\OtherTok{\textless{}\textless{}{-}}\NormalTok{ width }\SpecialCharTok{+} \DecValTok{1}
\NormalTok{    window }\OtherTok{\textless{}\textless{}{-}} \FunctionTok{c}\NormalTok{(window, value)}
\NormalTok{    total }\OtherTok{\textless{}\textless{}{-}}\NormalTok{ total }\SpecialCharTok{+}\NormalTok{ value}
    \ControlFlowTok{if}\NormalTok{ (width }\SpecialCharTok{\textgreater{}} \DecValTok{1}\NormalTok{) \{}
\NormalTok{      variance }\OtherTok{\textless{}\textless{}{-}} \FunctionTok{var}\NormalTok{(window, }\AttributeTok{na.rm =} \ConstantTok{TRUE}\NormalTok{) }\CommentTok{\# Calcula a variância, removendo NAs}
\NormalTok{    \}}
    \ControlFlowTok{if}\NormalTok{ (width }\SpecialCharTok{\textgreater{}} \DecValTok{1} \SpecialCharTok{\&\&} \FunctionTok{detect\_change}\NormalTok{()) \{}
      \FunctionTok{return}\NormalTok{(}\ConstantTok{TRUE}\NormalTok{)}
\NormalTok{    \}}
    \FunctionTok{return}\NormalTok{(}\ConstantTok{FALSE}\NormalTok{)}
\NormalTok{  \}}
  
\NormalTok{  detect\_change }\OtherTok{\textless{}{-}} \ControlFlowTok{function}\NormalTok{() \{}
\NormalTok{    mean\_val }\OtherTok{\textless{}{-}} \FunctionTok{mean}\NormalTok{(window, }\AttributeTok{na.rm =} \ConstantTok{TRUE}\NormalTok{) }\CommentTok{\# Calcula a média, removendo NAs}
    \ControlFlowTok{for}\NormalTok{ (n }\ControlFlowTok{in} \DecValTok{1}\SpecialCharTok{:}\NormalTok{(width }\SpecialCharTok{{-}} \DecValTok{1}\NormalTok{)) \{}
\NormalTok{      mean0 }\OtherTok{\textless{}{-}} \FunctionTok{mean}\NormalTok{(window[}\DecValTok{1}\SpecialCharTok{:}\NormalTok{n], }\AttributeTok{na.rm =} \ConstantTok{TRUE}\NormalTok{)}
\NormalTok{      mean1 }\OtherTok{\textless{}{-}} \FunctionTok{mean}\NormalTok{(window[(n }\SpecialCharTok{+} \DecValTok{1}\NormalTok{)}\SpecialCharTok{:}\NormalTok{width], }\AttributeTok{na.rm =} \ConstantTok{TRUE}\NormalTok{)}
      \ControlFlowTok{if}\NormalTok{ (}\FunctionTok{is.na}\NormalTok{(mean0) }\SpecialCharTok{||} \FunctionTok{is.na}\NormalTok{(mean1)) }\ControlFlowTok{next} \CommentTok{\# Pula iteração se a média for NA}
      \ControlFlowTok{if}\NormalTok{ (}\FunctionTok{abs}\NormalTok{(mean0 }\SpecialCharTok{{-}}\NormalTok{ mean1) }\SpecialCharTok{\textgreater{}} \FunctionTok{sqrt}\NormalTok{((variance }\SpecialCharTok{/}\NormalTok{ n) }\SpecialCharTok{+}\NormalTok{ (variance }\SpecialCharTok{/}\NormalTok{ (width }\SpecialCharTok{{-}}\NormalTok{ n))) }\SpecialCharTok{*} \FunctionTok{qnorm}\NormalTok{(}\DecValTok{1} \SpecialCharTok{{-}}\NormalTok{ delta)) \{}
\NormalTok{        window }\OtherTok{\textless{}\textless{}{-}}\NormalTok{ window[(n }\SpecialCharTok{+} \DecValTok{1}\NormalTok{)}\SpecialCharTok{:}\NormalTok{width]}
\NormalTok{        width }\OtherTok{\textless{}\textless{}{-}} \FunctionTok{length}\NormalTok{(window)}
\NormalTok{        total }\OtherTok{\textless{}\textless{}{-}} \FunctionTok{sum}\NormalTok{(window, }\AttributeTok{na.rm =} \ConstantTok{TRUE}\NormalTok{)}
\NormalTok{        variance }\OtherTok{\textless{}\textless{}{-}} \FunctionTok{var}\NormalTok{(window, }\AttributeTok{na.rm =} \ConstantTok{TRUE}\NormalTok{)}
        \FunctionTok{return}\NormalTok{(}\ConstantTok{TRUE}\NormalTok{)}
\NormalTok{      \}}
\NormalTok{    \}}
    \FunctionTok{return}\NormalTok{(}\ConstantTok{FALSE}\NormalTok{)}
\NormalTok{  \}}
  
  \FunctionTok{list}\NormalTok{(}\AttributeTok{update =}\NormalTok{ update)}
\NormalTok{\}}

\CommentTok{\# Exemplo de uso com visualização}
\FunctionTok{set.seed}\NormalTok{(}\DecValTok{123}\NormalTok{) }\CommentTok{\# Define a semente para reprodutibilidade}
\NormalTok{data\_stream }\OtherTok{\textless{}{-}} \FunctionTok{c}\NormalTok{(}\FunctionTok{rnorm}\NormalTok{(}\DecValTok{100}\NormalTok{, }\AttributeTok{mean =} \DecValTok{20}\NormalTok{), }\FunctionTok{rnorm}\NormalTok{(}\DecValTok{100}\NormalTok{, }\AttributeTok{mean =} \DecValTok{25}\NormalTok{)) }\CommentTok{\# Gera dados de exemplo com duas médias diferentes}

\CommentTok{\# Calcular padrões ordinais}
\NormalTok{patterns }\OtherTok{\textless{}{-}} \FunctionTok{ordinal\_patterns}\NormalTok{(data\_stream, }\AttributeTok{emb\_dim =} \DecValTok{3}\NormalTok{)}

\CommentTok{\# Inicializa o detector ADWIN}
\NormalTok{adwin }\OtherTok{\textless{}{-}} \FunctionTok{ADWIN}\NormalTok{(}\AttributeTok{delta =} \FloatTok{0.002}\NormalTok{)}

\CommentTok{\# Vetor para armazenar os pontos de detecção de mudança}
\NormalTok{change\_points }\OtherTok{\textless{}{-}} \FunctionTok{numeric}\NormalTok{(}\DecValTok{0}\NormalTok{)}

\CommentTok{\# Processa o fluxo de dados usando padrões ordinais}
\ControlFlowTok{for}\NormalTok{ (i }\ControlFlowTok{in} \DecValTok{1}\SpecialCharTok{:}\FunctionTok{length}\NormalTok{(patterns)) \{}
  \ControlFlowTok{if}\NormalTok{ (adwin}\SpecialCharTok{$}\FunctionTok{update}\NormalTok{(patterns[i])) \{}
\NormalTok{    change\_points }\OtherTok{\textless{}{-}} \FunctionTok{c}\NormalTok{(change\_points, i }\SpecialCharTok{+} \DecValTok{2}\NormalTok{) }\CommentTok{\# Ajuste do índice devido ao embedding dimension}
\NormalTok{  \}}
\NormalTok{\}}

\CommentTok{\# Plotar os dados e os pontos de mudança}
\NormalTok{df }\OtherTok{\textless{}{-}} \FunctionTok{data.frame}\NormalTok{(}
  \AttributeTok{index =} \DecValTok{1}\SpecialCharTok{:}\FunctionTok{length}\NormalTok{(data\_stream),}
  \AttributeTok{value =}\NormalTok{ data\_stream,}
  \AttributeTok{change =} \FunctionTok{ifelse}\NormalTok{(}\DecValTok{1}\SpecialCharTok{:}\FunctionTok{length}\NormalTok{(data\_stream) }\SpecialCharTok{\%in\%}\NormalTok{ change\_points, }\StringTok{"Change Detected"}\NormalTok{, }\StringTok{"No Change"}\NormalTok{)}
\NormalTok{)}

\CommentTok{\# Plotar os dados}
\FunctionTok{ggplot}\NormalTok{(df, }\FunctionTok{aes}\NormalTok{(}\AttributeTok{x =}\NormalTok{ index, }\AttributeTok{y =}\NormalTok{ value)) }\SpecialCharTok{+}
  \FunctionTok{geom\_line}\NormalTok{() }\SpecialCharTok{+}
  \FunctionTok{geom\_point}\NormalTok{(}\AttributeTok{data =} \FunctionTok{subset}\NormalTok{(df, change }\SpecialCharTok{==} \StringTok{"Change Detected"}\NormalTok{), }\FunctionTok{aes}\NormalTok{(}\AttributeTok{x =}\NormalTok{ index, }\AttributeTok{y =}\NormalTok{ value), }\AttributeTok{color =} \StringTok{"blue"}\NormalTok{, }\AttributeTok{size =} \DecValTok{2}\NormalTok{) }\SpecialCharTok{+}
  \FunctionTok{labs}\NormalTok{(}\AttributeTok{title =} \StringTok{"Change Detected with ADWIN and OP"}\NormalTok{, }\AttributeTok{x =} \StringTok{"Índice"}\NormalTok{, }\AttributeTok{y =} \StringTok{"Valor"}\NormalTok{) }\SpecialCharTok{+}
  \FunctionTok{theme\_minimal}\NormalTok{() }\SpecialCharTok{+}
  \FunctionTok{theme\_tufte}\NormalTok{()}
\end{Highlighting}
\end{Shaded}

\includegraphics{TrickyTimeSeries_files/figure-latex/unnamed-chunk-5-1.pdf}
\#\# Comparação dos Gráficos

\hypertarget{sem-op-primeira-imagem}{%
\subsubsection{Sem OP (Primeira Imagem):}\label{sem-op-primeira-imagem}}

\begin{itemize}
\tightlist
\item
  O ADWIN detectou mudanças principalmente após a transição dos dados de
  uma média de 20 para 25.
\item
  Há algumas detecções de mudança adicionais em torno dos pontos onde há
  variações mais abruptas.
\end{itemize}

\hypertarget{com-op-segunda-imagem}{%
\subsubsection{Com OP (Segunda Imagem):}\label{com-op-segunda-imagem}}

\begin{itemize}
\item
  O ADWIN detectou mudanças em alguns pontos antes da transição
  principal. Isso pode indicar que o uso de padrões ordinais aumentou a
  sensibilidade do ADWIN, detectando mudanças mais sutis ou mudanças na
  estrutura dos dados.
\item
  No entanto, a detecção da transição principal (em torno do ponto 100)
  parece menos pronunciada, o que pode ser uma área a ser melhorada.
\end{itemize}

\hypertarget{sensibilidade-uxe0-mudanuxe7a}{%
\subsubsection{Sensibilidade à
Mudança:}\label{sensibilidade-uxe0-mudanuxe7a}}

\begin{itemize}
\item
  \textbf{Sem OP}: O ADWIN detecta mudanças principalmente nas áreas
  onde há grandes variações nos valores dos dados, o ADWIN é sensível a
  mudanças na média e variância dos dados.
\item
  \textbf{Com OP}: O uso de padrões ordinais permite que o ADWIN detecte
  mudanças mais sutis na estrutura dos dados.
\end{itemize}

A utilização de padrões ordinais adiciona uma camada adicional que pode
capturar mudanças sutis na ordem dos dados, algo que a análise direta
dos valores pode não captar. No entanto, isso pode levar a uma
sensibilidade aumentada, \emph{o que pode resultar em detecções de
mudanças que não são imediatamente óbvias pela análise dos valores
brutos}

\hypertarget{ajustar-os-paruxe2metros}{%
\subsubsection{Ajustar os parâmetros}\label{ajustar-os-paruxe2metros}}

\begin{itemize}
\item
  Ajustar os parâmetros \texttt{delta} e \texttt{emb\_dim} para otimizar
  a detecção de mudanças. Valores diferentes podem melhorar a precisão e
  reduzir falsos positivos.
\item
  Ajustado os parâmetros \texttt{delta} e \texttt{emb\_dim} e realizar
  uma análise mais detalhada dos pontos de mudança detectados, incluir a
  capacidade de variar esses parâmetros e inspeccionar os padrões
  ordinais antes e após os pontos de mudança.
\end{itemize}

\hypertarget{anuxe1lise-detalhada}{%
\subsubsection{Análise Detalhada:}\label{anuxe1lise-detalhada}}

\begin{itemize}
\tightlist
\item
  Realize uma análise mais detalhada dos pontos de mudança detectados
  para entender melhor as causas das detecções. Isso pode incluir a
  inspeção dos padrões ordinais antes e após os pontos de mudança.
\end{itemize}

O uso de padrões ordinais com o ADWIN oferece uma abordagem mais robusta
para a detecção de mudanças, capturando variações na estrutura dos dados
que podem não ser evidentes pela análise direta dos valores. Ajustes
adicionais e uma análise mais detalhada são recomendados para otimizar a
metodologia e validar sua eficácia em diferentes contextos.

\hypertarget{anuxe1lise-detalhada-dos-pontos-de-mudanuxe7a-sem-op}{%
\section{Análise Detalhada dos Pontos de Mudança Sem
OP}\label{anuxe1lise-detalhada-dos-pontos-de-mudanuxe7a-sem-op}}

\begin{Shaded}
\begin{Highlighting}[]
\CommentTok{\# Função ADWIN para detectar mudanças de conceito em fluxos de dados}
\NormalTok{ADWIN }\OtherTok{\textless{}{-}} \ControlFlowTok{function}\NormalTok{(}\AttributeTok{delta =} \FloatTok{0.002}\NormalTok{) \{}
  \CommentTok{\# Inicializa as variáveis}
\NormalTok{  width }\OtherTok{\textless{}{-}} \DecValTok{0} \CommentTok{\# Tamanho da janela}
\NormalTok{  total }\OtherTok{\textless{}{-}} \DecValTok{0} \CommentTok{\# Soma dos valores na janela}
\NormalTok{  variance }\OtherTok{\textless{}{-}} \DecValTok{0} \CommentTok{\# Variância dos valores na janela}
\NormalTok{  window }\OtherTok{\textless{}{-}} \FunctionTok{numeric}\NormalTok{(}\DecValTok{0}\NormalTok{) }\CommentTok{\# Vetor que armazena os valores na janela}
  
  \CommentTok{\# Função para atualizar o ADWIN com um novo valor}
\NormalTok{  update }\OtherTok{\textless{}{-}} \ControlFlowTok{function}\NormalTok{(value) \{}
\NormalTok{    width }\OtherTok{\textless{}\textless{}{-}}\NormalTok{ width }\SpecialCharTok{+} \DecValTok{1} \CommentTok{\# Incrementa o tamanho da janela}
\NormalTok{    window }\OtherTok{\textless{}\textless{}{-}} \FunctionTok{c}\NormalTok{(window, value) }\CommentTok{\# Adiciona o novo valor à janela}
\NormalTok{    total }\OtherTok{\textless{}\textless{}{-}}\NormalTok{ total }\SpecialCharTok{+}\NormalTok{ value }\CommentTok{\# Atualiza a soma total}
    \ControlFlowTok{if}\NormalTok{ (width }\SpecialCharTok{\textgreater{}} \DecValTok{1}\NormalTok{) \{}
\NormalTok{      variance }\OtherTok{\textless{}\textless{}{-}} \FunctionTok{var}\NormalTok{(window) }\CommentTok{\# Calcula a variância se a janela tiver mais de um valor}
\NormalTok{    \}}
    \CommentTok{\# Checa por concept drift}
    \ControlFlowTok{if}\NormalTok{ (width }\SpecialCharTok{\textgreater{}} \DecValTok{1} \SpecialCharTok{\&\&} \FunctionTok{detect\_change}\NormalTok{()) \{}
      \FunctionTok{return}\NormalTok{(}\ConstantTok{TRUE}\NormalTok{) }\CommentTok{\# Retorna TRUE se uma mudança for detectada}
\NormalTok{    \}}
    \FunctionTok{return}\NormalTok{(}\ConstantTok{FALSE}\NormalTok{) }\CommentTok{\# Retorna FALSE se nenhuma mudança for detectada}
\NormalTok{  \}}
  
  \CommentTok{\# Função para detectar mudança}
\NormalTok{  detect\_change }\OtherTok{\textless{}{-}} \ControlFlowTok{function}\NormalTok{() \{}
\NormalTok{    mean\_val }\OtherTok{\textless{}{-}} \FunctionTok{mean}\NormalTok{(window) }\CommentTok{\# Calcula a média dos valores na janela}
    \ControlFlowTok{for}\NormalTok{ (n }\ControlFlowTok{in} \DecValTok{1}\SpecialCharTok{:}\NormalTok{(width }\SpecialCharTok{{-}} \DecValTok{1}\NormalTok{)) \{}
      \CommentTok{\# Divide a janela em duas sub{-}janelas e calcula as médias de cada sub{-}janela}
\NormalTok{      mean0 }\OtherTok{\textless{}{-}} \FunctionTok{mean}\NormalTok{(window[}\DecValTok{1}\SpecialCharTok{:}\NormalTok{n])}
\NormalTok{      mean1 }\OtherTok{\textless{}{-}} \FunctionTok{mean}\NormalTok{(window[(n }\SpecialCharTok{+} \DecValTok{1}\NormalTok{)}\SpecialCharTok{:}\NormalTok{width])}
      \CommentTok{\# Compara as médias das sub{-}janelas usando um teste estatístico}
      \ControlFlowTok{if}\NormalTok{ (}\FunctionTok{abs}\NormalTok{(mean0 }\SpecialCharTok{{-}}\NormalTok{ mean1) }\SpecialCharTok{\textgreater{}} \FunctionTok{sqrt}\NormalTok{((variance }\SpecialCharTok{/}\NormalTok{ n) }\SpecialCharTok{+}\NormalTok{ (variance }\SpecialCharTok{/}\NormalTok{ (width }\SpecialCharTok{{-}}\NormalTok{ n))) }\SpecialCharTok{*} \FunctionTok{qnorm}\NormalTok{(}\DecValTok{1} \SpecialCharTok{{-}}\NormalTok{ delta)) \{}
        \CommentTok{\# Se uma mudança for detectada, ajusta a janela para descartar os dados antigos}
\NormalTok{        window }\OtherTok{\textless{}\textless{}{-}}\NormalTok{ window[(n }\SpecialCharTok{+} \DecValTok{1}\NormalTok{)}\SpecialCharTok{:}\NormalTok{width]}
\NormalTok{        width }\OtherTok{\textless{}\textless{}{-}} \FunctionTok{length}\NormalTok{(window)}
\NormalTok{        total }\OtherTok{\textless{}\textless{}{-}} \FunctionTok{sum}\NormalTok{(window)}
\NormalTok{        variance }\OtherTok{\textless{}\textless{}{-}} \FunctionTok{var}\NormalTok{(window)}
        \FunctionTok{return}\NormalTok{(}\ConstantTok{TRUE}\NormalTok{) }\CommentTok{\# Retorna TRUE indicando que uma mudança foi detectada}
\NormalTok{      \}}
\NormalTok{    \}}
    \FunctionTok{return}\NormalTok{(}\ConstantTok{FALSE}\NormalTok{) }\CommentTok{\# Retorna FALSE se nenhuma mudança for detectada}
\NormalTok{  \}}
  
  \CommentTok{\# Retorna as funções de atualização e detecção}
  \FunctionTok{list}\NormalTok{(}\AttributeTok{update =}\NormalTok{ update)}
\NormalTok{\}}

\CommentTok{\# Exemplo de uso com visualização}
\FunctionTok{set.seed}\NormalTok{(}\DecValTok{123}\NormalTok{) }\CommentTok{\# Define a semente para reprodutibilidade}
\NormalTok{data\_stream }\OtherTok{\textless{}{-}} \FunctionTok{c}\NormalTok{(}\FunctionTok{rnorm}\NormalTok{(}\DecValTok{100}\NormalTok{, }\AttributeTok{mean =} \DecValTok{20}\NormalTok{), }\FunctionTok{rnorm}\NormalTok{(}\DecValTok{100}\NormalTok{, }\AttributeTok{mean =} \DecValTok{25}\NormalTok{)) }\CommentTok{\# Gera dados de exemplo com duas médias diferentes}
\NormalTok{adwin }\OtherTok{\textless{}{-}} \FunctionTok{ADWIN}\NormalTok{(}\AttributeTok{delta =} \FloatTok{0.002}\NormalTok{) }\CommentTok{\# Inicializa o detector ADWIN com delta = 0.002}

\CommentTok{\# Vetor para armazenar os pontos de detecção de mudança}
\NormalTok{change\_points }\OtherTok{\textless{}{-}} \FunctionTok{numeric}\NormalTok{(}\DecValTok{0}\NormalTok{)}

\CommentTok{\# Processa o fluxo de dados}
\ControlFlowTok{for}\NormalTok{ (i }\ControlFlowTok{in} \DecValTok{1}\SpecialCharTok{:}\FunctionTok{length}\NormalTok{(data\_stream)) \{}
  \ControlFlowTok{if}\NormalTok{ (adwin}\SpecialCharTok{$}\FunctionTok{update}\NormalTok{(data\_stream[i])) \{}
\NormalTok{    change\_points }\OtherTok{\textless{}{-}} \FunctionTok{c}\NormalTok{(change\_points, i) }\CommentTok{\# Armazena o índice onde a mudança foi detectada}
\NormalTok{  \}}
\NormalTok{\}}

\CommentTok{\# Plotar os dados e os pontos de mudança}
\FunctionTok{library}\NormalTok{(ggplot2)}

\CommentTok{\# Criar um data frame com os dados e os pontos de mudança}
\NormalTok{df1 }\OtherTok{\textless{}{-}} \FunctionTok{data.frame}\NormalTok{(}
  \AttributeTok{index =} \DecValTok{1}\SpecialCharTok{:}\FunctionTok{length}\NormalTok{(data\_stream),}
  \AttributeTok{value =}\NormalTok{ data\_stream,}
  \AttributeTok{change =} \FunctionTok{ifelse}\NormalTok{(}\DecValTok{1}\SpecialCharTok{:}\FunctionTok{length}\NormalTok{(data\_stream) }\SpecialCharTok{\%in\%}\NormalTok{ change\_points, }\StringTok{"Change Detected"}\NormalTok{, }\StringTok{"No Change"}\NormalTok{)}
\NormalTok{)}

\CommentTok{\# Plotar os dados}
\FunctionTok{ggplot}\NormalTok{(df1, }\FunctionTok{aes}\NormalTok{(}\AttributeTok{x =}\NormalTok{ index, }\AttributeTok{y =}\NormalTok{ value)) }\SpecialCharTok{+}
  \FunctionTok{geom\_line}\NormalTok{() }\SpecialCharTok{+}
  \FunctionTok{geom\_point}\NormalTok{(}\AttributeTok{data =} \FunctionTok{subset}\NormalTok{(df1, change }\SpecialCharTok{==} \StringTok{"Change Detected"}\NormalTok{), }\FunctionTok{aes}\NormalTok{(}\AttributeTok{x =}\NormalTok{ index, }\AttributeTok{y =}\NormalTok{ value), }\AttributeTok{color =} \StringTok{"red"}\NormalTok{, }\AttributeTok{size =} \DecValTok{2}\NormalTok{) }\SpecialCharTok{+}
  \FunctionTok{labs}\NormalTok{(}\AttributeTok{title =} \StringTok{"Change Detected whit ADWIN"}\NormalTok{, }\AttributeTok{x =} \StringTok{"Índice"}\NormalTok{, }\AttributeTok{y =} \StringTok{"Valor"}\NormalTok{) }\SpecialCharTok{+}
  \FunctionTok{theme\_minimal}\NormalTok{()}
\end{Highlighting}
\end{Shaded}

\includegraphics{TrickyTimeSeries_files/figure-latex/unnamed-chunk-6-1.pdf}

\begin{Shaded}
\begin{Highlighting}[]
\CommentTok{\# Análise detalhada dos pontos de mudança}
\FunctionTok{print}\NormalTok{(}\StringTok{"Análise dos Pontos de Mudança:"}\NormalTok{)}
\end{Highlighting}
\end{Shaded}

\begin{verbatim}
## [1] "Análise dos Pontos de Mudança:"
\end{verbatim}

\begin{Shaded}
\begin{Highlighting}[]
\ControlFlowTok{for}\NormalTok{ (point }\ControlFlowTok{in}\NormalTok{ change\_points) \{}
  \FunctionTok{cat}\NormalTok{(}\StringTok{"Ponto de mudança detectado em:"}\NormalTok{, point, }\StringTok{"}\SpecialCharTok{\textbackslash{}n}\StringTok{"}\NormalTok{)}
  \ControlFlowTok{if}\NormalTok{ (point }\SpecialCharTok{\textgreater{}} \DecValTok{1} \SpecialCharTok{\&}\NormalTok{ point }\SpecialCharTok{\textless{}} \FunctionTok{length}\NormalTok{(data\_stream)) \{}
\NormalTok{    before\_change }\OtherTok{\textless{}{-}}\NormalTok{ data\_stream[(point}\DecValTok{{-}2}\NormalTok{)}\SpecialCharTok{:}\NormalTok{(point}\DecValTok{{-}1}\NormalTok{)]}
\NormalTok{    after\_change }\OtherTok{\textless{}{-}}\NormalTok{ data\_stream[(point}\SpecialCharTok{+}\DecValTok{1}\NormalTok{)}\SpecialCharTok{:}\NormalTok{(point}\SpecialCharTok{+}\DecValTok{2}\NormalTok{)]}
    \FunctionTok{cat}\NormalTok{(}\StringTok{"Valores antes da mudança:"}\NormalTok{, before\_change, }\StringTok{"}\SpecialCharTok{\textbackslash{}n}\StringTok{"}\NormalTok{)}
    \FunctionTok{cat}\NormalTok{(}\StringTok{"Valores após a mudança:"}\NormalTok{, after\_change, }\StringTok{"}\SpecialCharTok{\textbackslash{}n\textbackslash{}n}\StringTok{"}\NormalTok{)}
\NormalTok{  \}}
\NormalTok{\}}
\end{Highlighting}
\end{Shaded}

\begin{verbatim}
## Ponto de mudança detectado em: 101 
## Valores antes da mudança: 19.7643 18.97358 
## Valores após a mudança: 25.25688 24.75331 
## 
## Ponto de mudança detectado em: 164 
## Valores antes da mudança: 23.95082 23.73984 
## Valores após a mudança: 24.58314 25.29823 
## 
## Ponto de mudança detectado em: 180 
## Valores antes da mudança: 25.31048 25.43652 
## Valores após a mudança: 23.93667 26.26319
\end{verbatim}

\hypertarget{plano-hxc-entropia-de-shannon-x-complexidade-de-jensen-shannon-sem-op}{%
\subsection{Plano HxC (Entropia de Shannon x Complexidade de
Jensen-Shannon) Sem
OP}\label{plano-hxc-entropia-de-shannon-x-complexidade-de-jensen-shannon-sem-op}}

Para visualizar a evolução ao longo do tempo e obter vários pontos no
plano HxC, é preciso calcular essas métricas em janelas deslizantes ao
longo da série temporal. O plano HxC plota a entropia de Shannon (H) no
eixo x e a complexidade de Jensen-Shannon (C) no eixo y para janelas
deslizantes da série temporal.

\begin{Shaded}
\begin{Highlighting}[]
\CommentTok{\# Carregar bibliotecas necessárias}
\FunctionTok{library}\NormalTok{(ggplot2)}
\FunctionTok{library}\NormalTok{(ggthemes)}

\CommentTok{\# Função para calcular a entropia de Shannon}
\NormalTok{shannon\_entropy }\OtherTok{\textless{}{-}} \ControlFlowTok{function}\NormalTok{(probabilities) \{}
  \SpecialCharTok{{-}}\FunctionTok{sum}\NormalTok{(probabilities }\SpecialCharTok{*} \FunctionTok{log2}\NormalTok{(probabilities), }\AttributeTok{na.rm =} \ConstantTok{TRUE}\NormalTok{)}
\NormalTok{\}}

\CommentTok{\# Função para calcular a complexidade de Jensen{-}Shannon}
\NormalTok{js\_complexity }\OtherTok{\textless{}{-}} \ControlFlowTok{function}\NormalTok{(probabilities) \{}
\NormalTok{  q }\OtherTok{\textless{}{-}} \FunctionTok{rep}\NormalTok{(}\DecValTok{1}\SpecialCharTok{/}\FunctionTok{length}\NormalTok{(probabilities), }\FunctionTok{length}\NormalTok{(probabilities))}
\NormalTok{  m }\OtherTok{\textless{}{-}}\NormalTok{ (probabilities }\SpecialCharTok{+}\NormalTok{ q) }\SpecialCharTok{/} \DecValTok{2}
\NormalTok{  (}\FunctionTok{shannon\_entropy}\NormalTok{(m) }\SpecialCharTok{{-}} \FloatTok{0.5} \SpecialCharTok{*}\NormalTok{ (}\FunctionTok{shannon\_entropy}\NormalTok{(probabilities) }\SpecialCharTok{+} \FunctionTok{shannon\_entropy}\NormalTok{(q))) }\SpecialCharTok{/} \FunctionTok{log2}\NormalTok{(}\FunctionTok{length}\NormalTok{(probabilities))}
\NormalTok{\}}

\CommentTok{\# Função para calcular H e C em janelas deslizantes}
\NormalTok{calculate\_hxc }\OtherTok{\textless{}{-}} \ControlFlowTok{function}\NormalTok{(series, window\_size) \{}
\NormalTok{  n }\OtherTok{\textless{}{-}} \FunctionTok{length}\NormalTok{(series)}
\NormalTok{  h\_values }\OtherTok{\textless{}{-}} \FunctionTok{c}\NormalTok{()}
\NormalTok{  c\_values }\OtherTok{\textless{}{-}} \FunctionTok{c}\NormalTok{()}
  
  \ControlFlowTok{for}\NormalTok{ (i }\ControlFlowTok{in} \DecValTok{1}\SpecialCharTok{:}\NormalTok{(n }\SpecialCharTok{{-}}\NormalTok{ window\_size }\SpecialCharTok{+} \DecValTok{1}\NormalTok{)) \{}
\NormalTok{    window }\OtherTok{\textless{}{-}}\NormalTok{ series[i}\SpecialCharTok{:}\NormalTok{(i }\SpecialCharTok{+}\NormalTok{ window\_size }\SpecialCharTok{{-}} \DecValTok{1}\NormalTok{)]}
\NormalTok{    probabilities }\OtherTok{\textless{}{-}} \FunctionTok{hist}\NormalTok{(window, }\AttributeTok{plot =} \ConstantTok{FALSE}\NormalTok{)}\SpecialCharTok{$}\NormalTok{density}
\NormalTok{    h }\OtherTok{\textless{}{-}} \FunctionTok{shannon\_entropy}\NormalTok{(probabilities)}
\NormalTok{    c }\OtherTok{\textless{}{-}} \FunctionTok{js\_complexity}\NormalTok{(probabilities)}
\NormalTok{    h\_values }\OtherTok{\textless{}{-}} \FunctionTok{c}\NormalTok{(h\_values, h)}
\NormalTok{    c\_values }\OtherTok{\textless{}{-}} \FunctionTok{c}\NormalTok{(c\_values, c)}
\NormalTok{  \}}
  
  \FunctionTok{data.frame}\NormalTok{(}\AttributeTok{H =}\NormalTok{ h\_values, }\AttributeTok{C =}\NormalTok{ c\_values)}
\NormalTok{\}}

\CommentTok{\# Função ADWIN para detectar mudanças de conceito em fluxos de dados}
\NormalTok{ADWIN }\OtherTok{\textless{}{-}} \ControlFlowTok{function}\NormalTok{(}\AttributeTok{delta =} \FloatTok{0.002}\NormalTok{) \{}
  \CommentTok{\# Inicializa as variáveis}
\NormalTok{  width }\OtherTok{\textless{}{-}} \DecValTok{0} \CommentTok{\# Tamanho da janela}
\NormalTok{  total }\OtherTok{\textless{}{-}} \DecValTok{0} \CommentTok{\# Soma dos valores na janela}
\NormalTok{  variance }\OtherTok{\textless{}{-}} \DecValTok{0} \CommentTok{\# Variância dos valores na janela}
\NormalTok{  window }\OtherTok{\textless{}{-}} \FunctionTok{numeric}\NormalTok{(}\DecValTok{0}\NormalTok{) }\CommentTok{\# Vetor que armazena os valores na janela}
  
  \CommentTok{\# Função para atualizar o ADWIN com um novo valor}
\NormalTok{  update }\OtherTok{\textless{}{-}} \ControlFlowTok{function}\NormalTok{(value) \{}
\NormalTok{    width }\OtherTok{\textless{}\textless{}{-}}\NormalTok{ width }\SpecialCharTok{+} \DecValTok{1} \CommentTok{\# Incrementa o tamanho da janela}
\NormalTok{    window }\OtherTok{\textless{}\textless{}{-}} \FunctionTok{c}\NormalTok{(window, value) }\CommentTok{\# Adiciona o novo valor à janela}
\NormalTok{    total }\OtherTok{\textless{}\textless{}{-}}\NormalTok{ total }\SpecialCharTok{+}\NormalTok{ value }\CommentTok{\# Atualiza a soma total}
    \ControlFlowTok{if}\NormalTok{ (width }\SpecialCharTok{\textgreater{}} \DecValTok{1}\NormalTok{) \{}
\NormalTok{      variance }\OtherTok{\textless{}\textless{}{-}} \FunctionTok{var}\NormalTok{(window) }\CommentTok{\# Calcula a variância se a janela tiver mais de um valor}
\NormalTok{    \}}
    \CommentTok{\# Checa por concept drift}
    \ControlFlowTok{if}\NormalTok{ (width }\SpecialCharTok{\textgreater{}} \DecValTok{1} \SpecialCharTok{\&\&} \FunctionTok{detect\_change}\NormalTok{()) \{}
      \FunctionTok{return}\NormalTok{(}\ConstantTok{TRUE}\NormalTok{) }\CommentTok{\# Retorna TRUE se uma mudança for detectada}
\NormalTok{    \}}
    \FunctionTok{return}\NormalTok{(}\ConstantTok{FALSE}\NormalTok{) }\CommentTok{\# Retorna FALSE se nenhuma mudança for detectada}
\NormalTok{  \}}
  
  \CommentTok{\# Função para detectar mudança}
\NormalTok{  detect\_change }\OtherTok{\textless{}{-}} \ControlFlowTok{function}\NormalTok{() \{}
\NormalTok{    mean\_val }\OtherTok{\textless{}{-}} \FunctionTok{mean}\NormalTok{(window) }\CommentTok{\# Calcula a média dos valores na janela}
    \ControlFlowTok{for}\NormalTok{ (n }\ControlFlowTok{in} \DecValTok{1}\SpecialCharTok{:}\NormalTok{(width }\SpecialCharTok{{-}} \DecValTok{1}\NormalTok{)) \{}
      \CommentTok{\# Divide a janela em duas sub{-}janelas e calcula as médias de cada sub{-}janela}
\NormalTok{      mean0 }\OtherTok{\textless{}{-}} \FunctionTok{mean}\NormalTok{(window[}\DecValTok{1}\SpecialCharTok{:}\NormalTok{n])}
\NormalTok{      mean1 }\OtherTok{\textless{}{-}} \FunctionTok{mean}\NormalTok{(window[(n }\SpecialCharTok{+} \DecValTok{1}\NormalTok{)}\SpecialCharTok{:}\NormalTok{width])}
      \CommentTok{\# Compara as médias das sub{-}janelas usando um teste estatístico}
      \ControlFlowTok{if}\NormalTok{ (}\FunctionTok{abs}\NormalTok{(mean0 }\SpecialCharTok{{-}}\NormalTok{ mean1) }\SpecialCharTok{\textgreater{}} \FunctionTok{sqrt}\NormalTok{((variance }\SpecialCharTok{/}\NormalTok{ n) }\SpecialCharTok{+}\NormalTok{ (variance }\SpecialCharTok{/}\NormalTok{ (width }\SpecialCharTok{{-}}\NormalTok{ n))) }\SpecialCharTok{*} \FunctionTok{qnorm}\NormalTok{(}\DecValTok{1} \SpecialCharTok{{-}}\NormalTok{ delta)) \{}
        \CommentTok{\# Se uma mudança for detectada, ajusta a janela para descartar os dados antigos}
\NormalTok{        window }\OtherTok{\textless{}\textless{}{-}}\NormalTok{ window[(n }\SpecialCharTok{+} \DecValTok{1}\NormalTok{)}\SpecialCharTok{:}\NormalTok{width]}
\NormalTok{        width }\OtherTok{\textless{}\textless{}{-}} \FunctionTok{length}\NormalTok{(window)}
\NormalTok{        total }\OtherTok{\textless{}\textless{}{-}} \FunctionTok{sum}\NormalTok{(window)}
\NormalTok{        variance }\OtherTok{\textless{}\textless{}{-}} \FunctionTok{var}\NormalTok{(window)}
        \FunctionTok{return}\NormalTok{(}\ConstantTok{TRUE}\NormalTok{) }\CommentTok{\# Retorna TRUE indicando que uma mudança foi detectada}
\NormalTok{      \}}
\NormalTok{    \}}
    \FunctionTok{return}\NormalTok{(}\ConstantTok{FALSE}\NormalTok{) }\CommentTok{\# Retorna FALSE se nenhuma mudança for detectada}
\NormalTok{  \}}
  
  \CommentTok{\# Retorna as funções de atualização e detecção}
  \FunctionTok{list}\NormalTok{(}\AttributeTok{update =}\NormalTok{ update)}
\NormalTok{\}}

\CommentTok{\# Exemplo de uso com visualização}
\FunctionTok{set.seed}\NormalTok{(}\DecValTok{123}\NormalTok{) }\CommentTok{\# Define a semente para reprodutibilidade}
\NormalTok{data\_stream }\OtherTok{\textless{}{-}} \FunctionTok{c}\NormalTok{(}\FunctionTok{rnorm}\NormalTok{(}\DecValTok{100}\NormalTok{, }\AttributeTok{mean =} \DecValTok{20}\NormalTok{), }\FunctionTok{rnorm}\NormalTok{(}\DecValTok{100}\NormalTok{, }\AttributeTok{mean =} \DecValTok{25}\NormalTok{)) }\CommentTok{\# Gera dados de exemplo com duas médias diferentes}
\NormalTok{adwin }\OtherTok{\textless{}{-}} \FunctionTok{ADWIN}\NormalTok{(}\AttributeTok{delta =} \FloatTok{0.002}\NormalTok{) }\CommentTok{\# Inicializa o detector ADWIN com delta = 0.002}

\CommentTok{\# Vetor para armazenar os pontos de detecção de mudança}
\NormalTok{change\_points }\OtherTok{\textless{}{-}} \FunctionTok{numeric}\NormalTok{(}\DecValTok{0}\NormalTok{)}

\CommentTok{\# Processa o fluxo de dados}
\ControlFlowTok{for}\NormalTok{ (i }\ControlFlowTok{in} \DecValTok{1}\SpecialCharTok{:}\FunctionTok{length}\NormalTok{(data\_stream)) \{}
  \ControlFlowTok{if}\NormalTok{ (adwin}\SpecialCharTok{$}\FunctionTok{update}\NormalTok{(data\_stream[i])) \{}
\NormalTok{    change\_points }\OtherTok{\textless{}{-}} \FunctionTok{c}\NormalTok{(change\_points, i) }\CommentTok{\# Armazena o índice onde a mudança foi detectada}
\NormalTok{  \}}
\NormalTok{\}}

\CommentTok{\# Calcular H e C em janelas deslizantes}
\NormalTok{window\_size }\OtherTok{\textless{}{-}} \DecValTok{50}
\NormalTok{hxc\_data }\OtherTok{\textless{}{-}} \FunctionTok{calculate\_hxc}\NormalTok{(data\_stream, window\_size)}

\CommentTok{\# Plotar o plano HxC}
\FunctionTok{ggplot}\NormalTok{(hxc\_data, }\FunctionTok{aes}\NormalTok{(}\AttributeTok{x =}\NormalTok{ H, }\AttributeTok{y =}\NormalTok{ C)) }\SpecialCharTok{+}
  \FunctionTok{geom\_point}\NormalTok{(}\AttributeTok{color =} \StringTok{"blue"}\NormalTok{) }\SpecialCharTok{+}
  \FunctionTok{labs}\NormalTok{(}\AttributeTok{title =} \StringTok{"Plano HxC with ADWIN"}\NormalTok{,}
       \AttributeTok{x =} \StringTok{"Entropia de Shannon (H)"}\NormalTok{, }\AttributeTok{y =} \StringTok{"Complexidade de Jensen{-}Shannon (C)"}\NormalTok{) }\SpecialCharTok{+}
  \FunctionTok{theme\_minimal}\NormalTok{() }\SpecialCharTok{+}
  \FunctionTok{geom\_smooth}\NormalTok{(}\AttributeTok{method =} \StringTok{"lm"}\NormalTok{, }\AttributeTok{color =} \StringTok{"red"}\NormalTok{) }\SpecialCharTok{+} \CommentTok{\# Adiciona uma linha de tendência}
  \FunctionTok{theme\_bw}\NormalTok{() }\CommentTok{\# Utiliza um tema com fundo branco para maior clareza}
\end{Highlighting}
\end{Shaded}

\begin{verbatim}
## `geom_smooth()` using formula = 'y ~ x'
\end{verbatim}

\includegraphics{TrickyTimeSeries_files/figure-latex/unnamed-chunk-7-1.pdf}

\begin{Shaded}
\begin{Highlighting}[]
\CommentTok{\# Criar um data frame com os dados e os pontos de mudança}
\NormalTok{df }\OtherTok{\textless{}{-}} \FunctionTok{data.frame}\NormalTok{(}
  \AttributeTok{index =} \DecValTok{1}\SpecialCharTok{:}\FunctionTok{length}\NormalTok{(data\_stream),}
  \AttributeTok{value =}\NormalTok{ data\_stream,}
  \AttributeTok{change =} \FunctionTok{ifelse}\NormalTok{(}\DecValTok{1}\SpecialCharTok{:}\FunctionTok{length}\NormalTok{(data\_stream) }\SpecialCharTok{\%in\%}\NormalTok{ change\_points, }\StringTok{"Change Detected"}\NormalTok{, }\StringTok{"No Change"}\NormalTok{)}
\NormalTok{)}

\CommentTok{\# Plotar os dados e os pontos de mudança}
\CommentTok{\# O gráfico mostra os pontos calculados para cada janela.}
\FunctionTok{ggplot}\NormalTok{(df, }\FunctionTok{aes}\NormalTok{(}\AttributeTok{x =}\NormalTok{ index, }\AttributeTok{y =}\NormalTok{ value)) }\SpecialCharTok{+}
  \FunctionTok{geom\_line}\NormalTok{() }\SpecialCharTok{+}
  \FunctionTok{geom\_point}\NormalTok{(}\AttributeTok{data =} \FunctionTok{subset}\NormalTok{(df, change }\SpecialCharTok{==} \StringTok{"Change Detected"}\NormalTok{), }\FunctionTok{aes}\NormalTok{(}\AttributeTok{x =}\NormalTok{ index, }\AttributeTok{y =}\NormalTok{ value), }\AttributeTok{color =} \StringTok{"red"}\NormalTok{, }\AttributeTok{size =} \DecValTok{2}\NormalTok{) }\SpecialCharTok{+}
  \FunctionTok{labs}\NormalTok{(}\AttributeTok{title =} \StringTok{"Change Detected with ADWIN"}\NormalTok{, }\AttributeTok{x =} \StringTok{"Índice"}\NormalTok{, }\AttributeTok{y =} \StringTok{"Valor"}\NormalTok{) }\SpecialCharTok{+}
  \FunctionTok{theme\_minimal}\NormalTok{()}
\end{Highlighting}
\end{Shaded}

\includegraphics{TrickyTimeSeries_files/figure-latex/unnamed-chunk-7-2.pdf}

\begin{Shaded}
\begin{Highlighting}[]
\CommentTok{\# Análise detalhada dos pontos de mudança}
\FunctionTok{print}\NormalTok{(}\StringTok{"Análise dos Pontos de Mudança:"}\NormalTok{)}
\end{Highlighting}
\end{Shaded}

\begin{verbatim}
## [1] "Análise dos Pontos de Mudança:"
\end{verbatim}

\begin{Shaded}
\begin{Highlighting}[]
\ControlFlowTok{for}\NormalTok{ (point }\ControlFlowTok{in}\NormalTok{ change\_points) \{}
  \FunctionTok{cat}\NormalTok{(}\StringTok{"Ponto de mudança detectado em:"}\NormalTok{, point, }\StringTok{"}\SpecialCharTok{\textbackslash{}n}\StringTok{"}\NormalTok{)}
  \ControlFlowTok{if}\NormalTok{ (point }\SpecialCharTok{\textgreater{}} \DecValTok{1} \SpecialCharTok{\&}\NormalTok{ point }\SpecialCharTok{\textless{}} \FunctionTok{length}\NormalTok{(data\_stream)) \{}
\NormalTok{    before\_change }\OtherTok{\textless{}{-}}\NormalTok{ data\_stream[(point}\DecValTok{{-}2}\NormalTok{)}\SpecialCharTok{:}\NormalTok{(point}\DecValTok{{-}1}\NormalTok{)]}
\NormalTok{    after\_change }\OtherTok{\textless{}{-}}\NormalTok{ data\_stream[(point}\SpecialCharTok{+}\DecValTok{1}\NormalTok{)}\SpecialCharTok{:}\NormalTok{(point}\SpecialCharTok{+}\DecValTok{2}\NormalTok{)]}
    \FunctionTok{cat}\NormalTok{(}\StringTok{"Valores antes da mudança:"}\NormalTok{, before\_change, }\StringTok{"}\SpecialCharTok{\textbackslash{}n}\StringTok{"}\NormalTok{)}
    \FunctionTok{cat}\NormalTok{(}\StringTok{"Valores após a mudança:"}\NormalTok{, after\_change, }\StringTok{"}\SpecialCharTok{\textbackslash{}n\textbackslash{}n}\StringTok{"}\NormalTok{)}
\NormalTok{  \}}
\NormalTok{\}}
\end{Highlighting}
\end{Shaded}

\begin{verbatim}
## Ponto de mudança detectado em: 101 
## Valores antes da mudança: 19.7643 18.97358 
## Valores após a mudança: 25.25688 24.75331 
## 
## Ponto de mudança detectado em: 164 
## Valores antes da mudança: 23.95082 23.73984 
## Valores após a mudança: 24.58314 25.29823 
## 
## Ponto de mudança detectado em: 180 
## Valores antes da mudança: 25.31048 25.43652 
## Valores após a mudança: 23.93667 26.26319
\end{verbatim}

\hypertarget{anuxe1lise-detalhada-dos-pontos-de-mudanuxe7a-with-op}{%
\subsubsection{Análise Detalhada dos Pontos de Mudança with
OP}\label{anuxe1lise-detalhada-dos-pontos-de-mudanuxe7a-with-op}}

\begin{Shaded}
\begin{Highlighting}[]
\CommentTok{\# Instale a biblioteca \textquotesingle{}pracma\textquotesingle{} se ainda não estiver instalada}
\CommentTok{\# install.packages("pracma")}

\CommentTok{\# Carregar bibliotecas necessárias}
\FunctionTok{library}\NormalTok{(pracma)}
\FunctionTok{library}\NormalTok{(ggplot2)}
\FunctionTok{library}\NormalTok{(ggthemes)}
\FunctionTok{library}\NormalTok{(statcomp)}

\CommentTok{\# Função para calcular padrões ordinais usando statcomp}
\NormalTok{ordinal\_patterns\_statcomp }\OtherTok{\textless{}{-}} \ControlFlowTok{function}\NormalTok{(series, emb\_dim) \{}
  \ControlFlowTok{if}\NormalTok{ (}\FunctionTok{length}\NormalTok{(series) }\SpecialCharTok{\textless{}}\NormalTok{ emb\_dim) \{}
    \FunctionTok{stop}\NormalTok{(}\StringTok{"A série temporal é muito curta para a dimensão de embedding especificada."}\NormalTok{)}
\NormalTok{  \}}
  
  \CommentTok{\# Utilizando a função from \textasciigrave{}statcomp\textasciigrave{} para calcular os padrões ordinais}
\NormalTok{  patterns }\OtherTok{\textless{}{-}} \FunctionTok{ordinal\_pattern}\NormalTok{(series, emb\_dim)}
  
  \FunctionTok{return}\NormalTok{(patterns)}
\NormalTok{\}}

\CommentTok{\# Função ADWIN ajustada para padrões ordinais}
\NormalTok{ADWIN }\OtherTok{\textless{}{-}} \ControlFlowTok{function}\NormalTok{(}\AttributeTok{delta =} \FloatTok{0.002}\NormalTok{) \{}
\NormalTok{  width }\OtherTok{\textless{}{-}} \DecValTok{0}
\NormalTok{  total }\OtherTok{\textless{}{-}} \DecValTok{0}
\NormalTok{  variance }\OtherTok{\textless{}{-}} \DecValTok{0}
\NormalTok{  window }\OtherTok{\textless{}{-}} \FunctionTok{numeric}\NormalTok{(}\DecValTok{0}\NormalTok{)}
  
\NormalTok{  update }\OtherTok{\textless{}{-}} \ControlFlowTok{function}\NormalTok{(value) \{}
\NormalTok{    width }\OtherTok{\textless{}\textless{}{-}}\NormalTok{ width }\SpecialCharTok{+} \DecValTok{1}
\NormalTok{    window }\OtherTok{\textless{}\textless{}{-}} \FunctionTok{c}\NormalTok{(window, value)}
\NormalTok{    total }\OtherTok{\textless{}\textless{}{-}}\NormalTok{ total }\SpecialCharTok{+}\NormalTok{ value}
    \ControlFlowTok{if}\NormalTok{ (width }\SpecialCharTok{\textgreater{}} \DecValTok{1}\NormalTok{) \{}
\NormalTok{      variance }\OtherTok{\textless{}\textless{}{-}} \FunctionTok{var}\NormalTok{(window, }\AttributeTok{na.rm =} \ConstantTok{TRUE}\NormalTok{) }\CommentTok{\# Calcula a variância, removendo NAs}
\NormalTok{    \}}
    \ControlFlowTok{if}\NormalTok{ (width }\SpecialCharTok{\textgreater{}} \DecValTok{1} \SpecialCharTok{\&\&} \FunctionTok{detect\_change}\NormalTok{()) \{}
      \FunctionTok{return}\NormalTok{(}\ConstantTok{TRUE}\NormalTok{)}
\NormalTok{    \}}
    \FunctionTok{return}\NormalTok{(}\ConstantTok{FALSE}\NormalTok{)}
\NormalTok{  \}}
  
\NormalTok{  detect\_change }\OtherTok{\textless{}{-}} \ControlFlowTok{function}\NormalTok{() \{}
\NormalTok{    mean\_val }\OtherTok{\textless{}{-}} \FunctionTok{mean}\NormalTok{(window, }\AttributeTok{na.rm =} \ConstantTok{TRUE}\NormalTok{) }\CommentTok{\# Calcula a média, removendo NAs}
    \ControlFlowTok{for}\NormalTok{ (n }\ControlFlowTok{in} \DecValTok{1}\SpecialCharTok{:}\NormalTok{(width }\SpecialCharTok{{-}} \DecValTok{1}\NormalTok{)) \{}
\NormalTok{      mean0 }\OtherTok{\textless{}{-}} \FunctionTok{mean}\NormalTok{(window[}\DecValTok{1}\SpecialCharTok{:}\NormalTok{n], }\AttributeTok{na.rm =} \ConstantTok{TRUE}\NormalTok{)}
\NormalTok{      mean1 }\OtherTok{\textless{}{-}} \FunctionTok{mean}\NormalTok{(window[(n }\SpecialCharTok{+} \DecValTok{1}\NormalTok{)}\SpecialCharTok{:}\NormalTok{width], }\AttributeTok{na.rm =} \ConstantTok{TRUE}\NormalTok{)}
      \ControlFlowTok{if}\NormalTok{ (}\FunctionTok{is.na}\NormalTok{(mean0) }\SpecialCharTok{||} \FunctionTok{is.na}\NormalTok{(mean1)) }\ControlFlowTok{next} \CommentTok{\# Pula iteração se a média for NA}
      \ControlFlowTok{if}\NormalTok{ (}\FunctionTok{abs}\NormalTok{(mean0 }\SpecialCharTok{{-}}\NormalTok{ mean1) }\SpecialCharTok{\textgreater{}} \FunctionTok{sqrt}\NormalTok{((variance }\SpecialCharTok{/}\NormalTok{ n) }\SpecialCharTok{+}\NormalTok{ (variance }\SpecialCharTok{/}\NormalTok{ (width }\SpecialCharTok{{-}}\NormalTok{ n))) }\SpecialCharTok{*} \FunctionTok{qnorm}\NormalTok{(}\DecValTok{1} \SpecialCharTok{{-}}\NormalTok{ delta)) \{}
\NormalTok{        window }\OtherTok{\textless{}\textless{}{-}}\NormalTok{ window[(n }\SpecialCharTok{+} \DecValTok{1}\NormalTok{)}\SpecialCharTok{:}\NormalTok{width]}
\NormalTok{        width }\OtherTok{\textless{}\textless{}{-}} \FunctionTok{length}\NormalTok{(window)}
\NormalTok{        total }\OtherTok{\textless{}\textless{}{-}} \FunctionTok{sum}\NormalTok{(window, }\AttributeTok{na.rm =} \ConstantTok{TRUE}\NormalTok{)}
\NormalTok{        variance }\OtherTok{\textless{}\textless{}{-}} \FunctionTok{var}\NormalTok{(window, }\AttributeTok{na.rm =} \ConstantTok{TRUE}\NormalTok{)}
        \FunctionTok{return}\NormalTok{(}\ConstantTok{TRUE}\NormalTok{)}
\NormalTok{      \}}
\NormalTok{    \}}
    \FunctionTok{return}\NormalTok{(}\ConstantTok{FALSE}\NormalTok{)}
\NormalTok{  \}}
  
  \FunctionTok{list}\NormalTok{(}\AttributeTok{update =}\NormalTok{ update)}
\NormalTok{\}}

\CommentTok{\# Função para ajustar parâmetros e realizar análise detalhada}
\NormalTok{adjust\_and\_analyze }\OtherTok{\textless{}{-}} \ControlFlowTok{function}\NormalTok{(data\_stream, delta\_values, emb\_dim\_values) \{}
\NormalTok{  results }\OtherTok{\textless{}{-}} \FunctionTok{list}\NormalTok{()}
  
  \ControlFlowTok{for}\NormalTok{ (delta }\ControlFlowTok{in}\NormalTok{ delta\_values) \{}
    \ControlFlowTok{for}\NormalTok{ (emb\_dim }\ControlFlowTok{in}\NormalTok{ emb\_dim\_values) \{}
      \CommentTok{\# Calcular padrões ordinais}
\NormalTok{      patterns }\OtherTok{\textless{}{-}} \FunctionTok{ordinal\_patterns}\NormalTok{(data\_stream, }\AttributeTok{emb\_dim =}\NormalTok{ emb\_dim)}
      
      \CommentTok{\# Inicializa o detector ADWIN}
\NormalTok{      adwin }\OtherTok{\textless{}{-}} \FunctionTok{ADWIN}\NormalTok{(}\AttributeTok{delta =}\NormalTok{ delta)}
      
      \CommentTok{\# Vetor para armazenar os pontos de detecção de mudança}
\NormalTok{      change\_points }\OtherTok{\textless{}{-}} \FunctionTok{numeric}\NormalTok{(}\DecValTok{0}\NormalTok{)}
      
      \CommentTok{\# Processa o fluxo de dados usando padrões ordinais}
      \ControlFlowTok{for}\NormalTok{ (i }\ControlFlowTok{in} \DecValTok{1}\SpecialCharTok{:}\FunctionTok{length}\NormalTok{(patterns)) \{}
        \ControlFlowTok{if}\NormalTok{ (adwin}\SpecialCharTok{$}\FunctionTok{update}\NormalTok{(patterns[i])) \{}
\NormalTok{          change\_points }\OtherTok{\textless{}{-}} \FunctionTok{c}\NormalTok{(change\_points, i }\SpecialCharTok{+}\NormalTok{ emb\_dim }\SpecialCharTok{{-}} \DecValTok{1}\NormalTok{) }\CommentTok{\# Ajuste do índice devido ao embedding dimension}
\NormalTok{        \}}
\NormalTok{      \}}
      
      \CommentTok{\# Armazena resultados}
\NormalTok{      results[[}\FunctionTok{paste}\NormalTok{(}\StringTok{"delta="}\NormalTok{, delta, }\StringTok{"emb\_dim="}\NormalTok{, emb\_dim, }\AttributeTok{sep =} \StringTok{""}\NormalTok{)]] }\OtherTok{\textless{}{-}} \FunctionTok{list}\NormalTok{(}
        \AttributeTok{delta =}\NormalTok{ delta,}
        \AttributeTok{emb\_dim =}\NormalTok{ emb\_dim,}
        \AttributeTok{change\_points =}\NormalTok{ change\_points,}
        \AttributeTok{patterns =}\NormalTok{ patterns}
\NormalTok{      )}
\NormalTok{    \}}
\NormalTok{  \}}
  \FunctionTok{return}\NormalTok{(results)}
\NormalTok{\}}

\CommentTok{\# Exemplo de uso com visualização}
\FunctionTok{set.seed}\NormalTok{(}\DecValTok{123}\NormalTok{) }\CommentTok{\# Define a semente para reprodutibilidade}
\NormalTok{data\_stream }\OtherTok{\textless{}{-}} \FunctionTok{c}\NormalTok{(}\FunctionTok{rnorm}\NormalTok{(}\DecValTok{100}\NormalTok{, }\AttributeTok{mean =} \DecValTok{20}\NormalTok{), }\FunctionTok{rnorm}\NormalTok{(}\DecValTok{100}\NormalTok{, }\AttributeTok{mean =} \DecValTok{25}\NormalTok{)) }\CommentTok{\# Gera dados de exemplo com duas médias diferentes}

\CommentTok{\# Ajustar os parâmetros delta e emb\_dim}
\NormalTok{delta\_values }\OtherTok{\textless{}{-}} \FunctionTok{c}\NormalTok{(}\FloatTok{0.001}\NormalTok{, }\FloatTok{0.002}\NormalTok{, }\FloatTok{0.005}\NormalTok{)}
\NormalTok{emb\_dim\_values }\OtherTok{\textless{}{-}} \FunctionTok{c}\NormalTok{(}\DecValTok{3}\NormalTok{, }\DecValTok{4}\NormalTok{, }\DecValTok{5}\NormalTok{)}
\NormalTok{results }\OtherTok{\textless{}{-}} \FunctionTok{adjust\_and\_analyze}\NormalTok{(data\_stream, delta\_values, emb\_dim\_values)}

\CommentTok{\# Selecionar o melhor resultado para visualização}
\NormalTok{best\_result }\OtherTok{\textless{}{-}}\NormalTok{ results[[}\DecValTok{1}\NormalTok{]] }\CommentTok{\# Aqui, selecionamos o primeiro resultado; você pode adicionar lógica para selecionar o melhor}
\NormalTok{change\_points }\OtherTok{\textless{}{-}}\NormalTok{ best\_result}\SpecialCharTok{$}\NormalTok{change\_points}

\CommentTok{\# Plotar os dados e os pontos de mudança}
\NormalTok{df }\OtherTok{\textless{}{-}} \FunctionTok{data.frame}\NormalTok{(}
  \AttributeTok{index =} \DecValTok{1}\SpecialCharTok{:}\FunctionTok{length}\NormalTok{(data\_stream),}
  \AttributeTok{value =}\NormalTok{ data\_stream,}
  \AttributeTok{change =} \FunctionTok{ifelse}\NormalTok{(}\DecValTok{1}\SpecialCharTok{:}\FunctionTok{length}\NormalTok{(data\_stream) }\SpecialCharTok{\%in\%}\NormalTok{ change\_points, }\StringTok{"Change Detected"}\NormalTok{, }\StringTok{"No Change"}\NormalTok{)}
\NormalTok{)}

\FunctionTok{ggplot}\NormalTok{(df, }\FunctionTok{aes}\NormalTok{(}\AttributeTok{x =}\NormalTok{ index, }\AttributeTok{y =}\NormalTok{ value)) }\SpecialCharTok{+}
  \FunctionTok{geom\_line}\NormalTok{() }\SpecialCharTok{+}
  \FunctionTok{geom\_point}\NormalTok{(}\AttributeTok{data =} \FunctionTok{subset}\NormalTok{(df, change }\SpecialCharTok{==} \StringTok{"Change Detected"}\NormalTok{), }\FunctionTok{aes}\NormalTok{(}\AttributeTok{x =}\NormalTok{ index, }\AttributeTok{y =}\NormalTok{ value), }\AttributeTok{color =} \StringTok{"blue"}\NormalTok{, }\AttributeTok{size =} \DecValTok{2}\NormalTok{) }\SpecialCharTok{+}
  \FunctionTok{labs}\NormalTok{(}\AttributeTok{title =} \StringTok{"Change Detected with ADWIN and OP"}\NormalTok{, }\AttributeTok{x =} \StringTok{"Índice"}\NormalTok{, }\AttributeTok{y =} \StringTok{"Valor"}\NormalTok{) }\SpecialCharTok{+}
  \FunctionTok{theme\_minimal}\NormalTok{() }\SpecialCharTok{+}
  \FunctionTok{theme\_tufte}\NormalTok{()}
\end{Highlighting}
\end{Shaded}

\includegraphics{TrickyTimeSeries_files/figure-latex/unnamed-chunk-8-1.pdf}

\begin{Shaded}
\begin{Highlighting}[]
\CommentTok{\# Análise detalhada dos pontos de mudança}
\FunctionTok{print}\NormalTok{(}\StringTok{"Análise dos Pontos de Mudança:"}\NormalTok{)}
\end{Highlighting}
\end{Shaded}

\begin{verbatim}
## [1] "Análise dos Pontos de Mudança:"
\end{verbatim}

\begin{Shaded}
\begin{Highlighting}[]
\ControlFlowTok{for}\NormalTok{ (point }\ControlFlowTok{in}\NormalTok{ change\_points) \{}
  \FunctionTok{cat}\NormalTok{(}\StringTok{"Ponto de mudança detectado em:"}\NormalTok{, point, }\StringTok{"}\SpecialCharTok{\textbackslash{}n}\StringTok{"}\NormalTok{)}
  \ControlFlowTok{if}\NormalTok{ (point }\SpecialCharTok{\textgreater{}} \DecValTok{1} \SpecialCharTok{\&}\NormalTok{ point }\SpecialCharTok{\textless{}} \FunctionTok{length}\NormalTok{(best\_result}\SpecialCharTok{$}\NormalTok{patterns)) \{}
    \FunctionTok{cat}\NormalTok{(}\StringTok{"Padrões Ordinais antes da mudança:"}\NormalTok{, best\_result}\SpecialCharTok{$}\NormalTok{patterns[(point}\DecValTok{{-}2}\NormalTok{)}\SpecialCharTok{:}\NormalTok{(point}\DecValTok{{-}1}\NormalTok{)], }\StringTok{"}\SpecialCharTok{\textbackslash{}n}\StringTok{"}\NormalTok{)}
    \FunctionTok{cat}\NormalTok{(}\StringTok{"Padrões Ordinais após a mudança:"}\NormalTok{, best\_result}\SpecialCharTok{$}\NormalTok{patterns[(point}\SpecialCharTok{+}\DecValTok{1}\NormalTok{)}\SpecialCharTok{:}\NormalTok{(point}\SpecialCharTok{+}\DecValTok{2}\NormalTok{)], }\StringTok{"}\SpecialCharTok{\textbackslash{}n\textbackslash{}n}\StringTok{"}\NormalTok{)}
\NormalTok{  \}}
\NormalTok{\}}
\end{Highlighting}
\end{Shaded}

\begin{verbatim}
## Ponto de mudança detectado em: 40 
## Padrões Ordinais antes da mudança: 5 5 
## Padrões Ordinais após a mudança: 7 19 
## 
## Ponto de mudança detectado em: 52 
## Padrões Ordinais antes da mudança: 15 5 
## Padrões Ordinais após a mudança: 7 19
\end{verbatim}

\hypertarget{visualizauxe7uxe3o-das-mudanuxe7as}{%
\subsection{Visualização das
Mudanças}\label{visualizauxe7uxe3o-das-mudanuxe7as}}

\hypertarget{sem-padruxf5es-ordinais-op}{%
\subsubsection{Sem Padrões Ordinais
(OP)}\label{sem-padruxf5es-ordinais-op}}

\textbf{Ponto de mudança detectado em: 101:}

\begin{itemize}
\tightlist
\item
  Há uma transição clara nos valores dos dados, de aproximadamente 19
  para 25.
\item
  Esta mudança corresponde à transição na média dos dados que foi
  introduzida no fluxo de dados gerado
  (\texttt{rnorm(100,\ mean\ =\ 20)},
  \texttt{rnorm(100,\ mean\ =\ 25)}).
\end{itemize}

\textbf{Ponto de mudança detectado em: 164 e 180:}

\begin{itemize}
\tightlist
\item
  Os valores antes e depois da mudança são relativamente próximos
  (23-26), indicando que o ADWIN é sensível a pequenas flutuações e
  mudanças na média dos dados.
\item
  Essas mudanças podem ser devidas a variações naturais nos dados ou a
  pequenas mudanças no padrão dos dados.
\end{itemize}

\hypertarget{com-padruxf5es-ordinais-op}{%
\subsubsection{Com Padrões Ordinais
(OP)}\label{com-padruxf5es-ordinais-op}}

\textbf{Ponto de mudança detectado em: 40:}

\begin{itemize}
\tightlist
\item
  Padrões Ordinais antes da mudança são 5, 5, e após a mudança são 7,
  19.
\item
  Indica que há uma mudança na estrutura ordinal dos dados, sugerindo
  uma mudança mais sutil que pode não ser capturada apenas pela média
  dos dados.
\end{itemize}

\textbf{Ponto de mudança detectado em: 52:}

\begin{itemize}
\tightlist
\item
  Padrões Ordinais antes da mudança são 15, 5, e após a mudança são 7,
  19.
\item
  Similar ao ponto de mudança em 40, há uma mudança nos padrões ordinais
  que indica uma mudança na estrutura dos dados.
\end{itemize}

\hypertarget{comparauxe7uxe3o}{%
\subsubsection{Comparação}\label{comparauxe7uxe3o}}

\textbf{Sensibilidade:}

\begin{itemize}
\tightlist
\item
  \textbf{Sem OP:} O ADWIN detecta principalmente mudanças
  significativas na média dos dados.
\item
  \textbf{Com OP:} O ADWIN é capaz de detectar mudanças mais sutis na
  estrutura dos dados, capturadas pelos padrões ordinais.
\end{itemize}

\textbf{Distribuição dos Pontos de Mudança:}

\begin{itemize}
\tightlist
\item
  \textbf{Sem OP:} Mudanças detectadas após a transição principal dos
  dados (em torno do ponto 100).
\item
  \textbf{Com OP:} Mudanças detectadas antes da transição principal,
  sugerindo maior sensibilidade a mudanças estruturais nos dados.
\end{itemize}

\textbf{Robustez à Escala:}

\begin{itemize}
\tightlist
\item
  \textbf{Sem OP:} A detecção de mudanças está mais focada em grandes
  variações na média dos dados.
\item
  \textbf{Com OP:} A detecção de mudanças leva em conta a ordem relativa
  dos valores, tornando-se menos suscetível a grandes variações de
  amplitude.
\end{itemize}

A adição de padrões ordinais melhora a sensibilidade do ADWIN para
detectar mudanças estruturais nos dados, que podem não ser evidentes
pela análise dos valores médios.

Os padrões ordinais ajudam a identificar mudanças sutis e estruturais na
série temporal.

\begin{Shaded}
\begin{Highlighting}[]
\CommentTok{\# Plotar os dados sem OP}
\FunctionTok{ggplot}\NormalTok{(df1, }\FunctionTok{aes}\NormalTok{(}\AttributeTok{x =}\NormalTok{ index, }\AttributeTok{y =}\NormalTok{ value)) }\SpecialCharTok{+}
  \FunctionTok{geom\_line}\NormalTok{() }\SpecialCharTok{+}
  \FunctionTok{geom\_point}\NormalTok{(}\AttributeTok{data =} \FunctionTok{subset}\NormalTok{(df1, change }\SpecialCharTok{==} \StringTok{"Change Detected"}\NormalTok{), }\FunctionTok{aes}\NormalTok{(}\AttributeTok{x =}\NormalTok{ index, }\AttributeTok{y =}\NormalTok{ value), }\AttributeTok{color =} \StringTok{"red"}\NormalTok{, }\AttributeTok{size =} \DecValTok{2}\NormalTok{) }\SpecialCharTok{+}
  \FunctionTok{labs}\NormalTok{(}\AttributeTok{title =} \StringTok{"Change Detected with ADWIN (No OP)"}\NormalTok{, }\AttributeTok{x =} \StringTok{"Índice"}\NormalTok{, }\AttributeTok{y =} \StringTok{"Valor"}\NormalTok{) }\SpecialCharTok{+}
  \FunctionTok{theme\_minimal}\NormalTok{()}
\end{Highlighting}
\end{Shaded}

\includegraphics{TrickyTimeSeries_files/figure-latex/unnamed-chunk-9-1.pdf}

\begin{Shaded}
\begin{Highlighting}[]
\CommentTok{\# Plotar os dados com OP}
\FunctionTok{ggplot}\NormalTok{(df, }\FunctionTok{aes}\NormalTok{(}\AttributeTok{x =}\NormalTok{ index, }\AttributeTok{y =}\NormalTok{ value)) }\SpecialCharTok{+}
  \FunctionTok{geom\_line}\NormalTok{() }\SpecialCharTok{+}
  \FunctionTok{geom\_point}\NormalTok{(}\AttributeTok{data =} \FunctionTok{subset}\NormalTok{(df, change }\SpecialCharTok{==} \StringTok{"Change Detected"}\NormalTok{), }\FunctionTok{aes}\NormalTok{(}\AttributeTok{x =}\NormalTok{ index, }\AttributeTok{y =}\NormalTok{ value), }\AttributeTok{color =} \StringTok{"blue"}\NormalTok{, }\AttributeTok{size =} \DecValTok{2}\NormalTok{) }\SpecialCharTok{+}
  \FunctionTok{labs}\NormalTok{(}\AttributeTok{title =} \StringTok{"Change Detected with ADWIN and OP"}\NormalTok{, }\AttributeTok{x =} \StringTok{"Índice"}\NormalTok{, }\AttributeTok{y =} \StringTok{"Valor"}\NormalTok{) }\SpecialCharTok{+}
  \FunctionTok{theme\_minimal}\NormalTok{()}
\end{Highlighting}
\end{Shaded}

\includegraphics{TrickyTimeSeries_files/figure-latex/unnamed-chunk-9-2.pdf}
\#\# Plano HxC (Entropia de Shannon x Complexidade de Jensen-Shannon)
with OP

Calculando e visualizar a evolução dos padrões ao longo do tempo no
plano HxC utilizando janelas deslizantes e padrões ordinais, vamos
calcular a entropia de Shannon e a complexidade de Jensen-Shannon em
cada janela deslizante da série temporal.

\begin{Shaded}
\begin{Highlighting}[]
\CommentTok{\# Carregar bibliotecas necessárias}
\FunctionTok{library}\NormalTok{(pracma)}
\FunctionTok{library}\NormalTok{(ggplot2)}
\FunctionTok{library}\NormalTok{(ggthemes)}
\FunctionTok{library}\NormalTok{(statcomp)}

\CommentTok{\# Função para calcular padrões ordinais usando statcomp}
\NormalTok{ordinal\_patterns\_statcomp }\OtherTok{\textless{}{-}} \ControlFlowTok{function}\NormalTok{(series, emb\_dim) \{}
  \ControlFlowTok{if}\NormalTok{ (}\FunctionTok{length}\NormalTok{(series) }\SpecialCharTok{\textless{}}\NormalTok{ emb\_dim) \{}
    \FunctionTok{stop}\NormalTok{(}\StringTok{"A série temporal é muito curta para a dimensão de embedding especificada."}\NormalTok{)}
\NormalTok{  \}}
  
  \CommentTok{\# Utilizando a função from \textasciigrave{}statcomp\textasciigrave{} para calcular os padrões ordinais}
\NormalTok{  patterns }\OtherTok{\textless{}{-}} \FunctionTok{ordinal\_pattern}\NormalTok{(series, emb\_dim)}
  
  \FunctionTok{return}\NormalTok{(patterns)}
\NormalTok{\}}

\CommentTok{\# Função para calcular a entropia de Shannon}
\NormalTok{shannon\_entropy }\OtherTok{\textless{}{-}} \ControlFlowTok{function}\NormalTok{(probabilities) \{}
  \SpecialCharTok{{-}}\FunctionTok{sum}\NormalTok{(probabilities }\SpecialCharTok{*} \FunctionTok{log2}\NormalTok{(probabilities), }\AttributeTok{na.rm =} \ConstantTok{TRUE}\NormalTok{)}
\NormalTok{\}}

\CommentTok{\# Função para calcular a complexidade de Jensen{-}Shannon}
\NormalTok{js\_complexity }\OtherTok{\textless{}{-}} \ControlFlowTok{function}\NormalTok{(probabilities) \{}
\NormalTok{  q }\OtherTok{\textless{}{-}} \FunctionTok{rep}\NormalTok{(}\DecValTok{1}\SpecialCharTok{/}\FunctionTok{length}\NormalTok{(probabilities), }\FunctionTok{length}\NormalTok{(probabilities))}
\NormalTok{  m }\OtherTok{\textless{}{-}}\NormalTok{ (probabilities }\SpecialCharTok{+}\NormalTok{ q) }\SpecialCharTok{/} \DecValTok{2}
\NormalTok{  (}\FunctionTok{shannon\_entropy}\NormalTok{(m) }\SpecialCharTok{{-}} \FloatTok{0.5} \SpecialCharTok{*}\NormalTok{ (}\FunctionTok{shannon\_entropy}\NormalTok{(probabilities) }\SpecialCharTok{+} \FunctionTok{shannon\_entropy}\NormalTok{(q))) }\SpecialCharTok{/} \FunctionTok{log2}\NormalTok{(}\FunctionTok{length}\NormalTok{(probabilities))}
\NormalTok{\}}

\CommentTok{\# Função para calcular probabilidades dos padrões ordinais}
\NormalTok{calculate\_probabilities }\OtherTok{\textless{}{-}} \ControlFlowTok{function}\NormalTok{(patterns) \{}
\NormalTok{  table\_patterns }\OtherTok{\textless{}{-}} \FunctionTok{table}\NormalTok{(patterns)}
\NormalTok{  probabilities }\OtherTok{\textless{}{-}} \FunctionTok{as.numeric}\NormalTok{(table\_patterns) }\SpecialCharTok{/} \FunctionTok{sum}\NormalTok{(table\_patterns)}
  \FunctionTok{return}\NormalTok{(probabilities)}
\NormalTok{\}}

\CommentTok{\# Função para calcular H e C em janelas deslizantes com padrões ordinais}
\NormalTok{calculate\_hxc\_with\_op }\OtherTok{\textless{}{-}} \ControlFlowTok{function}\NormalTok{(series, window\_size, emb\_dim) \{}
\NormalTok{  n }\OtherTok{\textless{}{-}} \FunctionTok{length}\NormalTok{(series)}
\NormalTok{  h\_values }\OtherTok{\textless{}{-}} \FunctionTok{c}\NormalTok{()}
\NormalTok{  c\_values }\OtherTok{\textless{}{-}} \FunctionTok{c}\NormalTok{()}
  
  \ControlFlowTok{for}\NormalTok{ (i }\ControlFlowTok{in} \DecValTok{1}\SpecialCharTok{:}\NormalTok{(n }\SpecialCharTok{{-}}\NormalTok{ window\_size }\SpecialCharTok{+} \DecValTok{1}\NormalTok{)) \{}
\NormalTok{    window }\OtherTok{\textless{}{-}}\NormalTok{ series[i}\SpecialCharTok{:}\NormalTok{(i }\SpecialCharTok{+}\NormalTok{ window\_size }\SpecialCharTok{{-}} \DecValTok{1}\NormalTok{)]}
\NormalTok{    patterns }\OtherTok{\textless{}{-}} \FunctionTok{ordinal\_patterns}\NormalTok{(window, emb\_dim)}
\NormalTok{    probabilities }\OtherTok{\textless{}{-}} \FunctionTok{calculate\_probabilities}\NormalTok{(patterns)}
\NormalTok{    h }\OtherTok{\textless{}{-}} \FunctionTok{shannon\_entropy}\NormalTok{(probabilities)}
\NormalTok{    c }\OtherTok{\textless{}{-}} \FunctionTok{js\_complexity}\NormalTok{(probabilities)}
\NormalTok{    h\_values }\OtherTok{\textless{}{-}} \FunctionTok{c}\NormalTok{(h\_values, h)}
\NormalTok{    c\_values }\OtherTok{\textless{}{-}} \FunctionTok{c}\NormalTok{(c\_values, c)}
\NormalTok{  \}}
  
  \FunctionTok{data.frame}\NormalTok{(}\AttributeTok{H =}\NormalTok{ h\_values, }\AttributeTok{C =}\NormalTok{ c\_values)}
\NormalTok{\}}

\CommentTok{\# Função ADWIN para detectar mudanças de conceito em fluxos de dados}
\NormalTok{ADWIN }\OtherTok{\textless{}{-}} \ControlFlowTok{function}\NormalTok{(}\AttributeTok{delta =} \FloatTok{0.002}\NormalTok{) \{}
  \CommentTok{\# Inicializa as variáveis}
\NormalTok{  width }\OtherTok{\textless{}{-}} \DecValTok{0} \CommentTok{\# Tamanho da janela}
\NormalTok{  total }\OtherTok{\textless{}{-}} \DecValTok{0} \CommentTok{\# Soma dos valores na janela}
\NormalTok{  variance }\OtherTok{\textless{}{-}} \DecValTok{0} \CommentTok{\# Variância dos valores na janela}
\NormalTok{  window }\OtherTok{\textless{}{-}} \FunctionTok{numeric}\NormalTok{(}\DecValTok{0}\NormalTok{) }\CommentTok{\# Vetor que armazena os valores na janela}
  
  \CommentTok{\# Função para atualizar o ADWIN com um novo valor}
\NormalTok{  update }\OtherTok{\textless{}{-}} \ControlFlowTok{function}\NormalTok{(value) \{}
\NormalTok{    width }\OtherTok{\textless{}\textless{}{-}}\NormalTok{ width }\SpecialCharTok{+} \DecValTok{1} \CommentTok{\# Incrementa o tamanho da janela}
\NormalTok{    window }\OtherTok{\textless{}\textless{}{-}} \FunctionTok{c}\NormalTok{(window, value) }\CommentTok{\# Adiciona o novo valor à janela}
\NormalTok{    total }\OtherTok{\textless{}\textless{}{-}}\NormalTok{ total }\SpecialCharTok{+}\NormalTok{ value }\CommentTok{\# Atualiza a soma total}
    \ControlFlowTok{if}\NormalTok{ (width }\SpecialCharTok{\textgreater{}} \DecValTok{1}\NormalTok{) \{}
\NormalTok{      variance }\OtherTok{\textless{}\textless{}{-}} \FunctionTok{var}\NormalTok{(window) }\CommentTok{\# Calcula a variância se a janela tiver mais de um valor}
\NormalTok{    \}}
    \CommentTok{\# Checa por concept drift}
    \ControlFlowTok{if}\NormalTok{ (width }\SpecialCharTok{\textgreater{}} \DecValTok{1} \SpecialCharTok{\&\&} \FunctionTok{detect\_change}\NormalTok{()) \{}
      \FunctionTok{return}\NormalTok{(}\ConstantTok{TRUE}\NormalTok{) }\CommentTok{\# Retorna TRUE se uma mudança for detectada}
\NormalTok{    \}}
    \FunctionTok{return}\NormalTok{(}\ConstantTok{FALSE}\NormalTok{) }\CommentTok{\# Retorna FALSE se nenhuma mudança for detectada}
\NormalTok{  \}}
  
  \CommentTok{\# Função para detectar mudança}
\NormalTok{  detect\_change }\OtherTok{\textless{}{-}} \ControlFlowTok{function}\NormalTok{() \{}
\NormalTok{    mean\_val }\OtherTok{\textless{}{-}} \FunctionTok{mean}\NormalTok{(window) }\CommentTok{\# Calcula a média dos valores na janela}
    \ControlFlowTok{for}\NormalTok{ (n }\ControlFlowTok{in} \DecValTok{1}\SpecialCharTok{:}\NormalTok{(width }\SpecialCharTok{{-}} \DecValTok{1}\NormalTok{)) \{}
      \CommentTok{\# Divide a janela em duas sub{-}janelas e calcula as médias de cada sub{-}janela}
\NormalTok{      mean0 }\OtherTok{\textless{}{-}} \FunctionTok{mean}\NormalTok{(window[}\DecValTok{1}\SpecialCharTok{:}\NormalTok{n])}
\NormalTok{      mean1 }\OtherTok{\textless{}{-}} \FunctionTok{mean}\NormalTok{(window[(n }\SpecialCharTok{+} \DecValTok{1}\NormalTok{)}\SpecialCharTok{:}\NormalTok{width])}
      \CommentTok{\# Compara as médias das sub{-}janelas usando um teste estatístico}
      \ControlFlowTok{if}\NormalTok{ (}\FunctionTok{abs}\NormalTok{(mean0 }\SpecialCharTok{{-}}\NormalTok{ mean1) }\SpecialCharTok{\textgreater{}} \FunctionTok{sqrt}\NormalTok{((variance }\SpecialCharTok{/}\NormalTok{ n) }\SpecialCharTok{+}\NormalTok{ (variance }\SpecialCharTok{/}\NormalTok{ (width }\SpecialCharTok{{-}}\NormalTok{ n))) }\SpecialCharTok{*} \FunctionTok{qnorm}\NormalTok{(}\DecValTok{1} \SpecialCharTok{{-}}\NormalTok{ delta)) \{}
        \CommentTok{\# Se uma mudança for detectada, ajusta a janela para descartar os dados antigos}
\NormalTok{        window }\OtherTok{\textless{}\textless{}{-}}\NormalTok{ window[(n }\SpecialCharTok{+} \DecValTok{1}\NormalTok{)}\SpecialCharTok{:}\NormalTok{width]}
\NormalTok{        width }\OtherTok{\textless{}\textless{}{-}} \FunctionTok{length}\NormalTok{(window)}
\NormalTok{        total }\OtherTok{\textless{}\textless{}{-}} \FunctionTok{sum}\NormalTok{(window)}
\NormalTok{        variance }\OtherTok{\textless{}\textless{}{-}} \FunctionTok{var}\NormalTok{(window)}
        \FunctionTok{return}\NormalTok{(}\ConstantTok{TRUE}\NormalTok{) }\CommentTok{\# Retorna TRUE indicando que uma mudança foi detectada}
\NormalTok{      \}}
\NormalTok{    \}}
    \FunctionTok{return}\NormalTok{(}\ConstantTok{FALSE}\NormalTok{) }\CommentTok{\# Retorna FALSE se nenhuma mudança for detectada}
\NormalTok{  \}}
  
  \CommentTok{\# Retorna as funções de atualização e detecção}
  \FunctionTok{list}\NormalTok{(}\AttributeTok{update =}\NormalTok{ update)}
\NormalTok{\}}

\CommentTok{\# Função para ajustar parâmetros e realizar análise detalhada}
\NormalTok{adjust\_and\_analyze }\OtherTok{\textless{}{-}} \ControlFlowTok{function}\NormalTok{(data\_stream, delta\_values, emb\_dim\_values) \{}
\NormalTok{  results }\OtherTok{\textless{}{-}} \FunctionTok{list}\NormalTok{()}
  
  \ControlFlowTok{for}\NormalTok{ (delta }\ControlFlowTok{in}\NormalTok{ delta\_values) \{}
    \ControlFlowTok{for}\NormalTok{ (emb\_dim }\ControlFlowTok{in}\NormalTok{ emb\_dim\_values) \{}
      \CommentTok{\# Calcular padrões ordinais}
\NormalTok{      patterns }\OtherTok{\textless{}{-}} \FunctionTok{ordinal\_patterns}\NormalTok{(data\_stream, }\AttributeTok{emb\_dim =}\NormalTok{ emb\_dim)}
      
      \CommentTok{\# Inicializa o detector ADWIN}
\NormalTok{      adwin }\OtherTok{\textless{}{-}} \FunctionTok{ADWIN}\NormalTok{(}\AttributeTok{delta =}\NormalTok{ delta)}
      
      \CommentTok{\# Vetor para armazenar os pontos de detecção de mudança}
\NormalTok{      change\_points }\OtherTok{\textless{}{-}} \FunctionTok{numeric}\NormalTok{(}\DecValTok{0}\NormalTok{)}
      
      \CommentTok{\# Processa o fluxo de dados usando padrões ordinais}
      \ControlFlowTok{for}\NormalTok{ (i }\ControlFlowTok{in} \DecValTok{1}\SpecialCharTok{:}\FunctionTok{length}\NormalTok{(patterns)) \{}
        \ControlFlowTok{if}\NormalTok{ (adwin}\SpecialCharTok{$}\FunctionTok{update}\NormalTok{(patterns[i])) \{}
\NormalTok{          change\_points }\OtherTok{\textless{}{-}} \FunctionTok{c}\NormalTok{(change\_points, i }\SpecialCharTok{+}\NormalTok{ emb\_dim }\SpecialCharTok{{-}} \DecValTok{1}\NormalTok{) }\CommentTok{\# Ajuste do índice devido ao embedding dimension}
\NormalTok{        \}}
\NormalTok{      \}}
      
      \CommentTok{\# Armazena resultados}
\NormalTok{      results[[}\FunctionTok{paste}\NormalTok{(}\StringTok{"delta="}\NormalTok{, delta, }\StringTok{"emb\_dim="}\NormalTok{, emb\_dim, }\AttributeTok{sep =} \StringTok{""}\NormalTok{)]] }\OtherTok{\textless{}{-}} \FunctionTok{list}\NormalTok{(}
        \AttributeTok{delta =}\NormalTok{ delta,}
        \AttributeTok{emb\_dim =}\NormalTok{ emb\_dim,}
        \AttributeTok{change\_points =}\NormalTok{ change\_points,}
        \AttributeTok{patterns =}\NormalTok{ patterns}
\NormalTok{      )}
\NormalTok{    \}}
\NormalTok{  \}}
  \FunctionTok{return}\NormalTok{(results)}
\NormalTok{\}}

\CommentTok{\# Exemplo de uso com visualização}
\FunctionTok{set.seed}\NormalTok{(}\DecValTok{123}\NormalTok{) }\CommentTok{\# Define a semente para reprodutibilidade}
\NormalTok{data\_stream }\OtherTok{\textless{}{-}} \FunctionTok{c}\NormalTok{(}\FunctionTok{rnorm}\NormalTok{(}\DecValTok{100}\NormalTok{, }\AttributeTok{mean =} \DecValTok{20}\NormalTok{), }\FunctionTok{rnorm}\NormalTok{(}\DecValTok{100}\NormalTok{, }\AttributeTok{mean =} \DecValTok{25}\NormalTok{)) }\CommentTok{\# Gera dados de exemplo com duas médias diferentes}

\CommentTok{\# Ajustar os parâmetros delta e emb\_dim}
\NormalTok{delta\_values }\OtherTok{\textless{}{-}} \FunctionTok{c}\NormalTok{(}\FloatTok{0.001}\NormalTok{, }\FloatTok{0.002}\NormalTok{, }\FloatTok{0.005}\NormalTok{)}
\NormalTok{emb\_dim\_values }\OtherTok{\textless{}{-}} \FunctionTok{c}\NormalTok{(}\DecValTok{3}\NormalTok{, }\DecValTok{4}\NormalTok{, }\DecValTok{5}\NormalTok{)}
\NormalTok{results }\OtherTok{\textless{}{-}} \FunctionTok{adjust\_and\_analyze}\NormalTok{(data\_stream, delta\_values, emb\_dim\_values)}

\CommentTok{\# Selecionar o melhor resultado para visualização}
\NormalTok{best\_result }\OtherTok{\textless{}{-}}\NormalTok{ results[[}\DecValTok{1}\NormalTok{]] }\CommentTok{\# Aqui, selecionamos o primeiro resultado; você pode adicionar lógica para selecionar o melhor}
\NormalTok{change\_points }\OtherTok{\textless{}{-}}\NormalTok{ best\_result}\SpecialCharTok{$}\NormalTok{change\_points}

\CommentTok{\# Calcular H e C em janelas deslizantes com padrões ordinais}
\NormalTok{window\_size }\OtherTok{\textless{}{-}} \DecValTok{50}
\NormalTok{emb\_dim }\OtherTok{\textless{}{-}} \DecValTok{3}
\NormalTok{hxc\_data }\OtherTok{\textless{}{-}} \FunctionTok{calculate\_hxc\_with\_op}\NormalTok{(data\_stream, window\_size, emb\_dim)}

\CommentTok{\# Plotar o plano HxC}
\FunctionTok{ggplot}\NormalTok{(hxc\_data, }\FunctionTok{aes}\NormalTok{(}\AttributeTok{x =}\NormalTok{ H, }\AttributeTok{y =}\NormalTok{ C)) }\SpecialCharTok{+}
  \FunctionTok{geom\_point}\NormalTok{(}\AttributeTok{color =} \StringTok{"blue"}\NormalTok{) }\SpecialCharTok{+}
  \FunctionTok{labs}\NormalTok{(}\AttributeTok{title =} \StringTok{"Plano HxC with ADWIN and Ordinal Patterns"}\NormalTok{,}
       \AttributeTok{x =} \StringTok{"Entropia de Shannon (H)"}\NormalTok{, }\AttributeTok{y =} \StringTok{"Complexidade de Jensen{-}Shannon (C)"}\NormalTok{) }\SpecialCharTok{+}
  \FunctionTok{theme\_minimal}\NormalTok{() }\SpecialCharTok{+}
  \FunctionTok{geom\_smooth}\NormalTok{(}\AttributeTok{method =} \StringTok{"lm"}\NormalTok{, }\AttributeTok{color =} \StringTok{"red"}\NormalTok{) }\SpecialCharTok{+} \CommentTok{\# Adiciona uma linha de tendência}
  \FunctionTok{theme\_bw}\NormalTok{() }\CommentTok{\# Utiliza um tema com fundo branco para maior clareza}
\end{Highlighting}
\end{Shaded}

\begin{verbatim}
## `geom_smooth()` using formula = 'y ~ x'
\end{verbatim}

\includegraphics{TrickyTimeSeries_files/figure-latex/unnamed-chunk-10-1.pdf}

\begin{Shaded}
\begin{Highlighting}[]
\CommentTok{\# Criar um data frame com os dados e os pontos de mudança}
\NormalTok{df }\OtherTok{\textless{}{-}} \FunctionTok{data.frame}\NormalTok{(}
  \AttributeTok{index =} \DecValTok{1}\SpecialCharTok{:}\FunctionTok{length}\NormalTok{(data\_stream),}
  \AttributeTok{value =}\NormalTok{ data\_stream,}
  \AttributeTok{change =} \FunctionTok{ifelse}\NormalTok{(}\DecValTok{1}\SpecialCharTok{:}\FunctionTok{length}\NormalTok{(data\_stream) }\SpecialCharTok{\%in\%}\NormalTok{ change\_points, }\StringTok{"Change Detected"}\NormalTok{, }\StringTok{"No Change"}\NormalTok{)}
\NormalTok{)}

\CommentTok{\# Plotar os dados e os pontos de mudança}
\FunctionTok{ggplot}\NormalTok{(df, }\FunctionTok{aes}\NormalTok{(}\AttributeTok{x =}\NormalTok{ index, }\AttributeTok{y =}\NormalTok{ value)) }\SpecialCharTok{+}
  \FunctionTok{geom\_line}\NormalTok{() }\SpecialCharTok{+}
  \FunctionTok{geom\_point}\NormalTok{(}\AttributeTok{data =} \FunctionTok{subset}\NormalTok{(df, change }\SpecialCharTok{==} \StringTok{"Change Detected"}\NormalTok{), }\FunctionTok{aes}\NormalTok{(}\AttributeTok{x =}\NormalTok{ index, }\AttributeTok{y =}\NormalTok{ value), }\AttributeTok{color =} \StringTok{"blue"}\NormalTok{, }\AttributeTok{size =} \DecValTok{2}\NormalTok{) }\SpecialCharTok{+}
  \FunctionTok{labs}\NormalTok{(}\AttributeTok{title =} \StringTok{"Change Detected with ADWIN and Ordinal Patterns"}\NormalTok{, }\AttributeTok{x =} \StringTok{"Índice"}\NormalTok{, }\AttributeTok{y =} \StringTok{"Valor"}\NormalTok{) }\SpecialCharTok{+}
  \FunctionTok{theme\_minimal}\NormalTok{() }\SpecialCharTok{+}
  \FunctionTok{theme\_tufte}\NormalTok{()}
\end{Highlighting}
\end{Shaded}

\includegraphics{TrickyTimeSeries_files/figure-latex/unnamed-chunk-10-2.pdf}

\begin{Shaded}
\begin{Highlighting}[]
\CommentTok{\# Análise detalhada dos pontos de mudança}
\FunctionTok{print}\NormalTok{(}\StringTok{"Análise dos Pontos de Mudança:"}\NormalTok{)}
\end{Highlighting}
\end{Shaded}

\begin{verbatim}
## [1] "Análise dos Pontos de Mudança:"
\end{verbatim}

\begin{Shaded}
\begin{Highlighting}[]
\ControlFlowTok{for}\NormalTok{ (point }\ControlFlowTok{in}\NormalTok{ change\_points) \{}
  \FunctionTok{cat}\NormalTok{(}\StringTok{"Ponto de mudança detectado em:"}\NormalTok{, point, }\StringTok{"}\SpecialCharTok{\textbackslash{}n}\StringTok{"}\NormalTok{)}
  \ControlFlowTok{if}\NormalTok{ (point }\SpecialCharTok{\textgreater{}} \DecValTok{1} \SpecialCharTok{\&}\NormalTok{ point }\SpecialCharTok{\textless{}} \FunctionTok{length}\NormalTok{(best\_result}\SpecialCharTok{$}\NormalTok{patterns)) \{}
    \FunctionTok{cat}\NormalTok{(}\StringTok{"Padrões Ordinais antes da mudança:"}\NormalTok{, best\_result}\SpecialCharTok{$}\NormalTok{patterns[(point}\DecValTok{{-}2}\NormalTok{)}\SpecialCharTok{:}\NormalTok{(point}\DecValTok{{-}1}\NormalTok{)], }\StringTok{"}\SpecialCharTok{\textbackslash{}n}\StringTok{"}\NormalTok{)}
    \FunctionTok{cat}\NormalTok{(}\StringTok{"Padrões Ordinais após a mudança:"}\NormalTok{, best\_result}\SpecialCharTok{$}\NormalTok{patterns[(point}\SpecialCharTok{+}\DecValTok{1}\NormalTok{)}\SpecialCharTok{:}\NormalTok{(point}\SpecialCharTok{+}\DecValTok{2}\NormalTok{)], }\StringTok{"}\SpecialCharTok{\textbackslash{}n\textbackslash{}n}\StringTok{"}\NormalTok{)}
\NormalTok{  \}}
\NormalTok{\}}
\end{Highlighting}
\end{Shaded}

\begin{verbatim}
## Ponto de mudança detectado em: 40 
## Padrões Ordinais antes da mudança: 5 5 
## Padrões Ordinais após a mudança: 7 19 
## 
## Ponto de mudança detectado em: 52 
## Padrões Ordinais antes da mudança: 15 5 
## Padrões Ordinais após a mudança: 7 19
\end{verbatim}

\hypertarget{algoritmo-combinado-para-detecuxe7uxe3o-de-mudanuxe7as}{%
\subsection{Algoritmo Combinado para Detecção de
Mudanças}\label{algoritmo-combinado-para-detecuxe7uxe3o-de-mudanuxe7as}}

Adaptação do ADWIN para detectar mudanças tanto na média dos dados
quanto nos padrões ordinais. Esse algoritmo permitirá \emph{detectar
mudanças significativas na média dos dados e mudanças sutis na estrutura
dos dados}.

Passos do Algoritmo:

\begin{enumerate}
\def\labelenumi{\arabic{enumi}.}
\tightlist
\item
  Calcular padrões ordinais da série temporal.
\item
  Aplicar ADWIN aos dados brutos para detectar mudanças na média.
\item
  Aplicar ADWIN aos padrões ordinais para detectar mudanças na estrutura
  dos dados.
\item
  Combinar os pontos de mudança detectados por ambos os métodos.
\end{enumerate}

\begin{Shaded}
\begin{Highlighting}[]
\CommentTok{\# Carregar bibliotecas necessárias}
\FunctionTok{library}\NormalTok{(pracma)}
\FunctionTok{library}\NormalTok{(ggplot2)}
\FunctionTok{library}\NormalTok{(ggthemes)}
\FunctionTok{library}\NormalTok{(statcomp)}

\CommentTok{\# Função para calcular padrões ordinais}
\CommentTok{\# Esta função calcula padrões ordinais para uma série temporal dada uma dimensão de embedding.}
\CommentTok{\# Patterns: Vetor com os padrões ordinais calculados}
\CommentTok{\# series: A série temporal de entrada}
\CommentTok{\# emb\_dim: Dimensão de embedding (por exemplo, 3)}

\CommentTok{\# ordinal\_patterns \textless{}{-} function(series, emb\_dim) \{}
\CommentTok{\#   n \textless{}{-} length(series)}
\CommentTok{\#   if (n \textless{} emb\_dim) \{}
\CommentTok{\#     stop("A série temporal é muito curta para a dimensão de embedding especificada.")}
\CommentTok{\#   \}}
\CommentTok{\#   }
\CommentTok{\#   patterns \textless{}{-} numeric(n {-} emb\_dim + 1)}
\CommentTok{\#   for (i in 1:(n {-} emb\_dim + 1)) \{}
\CommentTok{\#     subseq \textless{}{-} series[i:(i + emb\_dim {-} 1)]}
\CommentTok{\#     ranks \textless{}{-} rank(subseq, ties.method = "first")}
\CommentTok{\#     pattern \textless{}{-} sum((ranks {-} 1) * (emb\_dim \^{} (0:(emb\_dim {-} 1))))}
\CommentTok{\#     patterns[i] \textless{}{-} pattern}
\CommentTok{\#   \}}
\CommentTok{\#   return(patterns)}
\CommentTok{\# \}}

\CommentTok{\# Função para calcular padrões ordinais usando statcomp}
\NormalTok{ordinal\_patterns\_statcomp }\OtherTok{\textless{}{-}} \ControlFlowTok{function}\NormalTok{(series, emb\_dim) \{}
  \ControlFlowTok{if}\NormalTok{ (}\FunctionTok{length}\NormalTok{(series) }\SpecialCharTok{\textless{}}\NormalTok{ emb\_dim) \{}
    \FunctionTok{stop}\NormalTok{(}\StringTok{"A série temporal é muito curta para a dimensão de embedding especificada."}\NormalTok{)}
\NormalTok{  \}}
  
  \CommentTok{\# Utilizando a função from \textasciigrave{}statcomp\textasciigrave{} para calcular os padrões ordinais}
\NormalTok{  patterns }\OtherTok{\textless{}{-}} \FunctionTok{ordinal\_pattern}\NormalTok{(series, emb\_dim)}
  
  \FunctionTok{return}\NormalTok{(patterns)}
\NormalTok{\}}

\CommentTok{\# Função ADWIN para detectar mudanças de conceito em fluxos de dados}
\CommentTok{\# Esta função implementa o ADWIN que ajusta dinamicamente o tamanho de uma janela deslizante de dados e verifica se houve uma mudança significativa na distribuição dos dados.}
\CommentTok{\# delta: Parâmetro de sensibilidade para detectar mudanças}
\NormalTok{ADWIN }\OtherTok{\textless{}{-}} \ControlFlowTok{function}\NormalTok{(}\AttributeTok{delta =} \FloatTok{0.002}\NormalTok{) \{}
\NormalTok{  width }\OtherTok{\textless{}{-}} \DecValTok{0} \CommentTok{\# Tamanho da janela}
\NormalTok{  total }\OtherTok{\textless{}{-}} \DecValTok{0} \CommentTok{\# Soma dos valores na janela}
\NormalTok{  variance }\OtherTok{\textless{}{-}} \DecValTok{0} \CommentTok{\# Variância dos valores na janela}
\NormalTok{  window }\OtherTok{\textless{}{-}} \FunctionTok{numeric}\NormalTok{(}\DecValTok{0}\NormalTok{) }\CommentTok{\# Vetor que armazena os valores na janela}
  
  \CommentTok{\# Função para atualizar o ADWIN com um novo valor}
  \CommentTok{\# value: Novo valor a ser adicionado à janela}
\NormalTok{  update }\OtherTok{\textless{}{-}} \ControlFlowTok{function}\NormalTok{(value) \{}
\NormalTok{    width }\OtherTok{\textless{}\textless{}{-}}\NormalTok{ width }\SpecialCharTok{+} \DecValTok{1}
\NormalTok{    window }\OtherTok{\textless{}\textless{}{-}} \FunctionTok{c}\NormalTok{(window, value)}
\NormalTok{    total }\OtherTok{\textless{}\textless{}{-}}\NormalTok{ total }\SpecialCharTok{+}\NormalTok{ value}
    \ControlFlowTok{if}\NormalTok{ (width }\SpecialCharTok{\textgreater{}} \DecValTok{1}\NormalTok{) \{}
\NormalTok{      variance }\OtherTok{\textless{}\textless{}{-}} \FunctionTok{var}\NormalTok{(window, }\AttributeTok{na.rm =} \ConstantTok{TRUE}\NormalTok{)}
\NormalTok{    \}}
    \CommentTok{\# Checa por concept drift}
    \ControlFlowTok{if}\NormalTok{ (width }\SpecialCharTok{\textgreater{}} \DecValTok{1} \SpecialCharTok{\&\&} \FunctionTok{detect\_change}\NormalTok{()) \{}
      \FunctionTok{return}\NormalTok{(}\ConstantTok{TRUE}\NormalTok{)}
\NormalTok{    \}}
    \FunctionTok{return}\NormalTok{(}\ConstantTok{FALSE}\NormalTok{)}
\NormalTok{  \}}
  
  \CommentTok{\# Função para detectar mudança}
\NormalTok{  detect\_change }\OtherTok{\textless{}{-}} \ControlFlowTok{function}\NormalTok{() \{}
\NormalTok{    mean\_val }\OtherTok{\textless{}{-}} \FunctionTok{mean}\NormalTok{(window, }\AttributeTok{na.rm =} \ConstantTok{TRUE}\NormalTok{)}
    \ControlFlowTok{for}\NormalTok{ (n }\ControlFlowTok{in} \DecValTok{1}\SpecialCharTok{:}\NormalTok{(width }\SpecialCharTok{{-}} \DecValTok{1}\NormalTok{)) \{}
\NormalTok{      mean0 }\OtherTok{\textless{}{-}} \FunctionTok{mean}\NormalTok{(window[}\DecValTok{1}\SpecialCharTok{:}\NormalTok{n], }\AttributeTok{na.rm =} \ConstantTok{TRUE}\NormalTok{)}
\NormalTok{      mean1 }\OtherTok{\textless{}{-}} \FunctionTok{mean}\NormalTok{(window[(n }\SpecialCharTok{+} \DecValTok{1}\NormalTok{)}\SpecialCharTok{:}\NormalTok{width], }\AttributeTok{na.rm =} \ConstantTok{TRUE}\NormalTok{)}
      \ControlFlowTok{if}\NormalTok{ (}\FunctionTok{is.na}\NormalTok{(mean0) }\SpecialCharTok{||} \FunctionTok{is.na}\NormalTok{(mean1)) }\ControlFlowTok{next}
      \ControlFlowTok{if}\NormalTok{ (}\FunctionTok{abs}\NormalTok{(mean0 }\SpecialCharTok{{-}}\NormalTok{ mean1) }\SpecialCharTok{\textgreater{}} \FunctionTok{sqrt}\NormalTok{((variance }\SpecialCharTok{/}\NormalTok{ n) }\SpecialCharTok{+}\NormalTok{ (variance }\SpecialCharTok{/}\NormalTok{ (width }\SpecialCharTok{{-}}\NormalTok{ n))) }\SpecialCharTok{*} \FunctionTok{qnorm}\NormalTok{(}\DecValTok{1} \SpecialCharTok{{-}}\NormalTok{ delta)) \{}
\NormalTok{        window }\OtherTok{\textless{}\textless{}{-}}\NormalTok{ window[(n }\SpecialCharTok{+} \DecValTok{1}\NormalTok{)}\SpecialCharTok{:}\NormalTok{width]}
\NormalTok{        width }\OtherTok{\textless{}\textless{}{-}} \FunctionTok{length}\NormalTok{(window)}
\NormalTok{        total }\OtherTok{\textless{}\textless{}{-}} \FunctionTok{sum}\NormalTok{(window, }\AttributeTok{na.rm =} \ConstantTok{TRUE}\NormalTok{)}
\NormalTok{        variance }\OtherTok{\textless{}\textless{}{-}} \FunctionTok{var}\NormalTok{(window, }\AttributeTok{na.rm =} \ConstantTok{TRUE}\NormalTok{)}
        \FunctionTok{return}\NormalTok{(}\ConstantTok{TRUE}\NormalTok{)}
\NormalTok{      \}}
\NormalTok{    \}}
    \FunctionTok{return}\NormalTok{(}\ConstantTok{FALSE}\NormalTok{)}
\NormalTok{  \}}
  
  \FunctionTok{list}\NormalTok{(}\AttributeTok{update =}\NormalTok{ update)}
\NormalTok{\}}

\CommentTok{\# Exemplo }
\FunctionTok{set.seed}\NormalTok{(}\DecValTok{123}\NormalTok{) }\CommentTok{\# Define a semente para reprodutibilidade}
\NormalTok{data\_stream }\OtherTok{\textless{}{-}} \FunctionTok{c}\NormalTok{(}\FunctionTok{rnorm}\NormalTok{(}\DecValTok{100}\NormalTok{, }\AttributeTok{mean =} \DecValTok{20}\NormalTok{), }\FunctionTok{rnorm}\NormalTok{(}\DecValTok{100}\NormalTok{, }\AttributeTok{mean =} \DecValTok{25}\NormalTok{)) }\CommentTok{\# Gera dados de exemplo com duas médias diferentes}

\CommentTok{\# Inicializa detectores ADWIN}
\NormalTok{adwin\_data }\OtherTok{\textless{}{-}} \FunctionTok{ADWIN}\NormalTok{(}\AttributeTok{delta =} \FloatTok{0.002}\NormalTok{)}
\NormalTok{adwin\_patterns }\OtherTok{\textless{}{-}} \FunctionTok{ADWIN}\NormalTok{(}\AttributeTok{delta =} \FloatTok{0.002}\NormalTok{)}

\CommentTok{\# Calcular padrões ordinais}
\NormalTok{patterns }\OtherTok{\textless{}{-}} \FunctionTok{ordinal\_patterns}\NormalTok{(data\_stream, }\AttributeTok{emb\_dim =} \DecValTok{3}\NormalTok{)}

\CommentTok{\# Vetores para armazenar os pontos de detecção de mudança}
\NormalTok{change\_points\_data }\OtherTok{\textless{}{-}} \FunctionTok{numeric}\NormalTok{(}\DecValTok{0}\NormalTok{)}
\NormalTok{change\_points\_patterns }\OtherTok{\textless{}{-}} \FunctionTok{numeric}\NormalTok{(}\DecValTok{0}\NormalTok{)}

\CommentTok{\# Processa o fluxo de dados}
\CommentTok{\# Para cada ponto na série temporal, atualiza o ADWIN para dados brutos e para padrões ordinais}
\ControlFlowTok{for}\NormalTok{ (i }\ControlFlowTok{in} \DecValTok{1}\SpecialCharTok{:}\FunctionTok{length}\NormalTok{(data\_stream)) \{}
  \ControlFlowTok{if}\NormalTok{ (adwin\_data}\SpecialCharTok{$}\FunctionTok{update}\NormalTok{(data\_stream[i])) \{}
\NormalTok{    change\_points\_data }\OtherTok{\textless{}{-}} \FunctionTok{c}\NormalTok{(change\_points\_data, i)}
\NormalTok{  \}}
  \ControlFlowTok{if}\NormalTok{ (i }\SpecialCharTok{\textless{}=} \FunctionTok{length}\NormalTok{(patterns) }\SpecialCharTok{\&\&}\NormalTok{ adwin\_patterns}\SpecialCharTok{$}\FunctionTok{update}\NormalTok{(patterns[i])) \{}
\NormalTok{    change\_points\_patterns }\OtherTok{\textless{}{-}} \FunctionTok{c}\NormalTok{(change\_points\_patterns, i }\SpecialCharTok{+} \DecValTok{2}\NormalTok{)}
\NormalTok{  \}}
\NormalTok{\}}

\CommentTok{\# Combinar pontos de mudança}
\CommentTok{\# Junta os pontos de mudança detectados por ambos os métodos}
\NormalTok{change\_points\_combined }\OtherTok{\textless{}{-}} \FunctionTok{sort}\NormalTok{(}\FunctionTok{unique}\NormalTok{(}\FunctionTok{c}\NormalTok{(change\_points\_data, change\_points\_patterns)))}

\CommentTok{\# Plotar os dados e os pontos de mudança}
\CommentTok{\# Cria um data frame com os pontos de mudança detectados pelos dois métodos e os combina}
\NormalTok{df }\OtherTok{\textless{}{-}} \FunctionTok{data.frame}\NormalTok{(}
  \AttributeTok{index =} \DecValTok{1}\SpecialCharTok{:}\FunctionTok{length}\NormalTok{(data\_stream),}
  \AttributeTok{value =}\NormalTok{ data\_stream,}
  \AttributeTok{change\_data =} \FunctionTok{ifelse}\NormalTok{(}\DecValTok{1}\SpecialCharTok{:}\FunctionTok{length}\NormalTok{(data\_stream) }\SpecialCharTok{\%in\%}\NormalTok{ change\_points\_data, }\StringTok{"Change Detected (Data)"}\NormalTok{, }\StringTok{"No Change"}\NormalTok{),}
  \AttributeTok{change\_patterns =} \FunctionTok{ifelse}\NormalTok{(}\DecValTok{1}\SpecialCharTok{:}\FunctionTok{length}\NormalTok{(data\_stream) }\SpecialCharTok{\%in\%}\NormalTok{ change\_points\_patterns, }\StringTok{"Change Detected (Patterns)"}\NormalTok{, }\StringTok{"No Change"}\NormalTok{),}
  \AttributeTok{change\_combined =} \FunctionTok{ifelse}\NormalTok{(}\DecValTok{1}\SpecialCharTok{:}\FunctionTok{length}\NormalTok{(data\_stream) }\SpecialCharTok{\%in\%}\NormalTok{ change\_points\_combined, }\StringTok{"Change Detected (Combined)"}\NormalTok{, }\StringTok{"No Change"}\NormalTok{)}
\NormalTok{)}

\CommentTok{\# Plota a série temporal com os pontos de mudança detectados, destacando os diferentes métodos de detecção}
\FunctionTok{ggplot}\NormalTok{(df, }\FunctionTok{aes}\NormalTok{(}\AttributeTok{x =}\NormalTok{ index, }\AttributeTok{y =}\NormalTok{ value)) }\SpecialCharTok{+}
  \FunctionTok{geom\_line}\NormalTok{() }\SpecialCharTok{+}
  \FunctionTok{geom\_point}\NormalTok{(}\AttributeTok{data =} \FunctionTok{subset}\NormalTok{(df, change\_combined }\SpecialCharTok{==} \StringTok{"Change Detected (Combined)"}\NormalTok{), }\FunctionTok{aes}\NormalTok{(}\AttributeTok{x =}\NormalTok{ index, }\AttributeTok{y =}\NormalTok{ value), }\AttributeTok{color =} \StringTok{"blue"}\NormalTok{, }\AttributeTok{size =} \DecValTok{2}\NormalTok{) }\SpecialCharTok{+}
  \FunctionTok{geom\_point}\NormalTok{(}\AttributeTok{data =} \FunctionTok{subset}\NormalTok{(df, change\_data }\SpecialCharTok{==} \StringTok{"Change Detected (Data)"}\NormalTok{), }\FunctionTok{aes}\NormalTok{(}\AttributeTok{x =}\NormalTok{ index, }\AttributeTok{y =}\NormalTok{ value), }\AttributeTok{color =} \StringTok{"red"}\NormalTok{, }\AttributeTok{size =} \DecValTok{2}\NormalTok{) }\SpecialCharTok{+}
  \FunctionTok{geom\_point}\NormalTok{(}\AttributeTok{data =} \FunctionTok{subset}\NormalTok{(df, change\_patterns }\SpecialCharTok{==} \StringTok{"Change Detected (Patterns)"}\NormalTok{), }\FunctionTok{aes}\NormalTok{(}\AttributeTok{x =}\NormalTok{ index, }\AttributeTok{y =}\NormalTok{ value), }\AttributeTok{color =} \StringTok{"blue"}\NormalTok{, }\AttributeTok{size =} \DecValTok{2}\NormalTok{) }\SpecialCharTok{+}
  \FunctionTok{labs}\NormalTok{(}\AttributeTok{title =} \StringTok{"Change Detected com ADWIN and OP (Combinado)"}\NormalTok{, }\AttributeTok{x =} \StringTok{"Índice"}\NormalTok{, }\AttributeTok{y =} \StringTok{"Valor"}\NormalTok{) }\SpecialCharTok{+}
  \FunctionTok{theme\_minimal}\NormalTok{()}
\end{Highlighting}
\end{Shaded}

\includegraphics{TrickyTimeSeries_files/figure-latex/unnamed-chunk-11-1.pdf}

\begin{Shaded}
\begin{Highlighting}[]
\CommentTok{\# Análise detalhada dos pontos de mudança}
\FunctionTok{print}\NormalTok{(}\StringTok{"Análise dos Pontos de Mudança (Sem OP):"}\NormalTok{)}
\end{Highlighting}
\end{Shaded}

\begin{verbatim}
## [1] "Análise dos Pontos de Mudança (Sem OP):"
\end{verbatim}

\begin{Shaded}
\begin{Highlighting}[]
\ControlFlowTok{for}\NormalTok{ (point }\ControlFlowTok{in}\NormalTok{ change\_points\_data) \{}
  \FunctionTok{cat}\NormalTok{(}\StringTok{"Ponto de mudança detectado em:"}\NormalTok{, point, }\StringTok{"}\SpecialCharTok{\textbackslash{}n}\StringTok{"}\NormalTok{)}
  \ControlFlowTok{if}\NormalTok{ (point }\SpecialCharTok{\textgreater{}} \DecValTok{1} \SpecialCharTok{\&}\NormalTok{ point }\SpecialCharTok{\textless{}} \FunctionTok{length}\NormalTok{(data\_stream)) \{}
\NormalTok{    before\_change }\OtherTok{\textless{}{-}}\NormalTok{ data\_stream[(point}\DecValTok{{-}2}\NormalTok{)}\SpecialCharTok{:}\NormalTok{(point}\DecValTok{{-}1}\NormalTok{)]}
\NormalTok{    after\_change }\OtherTok{\textless{}{-}}\NormalTok{ data\_stream[(point}\SpecialCharTok{+}\DecValTok{1}\NormalTok{)}\SpecialCharTok{:}\NormalTok{(point}\SpecialCharTok{+}\DecValTok{2}\NormalTok{)]}
    \FunctionTok{cat}\NormalTok{(}\StringTok{"Valores antes da mudança:"}\NormalTok{, before\_change, }\StringTok{"}\SpecialCharTok{\textbackslash{}n}\StringTok{"}\NormalTok{)}
    \FunctionTok{cat}\NormalTok{(}\StringTok{"Valores após a mudança:"}\NormalTok{, after\_change, }\StringTok{"}\SpecialCharTok{\textbackslash{}n\textbackslash{}n}\StringTok{"}\NormalTok{)}
\NormalTok{  \}}
\NormalTok{\}}
\end{Highlighting}
\end{Shaded}

\begin{verbatim}
## Ponto de mudança detectado em: 101 
## Valores antes da mudança: 19.7643 18.97358 
## Valores após a mudança: 25.25688 24.75331 
## 
## Ponto de mudança detectado em: 164 
## Valores antes da mudança: 23.95082 23.73984 
## Valores após a mudança: 24.58314 25.29823 
## 
## Ponto de mudança detectado em: 180 
## Valores antes da mudança: 25.31048 25.43652 
## Valores após a mudança: 23.93667 26.26319
\end{verbatim}

\begin{Shaded}
\begin{Highlighting}[]
\FunctionTok{print}\NormalTok{(}\StringTok{"Análise dos Pontos de Mudança (Com OP):"}\NormalTok{)}
\end{Highlighting}
\end{Shaded}

\begin{verbatim}
## [1] "Análise dos Pontos de Mudança (Com OP):"
\end{verbatim}

\begin{Shaded}
\begin{Highlighting}[]
\ControlFlowTok{for}\NormalTok{ (point }\ControlFlowTok{in}\NormalTok{ change\_points\_patterns) \{}
  \FunctionTok{cat}\NormalTok{(}\StringTok{"Ponto de mudança detectado em:"}\NormalTok{, point, }\StringTok{"}\SpecialCharTok{\textbackslash{}n}\StringTok{"}\NormalTok{)}
  \ControlFlowTok{if}\NormalTok{ (point }\SpecialCharTok{\textgreater{}} \DecValTok{1} \SpecialCharTok{\&}\NormalTok{ point }\SpecialCharTok{\textless{}} \FunctionTok{length}\NormalTok{(patterns)) \{}
    \FunctionTok{cat}\NormalTok{(}\StringTok{"Padrões Ordinais antes da mudança:"}\NormalTok{, patterns[(point}\DecValTok{{-}2}\NormalTok{)}\SpecialCharTok{:}\NormalTok{(point}\DecValTok{{-}1}\NormalTok{)], }\StringTok{"}\SpecialCharTok{\textbackslash{}n}\StringTok{"}\NormalTok{)}
    \FunctionTok{cat}\NormalTok{(}\StringTok{"Padrões Ordinais após a mudança:"}\NormalTok{, patterns[(point}\SpecialCharTok{+}\DecValTok{1}\NormalTok{)}\SpecialCharTok{:}\NormalTok{(point}\SpecialCharTok{+}\DecValTok{2}\NormalTok{)], }\StringTok{"}\SpecialCharTok{\textbackslash{}n\textbackslash{}n}\StringTok{"}\NormalTok{)}
\NormalTok{  \}}
\NormalTok{\}}
\end{Highlighting}
\end{Shaded}

\begin{verbatim}
## Ponto de mudança detectado em: 39 
## Padrões Ordinais antes da mudança: 5 5 
## Padrões Ordinais após a mudança: 19 7 
## 
## Ponto de mudança detectado em: 50 
## Padrões Ordinais antes da mudança: 15 11 
## Padrões Ordinais após a mudança: 5 19
\end{verbatim}

\hypertarget{plano-hxc-entropia-de-shannon-x-complexidade-de-jensen-shannon-adwin-and-op-combinado}{%
\subsection{Plano HxC (Entropia de Shannon x Complexidade de
Jensen-Shannon) ADWIN and OP
(Combinado)}\label{plano-hxc-entropia-de-shannon-x-complexidade-de-jensen-shannon-adwin-and-op-combinado}}

Para visualizar a evolução dos padrões ao longo do tempo e obter vários
pontos no plano HxC, é preciso calcular essas métricas em janelas
deslizantes ao longo da série.

O plano HxC plota a entropia de Shannon (H) no eixo x e a complexidade
de Jensen-Shannon (C) no eixo y para janelas deslizantes da série
temporal. O gráfico mostra os pontos calculados para cada janela.

Os pontos estão distribuídos ao longo de uma linha decrescente,
indicando uma relação inversa entre a entropia e a complexidade.
\emph{Conforme a entropia aumenta, a complexidade tende a diminuir.}

A entropia de Shannon mede a incerteza ou imprevisibilidade na
distribuição dos padrões ordinais. \emph{Valores mais altos de H indicam
maior desordem na série temporal.} No gráfico, H varia entre
aproximadamente 2.40 e 2.55.

A complexidade de Jensen-Shannon mede a complexidade estrutural da
distribuição de probabilidades. No gráfico, C varia entre
aproximadamente 0.000 e 0.020.

\emph{A relação inversa entre entropia e complexidade sugere que, à
medida que a incerteza (H) aumenta, a estrutura da série temporal se
torna menos complexa (C). Isso é típico em séries temporais onde uma
maior entropia pode resultar em uma distribuição mais uniforme dos
padrões, reduzindo a complexidade.}

\emph{As mudanças no plano HxC podem indicar mudanças na estrutura da
série temporal que não são evidentes apenas pela análise dos valores
médios. Mudanças significativas na distribuição dos pontos podem
corresponder a mudanças detectadas pelo ADWIN.}

O uso do plano HxC proporciona uma maneira robusta de visualizar
mudanças na estrutura de séries temporais. \emph{A combinação da
entropia de Shannon e da complexidade de Jensen-Shannon pode ajudar a
identificar mudanças sutis que não são capturadas apenas pela análise
dos valores médios. Este método pode ser útil em diversas aplicações,
incluindo a análise de séries temporais financeiras, médicas, e outras
onde a detecção de mudanças estruturais é crucial.}

\begin{Shaded}
\begin{Highlighting}[]
\CommentTok{\# Carregar bibliotecas necessárias}
\FunctionTok{library}\NormalTok{(pracma)}
\FunctionTok{library}\NormalTok{(ggplot2)}
\FunctionTok{library}\NormalTok{(ggthemes)}
\FunctionTok{library}\NormalTok{(statcomp)}

\CommentTok{\# Função para calcular padrões ordinais usando statcomp}
\NormalTok{ordinal\_patterns\_statcomp }\OtherTok{\textless{}{-}} \ControlFlowTok{function}\NormalTok{(series, emb\_dim) \{}
  \ControlFlowTok{if}\NormalTok{ (}\FunctionTok{length}\NormalTok{(series) }\SpecialCharTok{\textless{}}\NormalTok{ emb\_dim) \{}
    \FunctionTok{stop}\NormalTok{(}\StringTok{"A série temporal é muito curta para a dimensão de embedding especificada."}\NormalTok{)}
\NormalTok{  \}}
  
  \CommentTok{\# Utilizando a função from \textasciigrave{}statcomp\textasciigrave{} para calcular os padrões ordinais}
\NormalTok{  patterns }\OtherTok{\textless{}{-}} \FunctionTok{ordinal\_pattern}\NormalTok{(series, emb\_dim)}
  
  \FunctionTok{return}\NormalTok{(patterns)}
\NormalTok{\}}


\CommentTok{\# Função para calcular a entropia de Shannon}
\NormalTok{shannon\_entropy }\OtherTok{\textless{}{-}} \ControlFlowTok{function}\NormalTok{(probabilities) \{}
  \SpecialCharTok{{-}}\FunctionTok{sum}\NormalTok{(probabilities }\SpecialCharTok{*} \FunctionTok{log2}\NormalTok{(probabilities), }\AttributeTok{na.rm =} \ConstantTok{TRUE}\NormalTok{)}
\NormalTok{\}}

\CommentTok{\# Função para calcular a complexidade de Jensen{-}Shannon}
\NormalTok{js\_complexity }\OtherTok{\textless{}{-}} \ControlFlowTok{function}\NormalTok{(probabilities) \{}
\NormalTok{  p }\OtherTok{\textless{}{-}}\NormalTok{ probabilities}
\NormalTok{  q }\OtherTok{\textless{}{-}} \FunctionTok{rep}\NormalTok{(}\DecValTok{1}\SpecialCharTok{/}\FunctionTok{length}\NormalTok{(probabilities), }\FunctionTok{length}\NormalTok{(probabilities))}
\NormalTok{  m }\OtherTok{\textless{}{-}}\NormalTok{ (p }\SpecialCharTok{+}\NormalTok{ q) }\SpecialCharTok{/} \DecValTok{2}
  \FunctionTok{return}\NormalTok{((}\FunctionTok{shannon\_entropy}\NormalTok{(m) }\SpecialCharTok{{-}} \FloatTok{0.5} \SpecialCharTok{*}\NormalTok{ (}\FunctionTok{shannon\_entropy}\NormalTok{(p) }\SpecialCharTok{+} \FunctionTok{shannon\_entropy}\NormalTok{(q))) }\SpecialCharTok{/} \FunctionTok{log2}\NormalTok{(}\FunctionTok{length}\NormalTok{(probabilities)))}
\NormalTok{\}}

\CommentTok{\# Função para calcular probabilidades dos padrões ordinais}
\NormalTok{calculate\_probabilities }\OtherTok{\textless{}{-}} \ControlFlowTok{function}\NormalTok{(patterns) \{}
\NormalTok{  table\_patterns }\OtherTok{\textless{}{-}} \FunctionTok{table}\NormalTok{(patterns)}
\NormalTok{  probabilities }\OtherTok{\textless{}{-}} \FunctionTok{as.numeric}\NormalTok{(table\_patterns) }\SpecialCharTok{/} \FunctionTok{sum}\NormalTok{(table\_patterns)}
  \FunctionTok{return}\NormalTok{(probabilities)}
\NormalTok{\}}

\CommentTok{\# Função ADWIN para detectar mudanças de conceito em fluxos de dados}
\NormalTok{ADWIN }\OtherTok{\textless{}{-}} \ControlFlowTok{function}\NormalTok{(}\AttributeTok{delta =} \FloatTok{0.002}\NormalTok{) \{}
\NormalTok{  width }\OtherTok{\textless{}{-}} \DecValTok{0} \CommentTok{\# Tamanho da janela}
\NormalTok{  total }\OtherTok{\textless{}{-}} \DecValTok{0} \CommentTok{\# Soma dos valores na janela}
\NormalTok{  variance }\OtherTok{\textless{}{-}} \DecValTok{0} \CommentTok{\# Variância dos valores na janela}
\NormalTok{  window }\OtherTok{\textless{}{-}} \FunctionTok{numeric}\NormalTok{(}\DecValTok{0}\NormalTok{) }\CommentTok{\# Vetor que armazena os valores na janela}
  
\NormalTok{  update }\OtherTok{\textless{}{-}} \ControlFlowTok{function}\NormalTok{(value) \{}
\NormalTok{    width }\OtherTok{\textless{}\textless{}{-}}\NormalTok{ width }\SpecialCharTok{+} \DecValTok{1}
\NormalTok{    window }\OtherTok{\textless{}\textless{}{-}} \FunctionTok{c}\NormalTok{(window, value)}
\NormalTok{    total }\OtherTok{\textless{}\textless{}{-}}\NormalTok{ total }\SpecialCharTok{+}\NormalTok{ value}
    \ControlFlowTok{if}\NormalTok{ (width }\SpecialCharTok{\textgreater{}} \DecValTok{1}\NormalTok{) \{}
\NormalTok{      variance }\OtherTok{\textless{}\textless{}{-}} \FunctionTok{var}\NormalTok{(window, }\AttributeTok{na.rm =} \ConstantTok{TRUE}\NormalTok{)}
\NormalTok{    \}}
    \ControlFlowTok{if}\NormalTok{ (width }\SpecialCharTok{\textgreater{}} \DecValTok{1} \SpecialCharTok{\&\&} \FunctionTok{detect\_change}\NormalTok{()) \{}
      \FunctionTok{return}\NormalTok{(}\ConstantTok{TRUE}\NormalTok{)}
\NormalTok{    \}}
    \FunctionTok{return}\NormalTok{(}\ConstantTok{FALSE}\NormalTok{)}
\NormalTok{  \}}
  
\NormalTok{  detect\_change }\OtherTok{\textless{}{-}} \ControlFlowTok{function}\NormalTok{() \{}
\NormalTok{    mean\_val }\OtherTok{\textless{}{-}} \FunctionTok{mean}\NormalTok{(window, }\AttributeTok{na.rm =} \ConstantTok{TRUE}\NormalTok{)}
    \ControlFlowTok{for}\NormalTok{ (n }\ControlFlowTok{in} \DecValTok{1}\SpecialCharTok{:}\NormalTok{(width }\SpecialCharTok{{-}} \DecValTok{1}\NormalTok{)) \{}
\NormalTok{      mean0 }\OtherTok{\textless{}{-}} \FunctionTok{mean}\NormalTok{(window[}\DecValTok{1}\SpecialCharTok{:}\NormalTok{n], }\AttributeTok{na.rm =} \ConstantTok{TRUE}\NormalTok{)}
\NormalTok{      mean1 }\OtherTok{\textless{}{-}} \FunctionTok{mean}\NormalTok{(window[(n }\SpecialCharTok{+} \DecValTok{1}\NormalTok{)}\SpecialCharTok{:}\NormalTok{width], }\AttributeTok{na.rm =} \ConstantTok{TRUE}\NormalTok{)}
      \ControlFlowTok{if}\NormalTok{ (}\FunctionTok{is.na}\NormalTok{(mean0) }\SpecialCharTok{||} \FunctionTok{is.na}\NormalTok{(mean1)) }\ControlFlowTok{next}
      \ControlFlowTok{if}\NormalTok{ (}\FunctionTok{abs}\NormalTok{(mean0 }\SpecialCharTok{{-}}\NormalTok{ mean1) }\SpecialCharTok{\textgreater{}} \FunctionTok{sqrt}\NormalTok{((variance }\SpecialCharTok{/}\NormalTok{ n) }\SpecialCharTok{+}\NormalTok{ (variance }\SpecialCharTok{/}\NormalTok{ (width }\SpecialCharTok{{-}}\NormalTok{ n))) }\SpecialCharTok{*} \FunctionTok{qnorm}\NormalTok{(}\DecValTok{1} \SpecialCharTok{{-}}\NormalTok{ delta)) \{}
\NormalTok{        window }\OtherTok{\textless{}\textless{}{-}}\NormalTok{ window[(n }\SpecialCharTok{+} \DecValTok{1}\NormalTok{)}\SpecialCharTok{:}\NormalTok{width]}
\NormalTok{        width }\OtherTok{\textless{}\textless{}{-}} \FunctionTok{length}\NormalTok{(window)}
\NormalTok{        total }\OtherTok{\textless{}\textless{}{-}} \FunctionTok{sum}\NormalTok{(window, }\AttributeTok{na.rm =} \ConstantTok{TRUE}\NormalTok{)}
\NormalTok{        variance }\OtherTok{\textless{}\textless{}{-}} \FunctionTok{var}\NormalTok{(window, }\AttributeTok{na.rm =} \ConstantTok{TRUE}\NormalTok{)}
        \FunctionTok{return}\NormalTok{(}\ConstantTok{TRUE}\NormalTok{)}
\NormalTok{      \}}
\NormalTok{    \}}
    \FunctionTok{return}\NormalTok{(}\ConstantTok{FALSE}\NormalTok{)}
\NormalTok{  \}}
  
  \FunctionTok{list}\NormalTok{(}\AttributeTok{update =}\NormalTok{ update)}
\NormalTok{\}}

\CommentTok{\# Exemplo }
\FunctionTok{set.seed}\NormalTok{(}\DecValTok{123}\NormalTok{) }\CommentTok{\# Define a semente para reprodutibilidade}
\NormalTok{data\_stream }\OtherTok{\textless{}{-}} \FunctionTok{c}\NormalTok{(}\FunctionTok{rnorm}\NormalTok{(}\DecValTok{100}\NormalTok{, }\AttributeTok{mean =} \DecValTok{20}\NormalTok{), }\FunctionTok{rnorm}\NormalTok{(}\DecValTok{100}\NormalTok{, }\AttributeTok{mean =} \DecValTok{25}\NormalTok{)) }\CommentTok{\# Gera dados de exemplo com duas médias diferentes}

\CommentTok{\# Inicializa detectores ADWIN}
\NormalTok{adwin\_data }\OtherTok{\textless{}{-}} \FunctionTok{ADWIN}\NormalTok{(}\AttributeTok{delta =} \FloatTok{0.002}\NormalTok{)}
\NormalTok{adwin\_patterns }\OtherTok{\textless{}{-}} \FunctionTok{ADWIN}\NormalTok{(}\AttributeTok{delta =} \FloatTok{0.002}\NormalTok{)}

\CommentTok{\# Calcular padrões ordinais}
\NormalTok{patterns }\OtherTok{\textless{}{-}} \FunctionTok{ordinal\_patterns}\NormalTok{(data\_stream, }\AttributeTok{emb\_dim =} \DecValTok{3}\NormalTok{)}

\CommentTok{\# Vetores para armazenar os pontos de detecção de mudança}
\NormalTok{change\_points\_data }\OtherTok{\textless{}{-}} \FunctionTok{numeric}\NormalTok{(}\DecValTok{0}\NormalTok{)}
\NormalTok{change\_points\_patterns }\OtherTok{\textless{}{-}} \FunctionTok{numeric}\NormalTok{(}\DecValTok{0}\NormalTok{)}

\CommentTok{\# Processa o fluxo de dados}
\ControlFlowTok{for}\NormalTok{ (i }\ControlFlowTok{in} \DecValTok{1}\SpecialCharTok{:}\FunctionTok{length}\NormalTok{(data\_stream)) \{}
  \ControlFlowTok{if}\NormalTok{ (adwin\_data}\SpecialCharTok{$}\FunctionTok{update}\NormalTok{(data\_stream[i])) \{}
\NormalTok{    change\_points\_data }\OtherTok{\textless{}{-}} \FunctionTok{c}\NormalTok{(change\_points\_data, i)}
\NormalTok{  \}}
  \ControlFlowTok{if}\NormalTok{ (i }\SpecialCharTok{\textless{}=} \FunctionTok{length}\NormalTok{(patterns) }\SpecialCharTok{\&\&}\NormalTok{ adwin\_patterns}\SpecialCharTok{$}\FunctionTok{update}\NormalTok{(patterns[i])) \{}
\NormalTok{    change\_points\_patterns }\OtherTok{\textless{}{-}} \FunctionTok{c}\NormalTok{(change\_points\_patterns, i }\SpecialCharTok{+} \DecValTok{2}\NormalTok{)}
\NormalTok{  \}}
\NormalTok{\}}

\CommentTok{\# Combinar pontos de mudança}
\NormalTok{change\_points\_combined }\OtherTok{\textless{}{-}} \FunctionTok{sort}\NormalTok{(}\FunctionTok{unique}\NormalTok{(}\FunctionTok{c}\NormalTok{(change\_points\_data, change\_points\_patterns)))}

\CommentTok{\# Calcular H e C em janelas deslizantes}
\NormalTok{window\_size }\OtherTok{\textless{}{-}} \DecValTok{50}
\NormalTok{H\_values }\OtherTok{\textless{}{-}} \FunctionTok{numeric}\NormalTok{()}
\NormalTok{C\_values }\OtherTok{\textless{}{-}} \FunctionTok{numeric}\NormalTok{()}
\NormalTok{indices }\OtherTok{\textless{}{-}} \FunctionTok{seq}\NormalTok{(}\DecValTok{1}\NormalTok{, }\FunctionTok{length}\NormalTok{(patterns) }\SpecialCharTok{{-}}\NormalTok{ window\_size }\SpecialCharTok{+} \DecValTok{1}\NormalTok{, }\AttributeTok{by =} \DecValTok{1}\NormalTok{)}

\ControlFlowTok{for}\NormalTok{ (i }\ControlFlowTok{in}\NormalTok{ indices) \{}
\NormalTok{  window\_patterns }\OtherTok{\textless{}{-}}\NormalTok{ patterns[i}\SpecialCharTok{:}\NormalTok{(i }\SpecialCharTok{+}\NormalTok{ window\_size }\SpecialCharTok{{-}} \DecValTok{1}\NormalTok{)]}
\NormalTok{  probabilities }\OtherTok{\textless{}{-}} \FunctionTok{calculate\_probabilities}\NormalTok{(window\_patterns)}
\NormalTok{  H\_values }\OtherTok{\textless{}{-}} \FunctionTok{c}\NormalTok{(H\_values, }\FunctionTok{shannon\_entropy}\NormalTok{(probabilities))}
\NormalTok{  C\_values }\OtherTok{\textless{}{-}} \FunctionTok{c}\NormalTok{(C\_values, }\FunctionTok{js\_complexity}\NormalTok{(probabilities))}
\NormalTok{\}}

\CommentTok{\# Criar um data frame para os valores H e C}
\NormalTok{df\_HxC }\OtherTok{\textless{}{-}} \FunctionTok{data.frame}\NormalTok{(}\AttributeTok{H =}\NormalTok{ H\_values, }\AttributeTok{C =}\NormalTok{ C\_values)}

\CommentTok{\# Plotar os resultados no plano HxC}
\FunctionTok{ggplot}\NormalTok{(df\_HxC, }\FunctionTok{aes}\NormalTok{(}\AttributeTok{x =}\NormalTok{ H, }\AttributeTok{y =}\NormalTok{ C)) }\SpecialCharTok{+}
  \FunctionTok{geom\_point}\NormalTok{(}\AttributeTok{color =} \StringTok{"blue"}\NormalTok{, }\AttributeTok{size =} \DecValTok{3}\NormalTok{) }\SpecialCharTok{+}
  \FunctionTok{labs}\NormalTok{(}\AttributeTok{title =} \StringTok{"Plano HxC Change Detected with ADWIN and Ordinal Patterns (Combined)"}\NormalTok{,}
       \AttributeTok{x =} \StringTok{"Entropia de Shannon (H)"}\NormalTok{,}
       \AttributeTok{y =} \StringTok{"Complexidade de Jensen{-}Shannon (C)"}\NormalTok{) }\SpecialCharTok{+}
  \FunctionTok{theme\_minimal}\NormalTok{() }\SpecialCharTok{+}
  \FunctionTok{geom\_smooth}\NormalTok{(}\AttributeTok{method =} \StringTok{"lm"}\NormalTok{, }\AttributeTok{color =} \StringTok{"red"}\NormalTok{) }\SpecialCharTok{+} \CommentTok{\# Adiciona uma linha de tendência}
  \FunctionTok{theme\_bw}\NormalTok{() }\CommentTok{\# Utiliza um tema com fundo branco para maior clareza}
\end{Highlighting}
\end{Shaded}

\begin{verbatim}
## `geom_smooth()` using formula = 'y ~ x'
\end{verbatim}

\includegraphics{TrickyTimeSeries_files/figure-latex/unnamed-chunk-12-1.pdf}

\begin{Shaded}
\begin{Highlighting}[]
\CommentTok{\# Plotar os dados e os pontos de mudança}
\NormalTok{df }\OtherTok{\textless{}{-}} \FunctionTok{data.frame}\NormalTok{(}
  \AttributeTok{index =} \DecValTok{1}\SpecialCharTok{:}\FunctionTok{length}\NormalTok{(data\_stream),}
  \AttributeTok{value =}\NormalTok{ data\_stream,}
  \AttributeTok{change\_data =} \FunctionTok{ifelse}\NormalTok{(}\DecValTok{1}\SpecialCharTok{:}\FunctionTok{length}\NormalTok{(data\_stream) }\SpecialCharTok{\%in\%}\NormalTok{ change\_points\_data, }\StringTok{"Change Detected (Data)"}\NormalTok{, }\StringTok{"No Change"}\NormalTok{),}
  \AttributeTok{change\_patterns =} \FunctionTok{ifelse}\NormalTok{(}\DecValTok{1}\SpecialCharTok{:}\FunctionTok{length}\NormalTok{(data\_stream) }\SpecialCharTok{\%in\%}\NormalTok{ change\_points\_patterns, }\StringTok{"Change Detected (Patterns)"}\NormalTok{, }\StringTok{"No Change"}\NormalTok{),}
  \AttributeTok{change\_combined =} \FunctionTok{ifelse}\NormalTok{(}\DecValTok{1}\SpecialCharTok{:}\FunctionTok{length}\NormalTok{(data\_stream) }\SpecialCharTok{\%in\%}\NormalTok{ change\_points\_combined, }\StringTok{"Change Detected (Combined)"}\NormalTok{, }\StringTok{"No Change"}\NormalTok{)}
\NormalTok{)}

\FunctionTok{ggplot}\NormalTok{(df, }\FunctionTok{aes}\NormalTok{(}\AttributeTok{x =}\NormalTok{ index, }\AttributeTok{y =}\NormalTok{ value)) }\SpecialCharTok{+}
  \FunctionTok{geom\_line}\NormalTok{() }\SpecialCharTok{+}
  \FunctionTok{geom\_point}\NormalTok{(}\AttributeTok{data =} \FunctionTok{subset}\NormalTok{(df, change\_combined }\SpecialCharTok{==} \StringTok{"Change Detected (Combined)"}\NormalTok{), }\FunctionTok{aes}\NormalTok{(}\AttributeTok{x =}\NormalTok{ index, }\AttributeTok{y =}\NormalTok{ value), }\AttributeTok{color =} \StringTok{"blue"}\NormalTok{, }\AttributeTok{size =} \DecValTok{2}\NormalTok{) }\SpecialCharTok{+}
  \FunctionTok{geom\_point}\NormalTok{(}\AttributeTok{data =} \FunctionTok{subset}\NormalTok{(df, change\_data }\SpecialCharTok{==} \StringTok{"Change Detected (Data)"}\NormalTok{), }\FunctionTok{aes}\NormalTok{(}\AttributeTok{x =}\NormalTok{ index, }\AttributeTok{y =}\NormalTok{ value), }\AttributeTok{color =} \StringTok{"red"}\NormalTok{, }\AttributeTok{size =} \DecValTok{2}\NormalTok{) }\SpecialCharTok{+}
  \FunctionTok{geom\_point}\NormalTok{(}\AttributeTok{data =} \FunctionTok{subset}\NormalTok{(df, change\_patterns }\SpecialCharTok{==} \StringTok{"Change Detected (Patterns)"}\NormalTok{), }\FunctionTok{aes}\NormalTok{(}\AttributeTok{x =}\NormalTok{ index, }\AttributeTok{y =}\NormalTok{ value), }\AttributeTok{color =} \StringTok{"blue"}\NormalTok{, }\AttributeTok{size =} \DecValTok{2}\NormalTok{) }\SpecialCharTok{+}
  \FunctionTok{labs}\NormalTok{(}\AttributeTok{title =} \StringTok{"Change Detected with ADWIN and Ordinal Patterns (Combined)"}\NormalTok{, }\AttributeTok{x =} \StringTok{"Índice"}\NormalTok{, }\AttributeTok{y =} \StringTok{"Valor"}\NormalTok{) }\SpecialCharTok{+}
  \FunctionTok{theme\_minimal}\NormalTok{()}
\end{Highlighting}
\end{Shaded}

\includegraphics{TrickyTimeSeries_files/figure-latex/unnamed-chunk-12-2.pdf}

\begin{Shaded}
\begin{Highlighting}[]
\CommentTok{\# Análise detalhada dos pontos de mudança}
\FunctionTok{print}\NormalTok{(}\StringTok{"Análise dos Pontos de Mudança (Sem OP):"}\NormalTok{)}
\end{Highlighting}
\end{Shaded}

\begin{verbatim}
## [1] "Análise dos Pontos de Mudança (Sem OP):"
\end{verbatim}

\begin{Shaded}
\begin{Highlighting}[]
\ControlFlowTok{for}\NormalTok{ (point }\ControlFlowTok{in}\NormalTok{ change\_points\_data) \{}
  \FunctionTok{cat}\NormalTok{(}\StringTok{"Ponto de mudança detectado em:"}\NormalTok{, point, }\StringTok{"}\SpecialCharTok{\textbackslash{}n}\StringTok{"}\NormalTok{)}
  \ControlFlowTok{if}\NormalTok{ (point }\SpecialCharTok{\textgreater{}} \DecValTok{1} \SpecialCharTok{\&}\NormalTok{ point }\SpecialCharTok{\textless{}} \FunctionTok{length}\NormalTok{(data\_stream)) \{}
\NormalTok{    before\_change }\OtherTok{\textless{}{-}}\NormalTok{ data\_stream[(point}\DecValTok{{-}2}\NormalTok{)}\SpecialCharTok{:}\NormalTok{(point}\DecValTok{{-}1}\NormalTok{)]}
\NormalTok{    after\_change }\OtherTok{\textless{}{-}}\NormalTok{ data\_stream[(point}\SpecialCharTok{+}\DecValTok{1}\NormalTok{)}\SpecialCharTok{:}\NormalTok{(point}\SpecialCharTok{+}\DecValTok{2}\NormalTok{)]}
    \FunctionTok{cat}\NormalTok{(}\StringTok{"Valores antes da mudança:"}\NormalTok{, before\_change, }\StringTok{"}\SpecialCharTok{\textbackslash{}n}\StringTok{"}\NormalTok{)}
    \FunctionTok{cat}\NormalTok{(}\StringTok{"Valores após a mudança:"}\NormalTok{, after\_change, }\StringTok{"}\SpecialCharTok{\textbackslash{}n\textbackslash{}n}\StringTok{"}\NormalTok{)}
\NormalTok{  \}}
\NormalTok{\}}
\end{Highlighting}
\end{Shaded}

\begin{verbatim}
## Ponto de mudança detectado em: 101 
## Valores antes da mudança: 19.7643 18.97358 
## Valores após a mudança: 25.25688 24.75331 
## 
## Ponto de mudança detectado em: 164 
## Valores antes da mudança: 23.95082 23.73984 
## Valores após a mudança: 24.58314 25.29823 
## 
## Ponto de mudança detectado em: 180 
## Valores antes da mudança: 25.31048 25.43652 
## Valores após a mudança: 23.93667 26.26319
\end{verbatim}

\begin{Shaded}
\begin{Highlighting}[]
\FunctionTok{print}\NormalTok{(}\StringTok{"Análise dos Pontos de Mudança (Com OP):"}\NormalTok{)}
\end{Highlighting}
\end{Shaded}

\begin{verbatim}
## [1] "Análise dos Pontos de Mudança (Com OP):"
\end{verbatim}

\begin{Shaded}
\begin{Highlighting}[]
\ControlFlowTok{for}\NormalTok{ (point }\ControlFlowTok{in}\NormalTok{ change\_points\_patterns) \{}
  \FunctionTok{cat}\NormalTok{(}\StringTok{"Ponto de mudança detectado em:"}\NormalTok{, point, }\StringTok{"}\SpecialCharTok{\textbackslash{}n}\StringTok{"}\NormalTok{)}
  \ControlFlowTok{if}\NormalTok{ (point }\SpecialCharTok{\textgreater{}} \DecValTok{1} \SpecialCharTok{\&}\NormalTok{ point }\SpecialCharTok{\textless{}} \FunctionTok{length}\NormalTok{(patterns)) \{}
    \FunctionTok{cat}\NormalTok{(}\StringTok{"Padrões Ordinais antes da mudança:"}\NormalTok{, patterns[(point}\DecValTok{{-}2}\NormalTok{)}\SpecialCharTok{:}\NormalTok{(point}\DecValTok{{-}1}\NormalTok{)], }\StringTok{"}\SpecialCharTok{\textbackslash{}n}\StringTok{"}\NormalTok{)}
    \FunctionTok{cat}\NormalTok{(}\StringTok{"Padrões Ordinais após a mudança:"}\NormalTok{, patterns[(point}\SpecialCharTok{+}\DecValTok{1}\NormalTok{)}\SpecialCharTok{:}\NormalTok{(point}\SpecialCharTok{+}\DecValTok{2}\NormalTok{)], }\StringTok{"}\SpecialCharTok{\textbackslash{}n\textbackslash{}n}\StringTok{"}\NormalTok{)}
\NormalTok{  \}}
\NormalTok{\}}
\end{Highlighting}
\end{Shaded}

\begin{verbatim}
## Ponto de mudança detectado em: 39 
## Padrões Ordinais antes da mudança: 5 5 
## Padrões Ordinais após a mudança: 19 7 
## 
## Ponto de mudança detectado em: 50 
## Padrões Ordinais antes da mudança: 15 11 
## Padrões Ordinais após a mudança: 5 19
\end{verbatim}

\hypertarget{anuxe1lise-dos-truxeas-cenuxe1rios-no-plano-hxc}{%
\subsubsection{Análise dos Três Cenários no Plano
HxC}\label{anuxe1lise-dos-truxeas-cenuxe1rios-no-plano-hxc}}

\hypertarget{cenuxe1rio-1-plano-hxc-with-adwin}{%
\paragraph{Cenário 1: Plano HxC with
ADWIN}\label{cenuxe1rio-1-plano-hxc-with-adwin}}

\begin{figure}
\centering
\includegraphics{sandbox:/mnt/data/file-6SYbMzoCUeazf4MdhIqinSnD}
\caption{Plano HxC com ADWIN}
\end{figure}

Neste cenário, a entropia de Shannon (H) e a complexidade de
Jensen-Shannon (C) foram calculadas para janelas deslizantes dos dados
brutos e o ADWIN foi utilizado para detectar mudanças.

\textbf{Observações:}

\begin{itemize}
\item
  A relação entre H e C mostra uma tendência crescente, o que indica
  que, à medida que a entropia aumenta, a complexidade também tende a
  aumentar.
\item
  A linha de tendência (em vermelho) confirma a correlação positiva
  entre H e C, mas há uma dispersão significativa em torno da linha,
  sugerindo variação nos dados.
\end{itemize}

\textbf{Interpretação:}

\begin{itemize}
\item
  A tendência crescente sugere que, nos dados brutos, aumentos na
  entropia estão associados a aumentos na complexidade.
\item
  A dispersão pode indicar a presença de variações locais nos dados,
  possivelmente devido a mudanças nos padrões dos dados ao longo do
  tempo.
\end{itemize}

\hypertarget{cenuxe1rio-2-plano-hxc-with-adwin-and-padruxf5es-ordinais}{%
\paragraph{Cenário 2: Plano HxC with ADWIN and Padrões
Ordinais}\label{cenuxe1rio-2-plano-hxc-with-adwin-and-padruxf5es-ordinais}}

\begin{figure}
\centering
\includegraphics{sandbox:/mnt/data/file-XWggTK3NBqVsshxl1PCW4LAV}
\caption{Plano HxC with ADWIN and Padrões Ordinais}
\end{figure}

Neste cenário, além do ADWIN, foram utilizados padrões ordinais para
calcular H e C.

\textbf{Observações:}

\begin{itemize}
\item
  A relação entre H e C mostra uma tendência decrescente, sugerindo uma
  relação inversa entre a entropia e a complexidade.
\item
  A linha de tendência (em vermelho) confirma a correlação negativa
  entre H e C.
\end{itemize}

\textbf{Interpretação:}

\begin{itemize}
\item
  A tendência decrescente indica que, quando os padrões ordinais são
  utilizados, há uma redução na complexidade à medida que a entropia
  aumenta.
\item
  Isso pode ser um sinal de que, em termos de padrões ordinais, a
  complexidade dos padrões diminui com o aumento da entropia,
  possivelmente indicando que os padrões se tornam mais previsíveis.
\end{itemize}

\hypertarget{cenuxe1rio-3-plano-hxc-change-detected-with-adwin-and-padruxf5es-ordinais-combinado}{%
\paragraph{Cenário 3: Plano HxC Change Detected with ADWIN and Padrões
Ordinais
(Combinado)}\label{cenuxe1rio-3-plano-hxc-change-detected-with-adwin-and-padruxf5es-ordinais-combinado}}

\begin{figure}
\centering
\includegraphics{sandbox:/mnt/data/file-k6NsCUnXYiMcQyz9iRH7W3Jv}
\caption{Plano HxC Change Detected with ADWIN and Padrões Ordinais
(Combinado)}
\end{figure}

Neste cenário, os pontos de mudança detectados pelo ADWIN e pelos
padrões ordinais foram combinados para calcular H e C.

\textbf{Observações:}

\begin{itemize}
\item
  A relação entre H e C mantém a tendência decrescente vista no Cenário
  2, mas com uma dispersão menor em torno da linha de tendência.
\item
  A linha de tendência (em vermelho) confirma a forte correlação
  negativa entre H e C.
\end{itemize}

\textbf{Interpretação:}

\begin{itemize}
\item
  A combinação dos métodos ADWIN e padrões ordinais proporciona uma
  visão mais clara e consistente da relação inversa entre a entropia e a
  complexidade.
\item
  A menor dispersão sugere que a combinação dos métodos melhora a
  robustez da detecção de mudanças, proporcionando uma detecção mais
  precisa e consistente das variações nos dados.
\end{itemize}

\hypertarget{conclusuxe3o-geral}{%
\subsubsection{Conclusão Geral}\label{conclusuxe3o-geral}}

Os três cenários analisados no plano HxC apresentam diferentes
comportamentos na relação entre a entropia de Shannon e a complexidade
de Jensen-Shannon:

\begin{enumerate}
\def\labelenumi{\arabic{enumi}.}
\item
  \textbf{Cenário 1:} A relação entre H e C nos dados brutos mostra uma
  correlação positiva, mas com dispersão significativa, sugerindo
  variabilidade local nos dados.
\item
  \textbf{Cenário 2:} A utilização de padrões ordinais revela uma
  correlação negativa entre H e C, indicando uma possível simplificação
  dos padrões à medida que a entropia aumenta.
\item
  \textbf{Cenário 3:} A combinação de ADWIN e padrões ordinais resulta
  em uma detecção de mudanças mais precisa, confirmando a correlação
  negativa com menor dispersão, o que sugere maior consistência na
  detecção de variações nos dados.
\end{enumerate}

Esses cenários ilustram como diferentes abordagens podem revelar
distintas características dos dados, fornecendo insights valiosos sobre
a natureza das mudanças nos fluxos de dados.

\textbf{A abordagem combinada de ADWIN com padrões ordinais é
significativa na medida em que pode oferecer melhorias na sensibilidade
a mudanças estruturais que não são facilmente detectáveis apenas por
mudanças na média ou variância. Isso é particularmente útil em
aplicações onde a dinâmica subjacente dos dados é complexa e envolve
mudanças no comportamento além de simples mudanças de amplitude.}

\begin{center}\rule{0.5\linewidth}{0.5pt}\end{center}

\hypertarget{start-serie-prof.-alejandro}{%
\paragraph{Start Serie
Prof.~Alejandro}\label{start-serie-prof.-alejandro}}

We will build tricky time series. They will be comprised to two parts:

\begin{enumerate}
\item The first half will consist of batches of observations from independent identically distributed random variables (white noise) with zero mean and unitary standard deviation.
We will use the following random variables:
  \begin{itemize}
  \item Standard Gaussian law: $Y_1\sim\text{N}(0,1)$.
  \item Scaled Uniform distribution $Y_2=\sqrt{12}(U-1/2)$, with $U\sim\text{U}(0,1)$.
  \item Noncentral Exponential distribution: $Y_3=Y-1$, with $Y\sim\text{E}(1)$.
  \end{itemize}
\item The second half will consist of observations of $1/f^k$ noise, with $k>0$ (coloured noise). These observations are standardized (zero mean, unitary variance).
\end{enumerate}

\hypertarget{detecuxe7uxe3o-de-mudanuxe7a-with-adwin-trickytimeseries}{%
\subsubsection{Detecção de Mudança with ADWIN
(TrickyTimeSeries)}\label{detecuxe7uxe3o-de-mudanuxe7a-with-adwin-trickytimeseries}}

\begin{Shaded}
\begin{Highlighting}[]
\CommentTok{\# From https://stackoverflow.com/questions/8697567/how{-}to{-}simulate{-}pink{-}noise{-}in{-}r}
\FunctionTok{library}\NormalTok{(ggplot2)}
\FunctionTok{library}\NormalTok{(ggthemes)}

\CommentTok{\# A função TK95 envolve cálculos de Fourier}
\CommentTok{\# Função para gerar ruído colorido}
\NormalTok{TK95 }\OtherTok{\textless{}{-}} \ControlFlowTok{function}\NormalTok{(N, }\AttributeTok{alpha =} \DecValTok{1}\NormalTok{)\{ }
    \CommentTok{\# Cria uma sequência de frequências de Fourier}
\NormalTok{    f }\OtherTok{\textless{}{-}} \FunctionTok{seq}\NormalTok{(}\AttributeTok{from=}\DecValTok{0}\NormalTok{, }\AttributeTok{to=}\NormalTok{pi, }\AttributeTok{length.out=}\NormalTok{(N}\SpecialCharTok{/}\DecValTok{2}\SpecialCharTok{+}\DecValTok{1}\NormalTok{))[}\SpecialCharTok{{-}}\FunctionTok{c}\NormalTok{(}\DecValTok{1}\NormalTok{,(N}\SpecialCharTok{/}\DecValTok{2}\SpecialCharTok{+}\DecValTok{1}\NormalTok{))] }\CommentTok{\# Frequências de Fourier}
\NormalTok{    f\_ }\OtherTok{\textless{}{-}} \DecValTok{1} \SpecialCharTok{/}\NormalTok{ f}\SpecialCharTok{\^{}}\NormalTok{alpha }\CommentTok{\# Lei de potência}
\NormalTok{    RW }\OtherTok{\textless{}{-}} \FunctionTok{sqrt}\NormalTok{(}\FloatTok{0.5}\SpecialCharTok{*}\NormalTok{f\_) }\SpecialCharTok{*} \FunctionTok{rnorm}\NormalTok{(N}\SpecialCharTok{/}\DecValTok{2{-}1}\NormalTok{) }\CommentTok{\# Parte real}
\NormalTok{    IW }\OtherTok{\textless{}{-}} \FunctionTok{sqrt}\NormalTok{(}\FloatTok{0.5}\SpecialCharTok{*}\NormalTok{f\_) }\SpecialCharTok{*} \FunctionTok{rnorm}\NormalTok{(N}\SpecialCharTok{/}\DecValTok{2{-}1}\NormalTok{) }\CommentTok{\# Parte imaginária}
\NormalTok{    fR }\OtherTok{\textless{}{-}} \FunctionTok{complex}\NormalTok{(}\AttributeTok{real =} \FunctionTok{c}\NormalTok{(}\FunctionTok{rnorm}\NormalTok{(}\DecValTok{1}\NormalTok{), RW, }\FunctionTok{rnorm}\NormalTok{(}\DecValTok{1}\NormalTok{), RW[(N}\SpecialCharTok{/}\DecValTok{2{-}1}\NormalTok{)}\SpecialCharTok{:}\DecValTok{1}\NormalTok{]), }
                  \AttributeTok{imaginary =} \FunctionTok{c}\NormalTok{(}\DecValTok{0}\NormalTok{, IW, }\DecValTok{0}\NormalTok{, }\SpecialCharTok{{-}}\NormalTok{IW[(N}\SpecialCharTok{/}\DecValTok{2{-}1}\NormalTok{)}\SpecialCharTok{:}\DecValTok{1}\NormalTok{]), }\AttributeTok{length.out=}\NormalTok{N)}
\NormalTok{    reihe }\OtherTok{\textless{}{-}} \FunctionTok{fft}\NormalTok{(fR, }\AttributeTok{inverse=}\ConstantTok{TRUE}\NormalTok{) }\CommentTok{\# Retornar ao domínio do tempo com transformada inversa de Fourier}
    \FunctionTok{return}\NormalTok{(}\FunctionTok{Re}\NormalTok{(reihe)) }\CommentTok{\# Retorna apenas a parte real}
\NormalTok{\}}

\CommentTok{\# A função TrickyTimeSeries gera 300 pontos de ruído colorido, além dos pontos de ruído branco.}
\CommentTok{\# Função para criar a série temporal}
\NormalTok{TrickyTimeSeries }\OtherTok{\textless{}{-}} \ControlFlowTok{function}\NormalTok{(n, k)\{}
  \CommentTok{\# Gera três tipos de ruído branco}
\NormalTok{  y1 }\OtherTok{\textless{}{-}} \FunctionTok{rnorm}\NormalTok{(n) }\CommentTok{\# Ruído branco gaussiano}
\NormalTok{  y2 }\OtherTok{\textless{}{-}} \FunctionTok{sqrt}\NormalTok{(}\DecValTok{12}\NormalTok{)}\SpecialCharTok{*}\FunctionTok{runif}\NormalTok{(n, }\AttributeTok{min=}\SpecialCharTok{{-}}\DecValTok{1}\SpecialCharTok{/}\DecValTok{2}\NormalTok{, }\AttributeTok{max=}\DecValTok{1}\SpecialCharTok{/}\DecValTok{2}\NormalTok{) }\CommentTok{\# Ruído branco uniforme escalado}
\NormalTok{  y3 }\OtherTok{\textless{}{-}} \FunctionTok{rexp}\NormalTok{(n)}\SpecialCharTok{{-}}\DecValTok{1} \CommentTok{\# Ruído branco exponencial não central}
  
  \CommentTok{\# Gera ruído colorido 1/f\^{}k}
\NormalTok{  fk }\OtherTok{\textless{}{-}} \FunctionTok{TK95}\NormalTok{(}\AttributeTok{N=}\DecValTok{3}\SpecialCharTok{*}\NormalTok{n, }\AttributeTok{alpha=}\NormalTok{k)}
\NormalTok{  fk }\OtherTok{\textless{}{-}}\NormalTok{ (fk}\SpecialCharTok{{-}}\FunctionTok{mean}\NormalTok{(fk))}\SpecialCharTok{/}\FunctionTok{sd}\NormalTok{(fk) }\CommentTok{\# Padroniza para ter média zero e desvio padrão unitário}
  
\NormalTok{  output }\OtherTok{\textless{}{-}} \FunctionTok{c}\NormalTok{(y1, y2, y3, fk) }\CommentTok{\# Combina as séries geradas}
  \FunctionTok{return}\NormalTok{(output)}
\NormalTok{\}}

\CommentTok{\# A função ADWIN é chamada repetidamente para cada ponto da série temporal. A lógica de detecção de mudanças envolve calcular a média e a variância de sub{-}janelas.}
\CommentTok{\# Função ADWIN para detectar mudanças de conceito}
\NormalTok{ADWIN }\OtherTok{\textless{}{-}} \ControlFlowTok{function}\NormalTok{(}\AttributeTok{delta =} \FloatTok{0.002}\NormalTok{) \{}
  \CommentTok{\# Inicializa as variáveis}
\NormalTok{  width }\OtherTok{\textless{}{-}} \DecValTok{0} \CommentTok{\# Tamanho da janela}
\NormalTok{  total }\OtherTok{\textless{}{-}} \DecValTok{0} \CommentTok{\# Soma dos valores na janela}
\NormalTok{  variance }\OtherTok{\textless{}{-}} \DecValTok{0} \CommentTok{\# Variância dos valores na janela}
\NormalTok{  window }\OtherTok{\textless{}{-}} \FunctionTok{numeric}\NormalTok{(}\DecValTok{0}\NormalTok{) }\CommentTok{\# Vetor que armazena os valores na janela}
  
  \CommentTok{\# Função para atualizar o ADWIN com um novo valor}
\NormalTok{  update }\OtherTok{\textless{}{-}} \ControlFlowTok{function}\NormalTok{(value) \{}
\NormalTok{    width }\OtherTok{\textless{}\textless{}{-}}\NormalTok{ width }\SpecialCharTok{+} \DecValTok{1} \CommentTok{\# Incrementa o tamanho da janela}
\NormalTok{    window }\OtherTok{\textless{}\textless{}{-}} \FunctionTok{c}\NormalTok{(window, value) }\CommentTok{\# Adiciona o novo valor à janela}
\NormalTok{    total }\OtherTok{\textless{}\textless{}{-}}\NormalTok{ total }\SpecialCharTok{+}\NormalTok{ value }\CommentTok{\# Atualiza a soma total}
    \ControlFlowTok{if}\NormalTok{ (width }\SpecialCharTok{\textgreater{}} \DecValTok{1}\NormalTok{) \{}
\NormalTok{      variance }\OtherTok{\textless{}\textless{}{-}} \FunctionTok{var}\NormalTok{(window) }\CommentTok{\# Calcula a variância se a janela tiver mais de um valor}
\NormalTok{    \}}
    \CommentTok{\# Checa por concept drift}
    \ControlFlowTok{if}\NormalTok{ (width }\SpecialCharTok{\textgreater{}} \DecValTok{1} \SpecialCharTok{\&\&} \FunctionTok{detect\_change}\NormalTok{()) \{}
      \FunctionTok{return}\NormalTok{(}\ConstantTok{TRUE}\NormalTok{) }\CommentTok{\# Retorna TRUE se uma mudança for detectada}
\NormalTok{    \}}
    \FunctionTok{return}\NormalTok{(}\ConstantTok{FALSE}\NormalTok{) }\CommentTok{\# Retorna FALSE se nenhuma mudança for detectada}
\NormalTok{  \}}
  
  \CommentTok{\# Função para detectar mudança}
\NormalTok{  detect\_change }\OtherTok{\textless{}{-}} \ControlFlowTok{function}\NormalTok{() \{}
\NormalTok{    mean\_val }\OtherTok{\textless{}{-}} \FunctionTok{mean}\NormalTok{(window) }\CommentTok{\# Calcula a média dos valores na janela}
    \ControlFlowTok{for}\NormalTok{ (n }\ControlFlowTok{in} \DecValTok{1}\SpecialCharTok{:}\NormalTok{(width }\SpecialCharTok{{-}} \DecValTok{1}\NormalTok{)) \{}
      \CommentTok{\# Divide a janela em duas sub{-}janelas e calcula as médias de cada sub{-}janela}
\NormalTok{      mean0 }\OtherTok{\textless{}{-}} \FunctionTok{mean}\NormalTok{(window[}\DecValTok{1}\SpecialCharTok{:}\NormalTok{n])}
\NormalTok{      mean1 }\OtherTok{\textless{}{-}} \FunctionTok{mean}\NormalTok{(window[(n }\SpecialCharTok{+} \DecValTok{1}\NormalTok{)}\SpecialCharTok{:}\NormalTok{width])}
      \CommentTok{\# Compara as médias das sub{-}janelas usando um teste estatístico}
      \ControlFlowTok{if}\NormalTok{ (}\FunctionTok{abs}\NormalTok{(mean0 }\SpecialCharTok{{-}}\NormalTok{ mean1) }\SpecialCharTok{\textgreater{}} \FunctionTok{sqrt}\NormalTok{((variance }\SpecialCharTok{/}\NormalTok{ n) }\SpecialCharTok{+}\NormalTok{ (variance }\SpecialCharTok{/}\NormalTok{ (width }\SpecialCharTok{{-}}\NormalTok{ n))) }\SpecialCharTok{*} \FunctionTok{qnorm}\NormalTok{(}\DecValTok{1} \SpecialCharTok{{-}}\NormalTok{ delta)) \{}
        \CommentTok{\# Se uma mudança for detectada, ajusta a janela para descartar os dados antigos}
\NormalTok{        window }\OtherTok{\textless{}\textless{}{-}}\NormalTok{ window[(n }\SpecialCharTok{+} \DecValTok{1}\NormalTok{)}\SpecialCharTok{:}\NormalTok{width]}
\NormalTok{        width }\OtherTok{\textless{}\textless{}{-}} \FunctionTok{length}\NormalTok{(window)}
\NormalTok{        total }\OtherTok{\textless{}\textless{}{-}} \FunctionTok{sum}\NormalTok{(window)}
\NormalTok{        variance }\OtherTok{\textless{}\textless{}{-}} \FunctionTok{var}\NormalTok{(window)}
        \FunctionTok{return}\NormalTok{(}\ConstantTok{TRUE}\NormalTok{) }\CommentTok{\# Retorna TRUE indicando que uma mudança foi detectada}
\NormalTok{      \}}
\NormalTok{    \}}
    \FunctionTok{return}\NormalTok{(}\ConstantTok{FALSE}\NormalTok{) }\CommentTok{\# Retorna FALSE se nenhuma mudança for detectada}
\NormalTok{  \}}
  
  \CommentTok{\# Retorna as funções de atualização e detecção}
  \FunctionTok{list}\NormalTok{(}\AttributeTok{update =}\NormalTok{ update)}
\NormalTok{\}}

\CommentTok{\# Geração da série temporal e detecção de mudanças}
\FunctionTok{set.seed}\NormalTok{(}\DecValTok{1234567890}\NormalTok{, }\AttributeTok{kind=}\StringTok{"Mersenne{-}Twister"}\NormalTok{) }\CommentTok{\# Define a semente para reprodutibilidade}
\NormalTok{y }\OtherTok{\textless{}{-}} \FunctionTok{TrickyTimeSeries}\NormalTok{(}\AttributeTok{n=}\DecValTok{100}\NormalTok{, }\AttributeTok{k=}\DecValTok{1}\NormalTok{) }\CommentTok{\# Gera a série temporal com duas médias diferentes}

\CommentTok{\# Inicializa o detector ADWIN}
\NormalTok{adwin }\OtherTok{\textless{}{-}} \FunctionTok{ADWIN}\NormalTok{(}\AttributeTok{delta =} \FloatTok{0.002}\NormalTok{)}

\CommentTok{\# Vetor para armazenar os pontos de detecção de mudança}
\NormalTok{change\_points }\OtherTok{\textless{}{-}} \FunctionTok{numeric}\NormalTok{(}\DecValTok{0}\NormalTok{)}

\CommentTok{\# Processa o fluxo de dados}
\ControlFlowTok{for}\NormalTok{ (i }\ControlFlowTok{in} \DecValTok{1}\SpecialCharTok{:}\FunctionTok{length}\NormalTok{(y)) \{}
  \ControlFlowTok{if}\NormalTok{ (adwin}\SpecialCharTok{$}\FunctionTok{update}\NormalTok{(y[i])) \{}
\NormalTok{    change\_points }\OtherTok{\textless{}{-}} \FunctionTok{c}\NormalTok{(change\_points, i) }\CommentTok{\# Armazena o índice onde a mudança foi detectada}
\NormalTok{  \}}
\NormalTok{\}}

\CommentTok{\# Criação do data frame para plotagem}
\NormalTok{Tricky }\OtherTok{\textless{}{-}} \FunctionTok{data.frame}\NormalTok{(}\AttributeTok{x =} \DecValTok{1}\SpecialCharTok{:}\FunctionTok{length}\NormalTok{(y), }\AttributeTok{y =}\NormalTok{ y, }\AttributeTok{change =} \FunctionTok{ifelse}\NormalTok{(}\DecValTok{1}\SpecialCharTok{:}\FunctionTok{length}\NormalTok{(y) }\SpecialCharTok{\%in\%}\NormalTok{ change\_points, }\StringTok{"Change Detected"}\NormalTok{, }\StringTok{"No Change"}\NormalTok{))}

\CommentTok{\# Plotar a série temporal com pontos de mudança}
\FunctionTok{ggplot}\NormalTok{(Tricky, }\FunctionTok{aes}\NormalTok{(}\AttributeTok{x =}\NormalTok{ x, }\AttributeTok{y =}\NormalTok{ y)) }\SpecialCharTok{+}
  \FunctionTok{geom\_line}\NormalTok{() }\SpecialCharTok{+}
  \FunctionTok{geom\_point}\NormalTok{(}\AttributeTok{data =} \FunctionTok{subset}\NormalTok{(Tricky, change }\SpecialCharTok{==} \StringTok{"Change Detected"}\NormalTok{), }\FunctionTok{aes}\NormalTok{(}\AttributeTok{x =}\NormalTok{ x, }\AttributeTok{y =}\NormalTok{ y), }\AttributeTok{color =} \StringTok{"red"}\NormalTok{, }\AttributeTok{size =} \DecValTok{2}\NormalTok{) }\SpecialCharTok{+}
  \FunctionTok{labs}\NormalTok{(}\AttributeTok{title =} \StringTok{"Change Detected with ADWIN in Série TrickyTimeSeries"}\NormalTok{, }\AttributeTok{x =} \StringTok{"Índice"}\NormalTok{, }\AttributeTok{y =} \StringTok{"Valor"}\NormalTok{) }\SpecialCharTok{+}
  \FunctionTok{theme\_minimal}\NormalTok{() }\SpecialCharTok{+}
  \FunctionTok{theme\_tufte}\NormalTok{()}
\end{Highlighting}
\end{Shaded}

\includegraphics{TrickyTimeSeries_files/figure-latex/unnamed-chunk-13-1.pdf}

\hypertarget{change-detected-with-adwin-trickytimeseries-plano-hxc}{%
\subsubsection{Change Detected with ADWIN (TrickyTimeSeries) Plano
HxC}\label{change-detected-with-adwin-trickytimeseries-plano-hxc}}

Para visualizar a evolução dos padrões ao longo do tempo e obter vários
pontos no plano HxC, é preciso calcular essas métricas em janelas
deslizantes ao longo da série temporal. O plano HxC plota a entropia de
Shannon (H) no eixo x e a complexidade de Jensen-Shannon (C) no eixo y
para janelas deslizantes da série temporal. O gráfico mostra os pontos
calculados para cada janela.

\begin{Shaded}
\begin{Highlighting}[]
\CommentTok{\# Carregar bibliotecas necessárias}
\FunctionTok{library}\NormalTok{(ggplot2)}
\FunctionTok{library}\NormalTok{(ggthemes)}

\CommentTok{\# Função para calcular a entropia de Shannon}
\NormalTok{shannon\_entropy }\OtherTok{\textless{}{-}} \ControlFlowTok{function}\NormalTok{(probabilities) \{}
  \SpecialCharTok{{-}}\FunctionTok{sum}\NormalTok{(probabilities }\SpecialCharTok{*} \FunctionTok{log2}\NormalTok{(probabilities), }\AttributeTok{na.rm =} \ConstantTok{TRUE}\NormalTok{)}
\NormalTok{\}}

\CommentTok{\# Função para calcular a complexidade de Jensen{-}Shannon}
\NormalTok{js\_complexity }\OtherTok{\textless{}{-}} \ControlFlowTok{function}\NormalTok{(probabilities) \{}
\NormalTok{  q }\OtherTok{\textless{}{-}} \FunctionTok{rep}\NormalTok{(}\DecValTok{1}\SpecialCharTok{/}\FunctionTok{length}\NormalTok{(probabilities), }\FunctionTok{length}\NormalTok{(probabilities))}
\NormalTok{  m }\OtherTok{\textless{}{-}}\NormalTok{ (probabilities }\SpecialCharTok{+}\NormalTok{ q) }\SpecialCharTok{/} \DecValTok{2}
\NormalTok{  (}\FunctionTok{shannon\_entropy}\NormalTok{(m) }\SpecialCharTok{{-}} \FloatTok{0.5} \SpecialCharTok{*}\NormalTok{ (}\FunctionTok{shannon\_entropy}\NormalTok{(probabilities) }\SpecialCharTok{+} \FunctionTok{shannon\_entropy}\NormalTok{(q))) }\SpecialCharTok{/} \FunctionTok{log2}\NormalTok{(}\FunctionTok{length}\NormalTok{(probabilities))}
\NormalTok{\}}

\CommentTok{\# Função para calcular H e C em janelas deslizantes}
\NormalTok{calculate\_hxc }\OtherTok{\textless{}{-}} \ControlFlowTok{function}\NormalTok{(series, window\_size) \{}
\NormalTok{  n }\OtherTok{\textless{}{-}} \FunctionTok{length}\NormalTok{(series)}
\NormalTok{  h\_values }\OtherTok{\textless{}{-}} \FunctionTok{c}\NormalTok{()}
\NormalTok{  c\_values }\OtherTok{\textless{}{-}} \FunctionTok{c}\NormalTok{()}
  
  \ControlFlowTok{for}\NormalTok{ (i }\ControlFlowTok{in} \DecValTok{1}\SpecialCharTok{:}\NormalTok{(n }\SpecialCharTok{{-}}\NormalTok{ window\_size }\SpecialCharTok{+} \DecValTok{1}\NormalTok{)) \{}
\NormalTok{    window }\OtherTok{\textless{}{-}}\NormalTok{ series[i}\SpecialCharTok{:}\NormalTok{(i }\SpecialCharTok{+}\NormalTok{ window\_size }\SpecialCharTok{{-}} \DecValTok{1}\NormalTok{)]}
\NormalTok{    probabilities }\OtherTok{\textless{}{-}} \FunctionTok{hist}\NormalTok{(window, }\AttributeTok{plot =} \ConstantTok{FALSE}\NormalTok{)}\SpecialCharTok{$}\NormalTok{density}
\NormalTok{    h }\OtherTok{\textless{}{-}} \FunctionTok{shannon\_entropy}\NormalTok{(probabilities)}
\NormalTok{    c }\OtherTok{\textless{}{-}} \FunctionTok{js\_complexity}\NormalTok{(probabilities)}
\NormalTok{    h\_values }\OtherTok{\textless{}{-}} \FunctionTok{c}\NormalTok{(h\_values, h)}
\NormalTok{    c\_values }\OtherTok{\textless{}{-}} \FunctionTok{c}\NormalTok{(c\_values, c)}
\NormalTok{  \}}
  
  \FunctionTok{data.frame}\NormalTok{(}\AttributeTok{H =}\NormalTok{ h\_values, }\AttributeTok{C =}\NormalTok{ c\_values)}
\NormalTok{\}}

\CommentTok{\# Função ADWIN para detectar mudanças de conceito em fluxos de dados}
\NormalTok{ADWIN }\OtherTok{\textless{}{-}} \ControlFlowTok{function}\NormalTok{(}\AttributeTok{delta =} \FloatTok{0.002}\NormalTok{) \{}
  \CommentTok{\# Inicializa as variáveis}
\NormalTok{  width }\OtherTok{\textless{}{-}} \DecValTok{0} \CommentTok{\# Tamanho da janela}
\NormalTok{  total }\OtherTok{\textless{}{-}} \DecValTok{0} \CommentTok{\# Soma dos valores na janela}
\NormalTok{  variance }\OtherTok{\textless{}{-}} \DecValTok{0} \CommentTok{\# Variância dos valores na janela}
\NormalTok{  window }\OtherTok{\textless{}{-}} \FunctionTok{numeric}\NormalTok{(}\DecValTok{0}\NormalTok{) }\CommentTok{\# Vetor que armazena os valores na janela}
  
  \CommentTok{\# Função para atualizar o ADWIN com um novo valor}
\NormalTok{  update }\OtherTok{\textless{}{-}} \ControlFlowTok{function}\NormalTok{(value) \{}
\NormalTok{    width }\OtherTok{\textless{}\textless{}{-}}\NormalTok{ width }\SpecialCharTok{+} \DecValTok{1} \CommentTok{\# Incrementa o tamanho da janela}
\NormalTok{    window }\OtherTok{\textless{}\textless{}{-}} \FunctionTok{c}\NormalTok{(window, value) }\CommentTok{\# Adiciona o novo valor à janela}
\NormalTok{    total }\OtherTok{\textless{}\textless{}{-}}\NormalTok{ total }\SpecialCharTok{+}\NormalTok{ value }\CommentTok{\# Atualiza a soma total}
    \ControlFlowTok{if}\NormalTok{ (width }\SpecialCharTok{\textgreater{}} \DecValTok{1}\NormalTok{) \{}
\NormalTok{      variance }\OtherTok{\textless{}\textless{}{-}} \FunctionTok{var}\NormalTok{(window) }\CommentTok{\# Calcula a variância se a janela tiver mais de um valor}
\NormalTok{    \}}
    \CommentTok{\# Checa por concept drift}
    \ControlFlowTok{if}\NormalTok{ (width }\SpecialCharTok{\textgreater{}} \DecValTok{1} \SpecialCharTok{\&\&} \FunctionTok{detect\_change}\NormalTok{()) \{}
      \FunctionTok{return}\NormalTok{(}\ConstantTok{TRUE}\NormalTok{) }\CommentTok{\# Retorna TRUE se uma mudança for detectada}
\NormalTok{    \}}
    \FunctionTok{return}\NormalTok{(}\ConstantTok{FALSE}\NormalTok{) }\CommentTok{\# Retorna FALSE se nenhuma mudança for detectada}
\NormalTok{  \}}
  
  \CommentTok{\# Função para detectar mudança}
\NormalTok{  detect\_change }\OtherTok{\textless{}{-}} \ControlFlowTok{function}\NormalTok{() \{}
\NormalTok{    mean\_val }\OtherTok{\textless{}{-}} \FunctionTok{mean}\NormalTok{(window) }\CommentTok{\# Calcula a média dos valores na janela}
    \ControlFlowTok{for}\NormalTok{ (n }\ControlFlowTok{in} \DecValTok{1}\SpecialCharTok{:}\NormalTok{(width }\SpecialCharTok{{-}} \DecValTok{1}\NormalTok{)) \{}
      \CommentTok{\# Divide a janela em duas sub{-}janelas e calcula as médias de cada sub{-}janela}
\NormalTok{      mean0 }\OtherTok{\textless{}{-}} \FunctionTok{mean}\NormalTok{(window[}\DecValTok{1}\SpecialCharTok{:}\NormalTok{n])}
\NormalTok{      mean1 }\OtherTok{\textless{}{-}} \FunctionTok{mean}\NormalTok{(window[(n }\SpecialCharTok{+} \DecValTok{1}\NormalTok{)}\SpecialCharTok{:}\NormalTok{width])}
      \CommentTok{\# Compara as médias das sub{-}janelas usando um teste estatístico}
      \ControlFlowTok{if}\NormalTok{ (}\FunctionTok{abs}\NormalTok{(mean0 }\SpecialCharTok{{-}}\NormalTok{ mean1) }\SpecialCharTok{\textgreater{}} \FunctionTok{sqrt}\NormalTok{((variance }\SpecialCharTok{/}\NormalTok{ n) }\SpecialCharTok{+}\NormalTok{ (variance }\SpecialCharTok{/}\NormalTok{ (width }\SpecialCharTok{{-}}\NormalTok{ n))) }\SpecialCharTok{*} \FunctionTok{qnorm}\NormalTok{(}\DecValTok{1} \SpecialCharTok{{-}}\NormalTok{ delta)) \{}
        \CommentTok{\# Se uma mudança for detectada, ajusta a janela para descartar os dados antigos}
\NormalTok{        window }\OtherTok{\textless{}\textless{}{-}}\NormalTok{ window[(n }\SpecialCharTok{+} \DecValTok{1}\NormalTok{)}\SpecialCharTok{:}\NormalTok{width]}
\NormalTok{        width }\OtherTok{\textless{}\textless{}{-}} \FunctionTok{length}\NormalTok{(window)}
\NormalTok{        total }\OtherTok{\textless{}\textless{}{-}} \FunctionTok{sum}\NormalTok{(window)}
\NormalTok{        variance }\OtherTok{\textless{}\textless{}{-}} \FunctionTok{var}\NormalTok{(window)}
        \FunctionTok{return}\NormalTok{(}\ConstantTok{TRUE}\NormalTok{) }\CommentTok{\# Retorna TRUE indicando que uma mudança foi detectada}
\NormalTok{      \}}
\NormalTok{    \}}
    \FunctionTok{return}\NormalTok{(}\ConstantTok{FALSE}\NormalTok{) }\CommentTok{\# Retorna FALSE se nenhuma mudança for detectada}
\NormalTok{  \}}
  
  \CommentTok{\# Retorna as funções de atualização e detecção}
  \FunctionTok{list}\NormalTok{(}\AttributeTok{update =}\NormalTok{ update)}
\NormalTok{\}}

\CommentTok{\# Função para gerar ruído colorido}
\NormalTok{TK95 }\OtherTok{\textless{}{-}} \ControlFlowTok{function}\NormalTok{(N, }\AttributeTok{alpha =} \DecValTok{1}\NormalTok{) \{ }
\NormalTok{  f }\OtherTok{\textless{}{-}} \FunctionTok{seq}\NormalTok{(}\AttributeTok{from =} \DecValTok{0}\NormalTok{, }\AttributeTok{to =}\NormalTok{ pi, }\AttributeTok{length.out =}\NormalTok{ (N}\SpecialCharTok{/}\DecValTok{2} \SpecialCharTok{+} \DecValTok{1}\NormalTok{))[}\SpecialCharTok{{-}}\FunctionTok{c}\NormalTok{(}\DecValTok{1}\NormalTok{, (N}\SpecialCharTok{/}\DecValTok{2} \SpecialCharTok{+} \DecValTok{1}\NormalTok{))]}
\NormalTok{  f\_ }\OtherTok{\textless{}{-}} \DecValTok{1} \SpecialCharTok{/}\NormalTok{ f}\SpecialCharTok{\^{}}\NormalTok{alpha}
\NormalTok{  RW }\OtherTok{\textless{}{-}} \FunctionTok{sqrt}\NormalTok{(}\FloatTok{0.5} \SpecialCharTok{*}\NormalTok{ f\_) }\SpecialCharTok{*} \FunctionTok{rnorm}\NormalTok{(N}\SpecialCharTok{/}\DecValTok{2} \SpecialCharTok{{-}} \DecValTok{1}\NormalTok{)}
\NormalTok{  IW }\OtherTok{\textless{}{-}} \FunctionTok{sqrt}\NormalTok{(}\FloatTok{0.5} \SpecialCharTok{*}\NormalTok{ f\_) }\SpecialCharTok{*} \FunctionTok{rnorm}\NormalTok{(N}\SpecialCharTok{/}\DecValTok{2} \SpecialCharTok{{-}} \DecValTok{1}\NormalTok{)}
\NormalTok{  fR }\OtherTok{\textless{}{-}} \FunctionTok{complex}\NormalTok{(}\AttributeTok{real =} \FunctionTok{c}\NormalTok{(}\FunctionTok{rnorm}\NormalTok{(}\DecValTok{1}\NormalTok{), RW, }\FunctionTok{rnorm}\NormalTok{(}\DecValTok{1}\NormalTok{), RW[(N}\SpecialCharTok{/}\DecValTok{2} \SpecialCharTok{{-}} \DecValTok{1}\NormalTok{)}\SpecialCharTok{:}\DecValTok{1}\NormalTok{]), }
                \AttributeTok{imaginary =} \FunctionTok{c}\NormalTok{(}\DecValTok{0}\NormalTok{, IW, }\DecValTok{0}\NormalTok{, }\SpecialCharTok{{-}}\NormalTok{IW[(N}\SpecialCharTok{/}\DecValTok{2} \SpecialCharTok{{-}} \DecValTok{1}\NormalTok{)}\SpecialCharTok{:}\DecValTok{1}\NormalTok{]), }\AttributeTok{length.out =}\NormalTok{ N)}
\NormalTok{  reihe }\OtherTok{\textless{}{-}} \FunctionTok{fft}\NormalTok{(fR, }\AttributeTok{inverse =} \ConstantTok{TRUE}\NormalTok{)}
  \FunctionTok{return}\NormalTok{(}\FunctionTok{Re}\NormalTok{(reihe))}
\NormalTok{\}}

\CommentTok{\# Função para criar a série temporal}
\NormalTok{TrickyTimeSeries }\OtherTok{\textless{}{-}} \ControlFlowTok{function}\NormalTok{(n, k) \{}
\NormalTok{  y1 }\OtherTok{\textless{}{-}} \FunctionTok{rnorm}\NormalTok{(n)}
\NormalTok{  y2 }\OtherTok{\textless{}{-}} \FunctionTok{sqrt}\NormalTok{(}\DecValTok{12}\NormalTok{) }\SpecialCharTok{*} \FunctionTok{runif}\NormalTok{(n, }\AttributeTok{min =} \SpecialCharTok{{-}}\DecValTok{1}\SpecialCharTok{/}\DecValTok{2}\NormalTok{, }\AttributeTok{max =} \DecValTok{1}\SpecialCharTok{/}\DecValTok{2}\NormalTok{)}
\NormalTok{  y3 }\OtherTok{\textless{}{-}} \FunctionTok{rexp}\NormalTok{(n) }\SpecialCharTok{{-}} \DecValTok{1}
\NormalTok{  fk }\OtherTok{\textless{}{-}} \FunctionTok{TK95}\NormalTok{(}\AttributeTok{N =} \DecValTok{3} \SpecialCharTok{*}\NormalTok{ n, }\AttributeTok{alpha =}\NormalTok{ k)}
\NormalTok{  fk }\OtherTok{\textless{}{-}}\NormalTok{ (fk }\SpecialCharTok{{-}} \FunctionTok{mean}\NormalTok{(fk)) }\SpecialCharTok{/} \FunctionTok{sd}\NormalTok{(fk)}
\NormalTok{  output }\OtherTok{\textless{}{-}} \FunctionTok{c}\NormalTok{(y1, y2, y3, fk)}
  \FunctionTok{return}\NormalTok{(output)}
\NormalTok{\}}

\CommentTok{\# Geração da série temporal e detecção de mudanças}
\FunctionTok{set.seed}\NormalTok{(}\DecValTok{1234567890}\NormalTok{, }\AttributeTok{kind =} \StringTok{"Mersenne{-}Twister"}\NormalTok{)}
\NormalTok{y }\OtherTok{\textless{}{-}} \FunctionTok{TrickyTimeSeries}\NormalTok{(}\AttributeTok{n =} \DecValTok{100}\NormalTok{, }\AttributeTok{k =} \DecValTok{1}\NormalTok{)}

\CommentTok{\# Inicializa o detector ADWIN}
\NormalTok{adwin }\OtherTok{\textless{}{-}} \FunctionTok{ADWIN}\NormalTok{(}\AttributeTok{delta =} \FloatTok{0.002}\NormalTok{)}

\CommentTok{\# Vetor para armazenar os pontos de detecção de mudança}
\NormalTok{change\_points }\OtherTok{\textless{}{-}} \FunctionTok{numeric}\NormalTok{(}\DecValTok{0}\NormalTok{)}

\CommentTok{\# Processa o fluxo de dados}
\ControlFlowTok{for}\NormalTok{ (i }\ControlFlowTok{in} \DecValTok{1}\SpecialCharTok{:}\FunctionTok{length}\NormalTok{(y)) \{}
  \ControlFlowTok{if}\NormalTok{ (adwin}\SpecialCharTok{$}\FunctionTok{update}\NormalTok{(y[i])) \{}
\NormalTok{    change\_points }\OtherTok{\textless{}{-}} \FunctionTok{c}\NormalTok{(change\_points, i)}
\NormalTok{  \}}
\NormalTok{\}}

\CommentTok{\# Calcular H e C em janelas deslizantes}
\NormalTok{window\_size }\OtherTok{\textless{}{-}} \DecValTok{50}
\NormalTok{hxc\_data }\OtherTok{\textless{}{-}} \FunctionTok{calculate\_hxc}\NormalTok{(y, window\_size)}

\CommentTok{\# Plotar o plano HxC}
\FunctionTok{ggplot}\NormalTok{(hxc\_data, }\FunctionTok{aes}\NormalTok{(}\AttributeTok{x =}\NormalTok{ H, }\AttributeTok{y =}\NormalTok{ C)) }\SpecialCharTok{+}
  \FunctionTok{geom\_point}\NormalTok{(}\AttributeTok{color =} \StringTok{"blue"}\NormalTok{) }\SpecialCharTok{+}
  \FunctionTok{labs}\NormalTok{(}\AttributeTok{title =} \StringTok{"Plano HxC with ADWIN (TrickyTimeSeries)"}\NormalTok{,}
       \AttributeTok{x =} \StringTok{"Entropia de Shannon (H)"}\NormalTok{, }\AttributeTok{y =} \StringTok{"Complexidade de Jensen{-}Shannon (C)"}\NormalTok{) }\SpecialCharTok{+}
  \FunctionTok{theme\_minimal}\NormalTok{() }\SpecialCharTok{+}
  \FunctionTok{geom\_smooth}\NormalTok{(}\AttributeTok{method =} \StringTok{"lm"}\NormalTok{, }\AttributeTok{color =} \StringTok{"red"}\NormalTok{) }\SpecialCharTok{+}
  \FunctionTok{theme\_bw}\NormalTok{()}
\end{Highlighting}
\end{Shaded}

\begin{verbatim}
## `geom_smooth()` using formula = 'y ~ x'
\end{verbatim}

\includegraphics{TrickyTimeSeries_files/figure-latex/unnamed-chunk-14-1.pdf}

\begin{Shaded}
\begin{Highlighting}[]
\CommentTok{\# Criação do data frame para plotagem}
\NormalTok{Tricky }\OtherTok{\textless{}{-}} \FunctionTok{data.frame}\NormalTok{(}
  \AttributeTok{x =} \DecValTok{1}\SpecialCharTok{:}\FunctionTok{length}\NormalTok{(y), }
  \AttributeTok{y =}\NormalTok{ y, }
  \AttributeTok{change =} \FunctionTok{ifelse}\NormalTok{(}\DecValTok{1}\SpecialCharTok{:}\FunctionTok{length}\NormalTok{(y) }\SpecialCharTok{\%in\%}\NormalTok{ change\_points, }\StringTok{"Change Detected"}\NormalTok{, }\StringTok{"No Change"}\NormalTok{)}
\NormalTok{)}

\CommentTok{\# Plotar a série temporal com pontos de mudança}
\FunctionTok{ggplot}\NormalTok{(Tricky, }\FunctionTok{aes}\NormalTok{(}\AttributeTok{x =}\NormalTok{ x, }\AttributeTok{y =}\NormalTok{ y)) }\SpecialCharTok{+}
  \FunctionTok{geom\_line}\NormalTok{() }\SpecialCharTok{+}
  \FunctionTok{geom\_point}\NormalTok{(}\AttributeTok{data =} \FunctionTok{subset}\NormalTok{(Tricky, change }\SpecialCharTok{==} \StringTok{"Change Detected"}\NormalTok{), }\FunctionTok{aes}\NormalTok{(}\AttributeTok{x =}\NormalTok{ x, }\AttributeTok{y =}\NormalTok{ y), }\AttributeTok{color =} \StringTok{"red"}\NormalTok{, }\AttributeTok{size =} \DecValTok{2}\NormalTok{) }\SpecialCharTok{+}
  \FunctionTok{labs}\NormalTok{(}\AttributeTok{title =} \StringTok{"Change Detected with ADWIN (TrickyTimeSeries)"}\NormalTok{, }\AttributeTok{x =} \StringTok{"Índice"}\NormalTok{, }\AttributeTok{y =} \StringTok{"Valor"}\NormalTok{) }\SpecialCharTok{+}
  \FunctionTok{theme\_minimal}\NormalTok{() }\SpecialCharTok{+}
  \FunctionTok{theme\_tufte}\NormalTok{()}
\end{Highlighting}
\end{Shaded}

\includegraphics{TrickyTimeSeries_files/figure-latex/unnamed-chunk-14-2.pdf}
\#\# TrickyTimeSeries

\hypertarget{funuxe7uxe3o-tk95}{%
\paragraph{\texorpdfstring{Função
\texttt{TK95}:}{Função TK95:}}\label{funuxe7uxe3o-tk95}}

\begin{itemize}
\tightlist
\item
  Gera ruído colorido usando a transformada inversa de Fourier.
\item
  \texttt{N} define o número de pontos, \texttt{alpha} controla a
  correlação do ruído.
\item
  Frequências de Fourier são calculadas e a lei de potência é aplicada.
\item
  Componentes real e imaginária são geradas e combinadas.
\item
  A transformada inversa de Fourier é aplicada para retornar ao domínio
  do tempo.
\end{itemize}

\hypertarget{funuxe7uxe3o-trickytimeseries}{%
\paragraph{\texorpdfstring{Função
\texttt{TrickyTimeSeries}:}{Função TrickyTimeSeries:}}\label{funuxe7uxe3o-trickytimeseries}}

\begin{itemize}
\tightlist
\item
  Gera três tipos de ruído branco (gaussiano, uniforme escalado,
  exponencial não central).
\item
  Gera ruído colorido 1/f\^{}k e padroniza.
\item
  Combina os diferentes tipos de ruído em uma única série temporal.
\end{itemize}

\hypertarget{funuxe7uxe3o-adwin}{%
\paragraph{\texorpdfstring{Função
\texttt{ADWIN}:}{Função ADWIN:}}\label{funuxe7uxe3o-adwin}}

\begin{itemize}
\tightlist
\item
  Detecta mudanças de conceito em fluxos de dados.
\item
  Inicializa variáveis para armazenar a janela deslizante e suas
  estatísticas.
\item
  \texttt{update}: Adiciona novos valores à janela e verifica se há
  mudança.
\item
  \texttt{detect\_change}: Compara médias de sub-janelas usando um teste
  estatístico.
\item
  Ajusta a janela quando uma mudança é detectada.
\end{itemize}

\hypertarget{gerauxe7uxe3o-da-suxe9rie-temporal-e-detecuxe7uxe3o-de-mudanuxe7as}{%
\paragraph{Geração da Série Temporal e Detecção de
Mudanças:}\label{gerauxe7uxe3o-da-suxe9rie-temporal-e-detecuxe7uxe3o-de-mudanuxe7as}}

\begin{itemize}
\tightlist
\item
  Gera a série temporal \texttt{TrickyTimeSeries}.
\item
  Inicializa o detector \texttt{ADWIN}.
\item
  Processa a série temporal e armazena os índices onde mudanças são
  detectadas.
\end{itemize}

\hypertarget{visualizauxe7uxe3o}{%
\paragraph{Visualização:}\label{visualizauxe7uxe3o}}

\begin{itemize}
\tightlist
\item
  Cria um data frame com os dados e pontos de mudança.
\item
  Plota a série temporal com pontos vermelhos destacando as mudanças
  detectadas.
\end{itemize}

Então temos um gráfico que mostra a série temporal com pontos destacados
onde o \texttt{ADWIN} detectou mudanças. Isso permite visualizar como a
janela é ajustada e como as mudanças na média dos dados são
identificadas ao longo do tempo.

\begin{center}\rule{0.5\linewidth}{0.5pt}\end{center}

\hypertarget{padruxf5es-ordinais-with-adwin-trickytimeseries}{%
\subsubsection{Padrões Ordinais with ADWIN
(TrickyTimeSeries)}\label{padruxf5es-ordinais-with-adwin-trickytimeseries}}

Usar padrões ordinais para ajudar na detecção de mudanças com o ADWIN
pode trazer benefícios significativos, especialmente em séries temporais
não estacionárias. Padrões ordinais capturam a ordem relativa dos
valores na série temporal, permitindo a detecção de mudanças na
estrutura da série, independentemente da escala absoluta dos valores.
\textbf{Isso pode complementar a detecção de mudanças baseada em média e
variância usada pelo ADWIN, proporcionando uma detecção mais robusta.}

Implementar a detecção de mudanças usando padrões ordinais junto com o
ADWIN.

\hypertarget{passos}{%
\paragraph{Passos:}\label{passos}}

\begin{enumerate}
\def\labelenumi{\arabic{enumi}.}
\tightlist
\item
  Calcular Padrões Ordinais:
\end{enumerate}

\begin{itemize}
\item
  Extrair padrões ordinais de subsequências da série temporal.
\item
  Contar a frequência de cada padrão ordinal.
\end{itemize}

\begin{enumerate}
\def\labelenumi{\arabic{enumi}.}
\setcounter{enumi}{1}
\tightlist
\item
  Usar ADWIN para Detecção de Mudanças:
\end{enumerate}

\begin{itemize}
\tightlist
\item
  Aplicar ADWIN na frequência dos padrões ordinais.
\end{itemize}

\begin{Shaded}
\begin{Highlighting}[]
\CommentTok{\# \# Carregar bibliotecas necessárias}
\CommentTok{\# library(pracma)}
\CommentTok{\# library(ggplot2)}
\CommentTok{\# library(ggthemes)}
\CommentTok{\# }
\CommentTok{\# \# Função para calcular padrões ordinais usando pracma}
\CommentTok{\# ordinal\_patterns \textless{}{-} function(series, emb\_dim) \{}
\CommentTok{\#   n \textless{}{-} length(series)}
\CommentTok{\#   if (n \textless{} emb\_dim) \{}
\CommentTok{\#     stop("A série temporal é muito curta para a dimensão de embedding especificada.")}
\CommentTok{\#   \}}
\CommentTok{\#   }
\CommentTok{\#   patterns \textless{}{-} numeric(n {-} emb\_dim + 1)}
\CommentTok{\#   for (i in 1:(n {-} emb\_dim + 1)) \{}
\CommentTok{\#     subseq \textless{}{-} series[i:(i + emb\_dim {-} 1)]}
\CommentTok{\#     ranks \textless{}{-} rank(subseq, ties.method = "first")}
\CommentTok{\#     pattern \textless{}{-} sum((ranks {-} 1) * (emb\_dim \^{} (0:(emb\_dim {-} 1))))}
\CommentTok{\#     patterns[i] \textless{}{-} pattern}
\CommentTok{\#   \}}
\CommentTok{\#   return(patterns)}
\CommentTok{\# \}}

\CommentTok{\# Carregar bibliotecas necessárias}
\FunctionTok{library}\NormalTok{(pracma)}
\FunctionTok{library}\NormalTok{(ggplot2)}
\FunctionTok{library}\NormalTok{(ggthemes)}
\FunctionTok{library}\NormalTok{(statcomp)}

\CommentTok{\# Função para calcular padrões ordinais usando statcomp}
\NormalTok{ordinal\_patterns\_statcomp }\OtherTok{\textless{}{-}} \ControlFlowTok{function}\NormalTok{(series, emb\_dim) \{}
  \ControlFlowTok{if}\NormalTok{ (}\FunctionTok{length}\NormalTok{(series) }\SpecialCharTok{\textless{}}\NormalTok{ emb\_dim) \{}
    \FunctionTok{stop}\NormalTok{(}\StringTok{"A série temporal é muito curta para a dimensão de embedding especificada."}\NormalTok{)}
\NormalTok{  \}}
  
  \CommentTok{\# Utilizando a função from \textasciigrave{}statcomp\textasciigrave{} para calcular os padrões ordinais}
\NormalTok{  patterns }\OtherTok{\textless{}{-}} \FunctionTok{ordinal\_pattern}\NormalTok{(series, emb\_dim)}
  
  \FunctionTok{return}\NormalTok{(patterns)}
\NormalTok{\}}

\CommentTok{\# Função ADWIN ajustada para padrões ordinais}
\NormalTok{ADWIN }\OtherTok{\textless{}{-}} \ControlFlowTok{function}\NormalTok{(}\AttributeTok{delta =} \FloatTok{0.002}\NormalTok{) \{}
\NormalTok{  width }\OtherTok{\textless{}{-}} \DecValTok{0}
\NormalTok{  total }\OtherTok{\textless{}{-}} \DecValTok{0}
\NormalTok{  variance }\OtherTok{\textless{}{-}} \DecValTok{0}
\NormalTok{  window }\OtherTok{\textless{}{-}} \FunctionTok{numeric}\NormalTok{(}\DecValTok{0}\NormalTok{)}
  
\NormalTok{  update }\OtherTok{\textless{}{-}} \ControlFlowTok{function}\NormalTok{(value) \{}
\NormalTok{    width }\OtherTok{\textless{}\textless{}{-}}\NormalTok{ width }\SpecialCharTok{+} \DecValTok{1}
\NormalTok{    window }\OtherTok{\textless{}\textless{}{-}} \FunctionTok{c}\NormalTok{(window, value)}
\NormalTok{    total }\OtherTok{\textless{}\textless{}{-}}\NormalTok{ total }\SpecialCharTok{+}\NormalTok{ value}
    \ControlFlowTok{if}\NormalTok{ (width }\SpecialCharTok{\textgreater{}} \DecValTok{1}\NormalTok{) \{}
\NormalTok{      variance }\OtherTok{\textless{}\textless{}{-}} \FunctionTok{var}\NormalTok{(window, }\AttributeTok{na.rm =} \ConstantTok{TRUE}\NormalTok{) }\CommentTok{\# Calcula a variância, removendo NAs}
\NormalTok{    \}}
    \ControlFlowTok{if}\NormalTok{ (width }\SpecialCharTok{\textgreater{}} \DecValTok{1} \SpecialCharTok{\&\&} \FunctionTok{detect\_change}\NormalTok{()) \{}
      \FunctionTok{return}\NormalTok{(}\ConstantTok{TRUE}\NormalTok{)}
\NormalTok{    \}}
    \FunctionTok{return}\NormalTok{(}\ConstantTok{FALSE}\NormalTok{)}
\NormalTok{  \}}
  
\NormalTok{  detect\_change }\OtherTok{\textless{}{-}} \ControlFlowTok{function}\NormalTok{() \{}
\NormalTok{    mean\_val }\OtherTok{\textless{}{-}} \FunctionTok{mean}\NormalTok{(window, }\AttributeTok{na.rm =} \ConstantTok{TRUE}\NormalTok{) }\CommentTok{\# Calcula a média, removendo NAs}
    \ControlFlowTok{for}\NormalTok{ (n }\ControlFlowTok{in} \DecValTok{1}\SpecialCharTok{:}\NormalTok{(width }\SpecialCharTok{{-}} \DecValTok{1}\NormalTok{)) \{}
\NormalTok{      mean0 }\OtherTok{\textless{}{-}} \FunctionTok{mean}\NormalTok{(window[}\DecValTok{1}\SpecialCharTok{:}\NormalTok{n], }\AttributeTok{na.rm =} \ConstantTok{TRUE}\NormalTok{)}
\NormalTok{      mean1 }\OtherTok{\textless{}{-}} \FunctionTok{mean}\NormalTok{(window[(n }\SpecialCharTok{+} \DecValTok{1}\NormalTok{)}\SpecialCharTok{:}\NormalTok{width], }\AttributeTok{na.rm =} \ConstantTok{TRUE}\NormalTok{)}
      \ControlFlowTok{if}\NormalTok{ (}\FunctionTok{is.na}\NormalTok{(mean0) }\SpecialCharTok{||} \FunctionTok{is.na}\NormalTok{(mean1)) }\ControlFlowTok{next} \CommentTok{\# Pula iteração se a média for NA}
      \ControlFlowTok{if}\NormalTok{ (}\FunctionTok{abs}\NormalTok{(mean0 }\SpecialCharTok{{-}}\NormalTok{ mean1) }\SpecialCharTok{\textgreater{}} \FunctionTok{sqrt}\NormalTok{((variance }\SpecialCharTok{/}\NormalTok{ n) }\SpecialCharTok{+}\NormalTok{ (variance }\SpecialCharTok{/}\NormalTok{ (width }\SpecialCharTok{{-}}\NormalTok{ n))) }\SpecialCharTok{*} \FunctionTok{qnorm}\NormalTok{(}\DecValTok{1} \SpecialCharTok{{-}}\NormalTok{ delta)) \{}
\NormalTok{        window }\OtherTok{\textless{}\textless{}{-}}\NormalTok{ window[(n }\SpecialCharTok{+} \DecValTok{1}\NormalTok{)}\SpecialCharTok{:}\NormalTok{width]}
\NormalTok{        width }\OtherTok{\textless{}\textless{}{-}} \FunctionTok{length}\NormalTok{(window)}
\NormalTok{        total }\OtherTok{\textless{}\textless{}{-}} \FunctionTok{sum}\NormalTok{(window, }\AttributeTok{na.rm =} \ConstantTok{TRUE}\NormalTok{)}
\NormalTok{        variance }\OtherTok{\textless{}\textless{}{-}} \FunctionTok{var}\NormalTok{(window, }\AttributeTok{na.rm =} \ConstantTok{TRUE}\NormalTok{)}
        \FunctionTok{return}\NormalTok{(}\ConstantTok{TRUE}\NormalTok{)}
\NormalTok{      \}}
\NormalTok{    \}}
    \FunctionTok{return}\NormalTok{(}\ConstantTok{FALSE}\NormalTok{)}
\NormalTok{  \}}
  
  \FunctionTok{list}\NormalTok{(}\AttributeTok{update =}\NormalTok{ update)}
\NormalTok{\}}

\CommentTok{\# Função para criar a série temporal}
\NormalTok{TrickyTimeSeries }\OtherTok{\textless{}{-}} \ControlFlowTok{function}\NormalTok{(n, k)\{}
\NormalTok{  y1 }\OtherTok{\textless{}{-}} \FunctionTok{rnorm}\NormalTok{(n)}
\NormalTok{  y2 }\OtherTok{\textless{}{-}} \FunctionTok{sqrt}\NormalTok{(}\DecValTok{12}\NormalTok{)}\SpecialCharTok{*}\FunctionTok{runif}\NormalTok{(n, }\AttributeTok{min=}\SpecialCharTok{{-}}\DecValTok{1}\SpecialCharTok{/}\DecValTok{2}\NormalTok{, }\AttributeTok{max=}\DecValTok{1}\SpecialCharTok{/}\DecValTok{2}\NormalTok{)}
\NormalTok{  y3 }\OtherTok{\textless{}{-}} \FunctionTok{rexp}\NormalTok{(n)}\SpecialCharTok{{-}}\DecValTok{1}
\NormalTok{  fk }\OtherTok{\textless{}{-}} \FunctionTok{TK95}\NormalTok{(}\AttributeTok{N=}\DecValTok{3}\SpecialCharTok{*}\NormalTok{n, }\AttributeTok{alpha=}\NormalTok{k)}
\NormalTok{  fk }\OtherTok{\textless{}{-}}\NormalTok{ (fk}\SpecialCharTok{{-}}\FunctionTok{mean}\NormalTok{(fk))}\SpecialCharTok{/}\FunctionTok{sd}\NormalTok{(fk)}
\NormalTok{  output }\OtherTok{\textless{}{-}} \FunctionTok{c}\NormalTok{(y1, y2, y3, fk)}
  \FunctionTok{return}\NormalTok{(output)}
\NormalTok{\}}

\CommentTok{\# Função para gerar ruído colorido}
\NormalTok{TK95 }\OtherTok{\textless{}{-}} \ControlFlowTok{function}\NormalTok{(N, }\AttributeTok{alpha =} \DecValTok{1}\NormalTok{)\{ }
\NormalTok{    f }\OtherTok{\textless{}{-}} \FunctionTok{seq}\NormalTok{(}\AttributeTok{from=}\DecValTok{0}\NormalTok{, }\AttributeTok{to=}\NormalTok{pi, }\AttributeTok{length.out=}\NormalTok{(N}\SpecialCharTok{/}\DecValTok{2}\SpecialCharTok{+}\DecValTok{1}\NormalTok{))[}\SpecialCharTok{{-}}\FunctionTok{c}\NormalTok{(}\DecValTok{1}\NormalTok{,(N}\SpecialCharTok{/}\DecValTok{2}\SpecialCharTok{+}\DecValTok{1}\NormalTok{))] }\CommentTok{\# Frequências de Fourier}
\NormalTok{    f\_ }\OtherTok{\textless{}{-}} \DecValTok{1} \SpecialCharTok{/}\NormalTok{ f}\SpecialCharTok{\^{}}\NormalTok{alpha }\CommentTok{\# Lei de potência}
\NormalTok{    RW }\OtherTok{\textless{}{-}} \FunctionTok{sqrt}\NormalTok{(}\FloatTok{0.5}\SpecialCharTok{*}\NormalTok{f\_) }\SpecialCharTok{*} \FunctionTok{rnorm}\NormalTok{(N}\SpecialCharTok{/}\DecValTok{2{-}1}\NormalTok{) }\CommentTok{\# Parte real}
\NormalTok{    IW }\OtherTok{\textless{}{-}} \FunctionTok{sqrt}\NormalTok{(}\FloatTok{0.5}\SpecialCharTok{*}\NormalTok{f\_) }\SpecialCharTok{*} \FunctionTok{rnorm}\NormalTok{(N}\SpecialCharTok{/}\DecValTok{2{-}1}\NormalTok{) }\CommentTok{\# Parte imaginária}
\NormalTok{    fR }\OtherTok{\textless{}{-}} \FunctionTok{complex}\NormalTok{(}\AttributeTok{real =} \FunctionTok{c}\NormalTok{(}\FunctionTok{rnorm}\NormalTok{(}\DecValTok{1}\NormalTok{), RW, }\FunctionTok{rnorm}\NormalTok{(}\DecValTok{1}\NormalTok{), RW[(N}\SpecialCharTok{/}\DecValTok{2{-}1}\NormalTok{)}\SpecialCharTok{:}\DecValTok{1}\NormalTok{]), }
                  \AttributeTok{imaginary =} \FunctionTok{c}\NormalTok{(}\DecValTok{0}\NormalTok{, IW, }\DecValTok{0}\NormalTok{, }\SpecialCharTok{{-}}\NormalTok{IW[(N}\SpecialCharTok{/}\DecValTok{2{-}1}\NormalTok{)}\SpecialCharTok{:}\DecValTok{1}\NormalTok{]), }\AttributeTok{length.out=}\NormalTok{N)}
\NormalTok{    reihe }\OtherTok{\textless{}{-}} \FunctionTok{fft}\NormalTok{(fR, }\AttributeTok{inverse=}\ConstantTok{TRUE}\NormalTok{) }\CommentTok{\# Retornar ao domínio do tempo com transformada inversa de Fourier}
    \FunctionTok{return}\NormalTok{(}\FunctionTok{Re}\NormalTok{(reihe)) }\CommentTok{\# Retorna apenas a parte real}
\NormalTok{\}}

\CommentTok{\# Geração da série temporal}
\FunctionTok{set.seed}\NormalTok{(}\DecValTok{1234567890}\NormalTok{, }\AttributeTok{kind=}\StringTok{"Mersenne{-}Twister"}\NormalTok{)}
\NormalTok{y }\OtherTok{\textless{}{-}} \FunctionTok{TrickyTimeSeries}\NormalTok{(}\AttributeTok{n=}\DecValTok{100}\NormalTok{, }\AttributeTok{k=}\DecValTok{1}\NormalTok{)}

\CommentTok{\# Funções para calcular H e C}
\NormalTok{shannon\_entropy }\OtherTok{\textless{}{-}} \ControlFlowTok{function}\NormalTok{(probabilities) \{}
  \SpecialCharTok{{-}}\FunctionTok{sum}\NormalTok{(probabilities }\SpecialCharTok{*} \FunctionTok{log2}\NormalTok{(probabilities), }\AttributeTok{na.rm =} \ConstantTok{TRUE}\NormalTok{)}
\NormalTok{\}}

\NormalTok{js\_complexity }\OtherTok{\textless{}{-}} \ControlFlowTok{function}\NormalTok{(probabilities) \{}
\NormalTok{  q }\OtherTok{\textless{}{-}} \FunctionTok{rep}\NormalTok{(}\DecValTok{1}\SpecialCharTok{/}\FunctionTok{length}\NormalTok{(probabilities), }\FunctionTok{length}\NormalTok{(probabilities))}
\NormalTok{  m }\OtherTok{\textless{}{-}}\NormalTok{ (probabilities }\SpecialCharTok{+}\NormalTok{ q) }\SpecialCharTok{/} \DecValTok{2}
\NormalTok{  (}\FunctionTok{shannon\_entropy}\NormalTok{(m) }\SpecialCharTok{{-}} \FloatTok{0.5} \SpecialCharTok{*}\NormalTok{ (}\FunctionTok{shannon\_entropy}\NormalTok{(probabilities) }\SpecialCharTok{+} \FunctionTok{shannon\_entropy}\NormalTok{(q))) }\SpecialCharTok{/} \FunctionTok{log2}\NormalTok{(}\FunctionTok{length}\NormalTok{(probabilities))}
\NormalTok{\}}

\NormalTok{calculate\_probabilities }\OtherTok{\textless{}{-}} \ControlFlowTok{function}\NormalTok{(patterns) \{}
\NormalTok{  table\_patterns }\OtherTok{\textless{}{-}} \FunctionTok{table}\NormalTok{(patterns)}
\NormalTok{  probabilities }\OtherTok{\textless{}{-}} \FunctionTok{as.numeric}\NormalTok{(table\_patterns) }\SpecialCharTok{/} \FunctionTok{sum}\NormalTok{(table\_patterns)}
  \FunctionTok{return}\NormalTok{(probabilities)}
\NormalTok{\}}

\NormalTok{calculate\_hxc\_with\_op }\OtherTok{\textless{}{-}} \ControlFlowTok{function}\NormalTok{(series, window\_size, emb\_dim) \{}
\NormalTok{  n }\OtherTok{\textless{}{-}} \FunctionTok{length}\NormalTok{(series)}
\NormalTok{  h\_values }\OtherTok{\textless{}{-}} \FunctionTok{c}\NormalTok{()}
\NormalTok{  c\_values }\OtherTok{\textless{}{-}} \FunctionTok{c}\NormalTok{()}
  
  \ControlFlowTok{for}\NormalTok{ (i }\ControlFlowTok{in} \DecValTok{1}\SpecialCharTok{:}\NormalTok{(n }\SpecialCharTok{{-}}\NormalTok{ window\_size }\SpecialCharTok{+} \DecValTok{1}\NormalTok{)) \{}
\NormalTok{    window }\OtherTok{\textless{}{-}}\NormalTok{ series[i}\SpecialCharTok{:}\NormalTok{(i }\SpecialCharTok{+}\NormalTok{ window\_size }\SpecialCharTok{{-}} \DecValTok{1}\NormalTok{)]}
\NormalTok{    patterns }\OtherTok{\textless{}{-}} \FunctionTok{ordinal\_patterns}\NormalTok{(window, emb\_dim)}
\NormalTok{    probabilities }\OtherTok{\textless{}{-}} \FunctionTok{calculate\_probabilities}\NormalTok{(patterns)}
\NormalTok{    h }\OtherTok{\textless{}{-}} \FunctionTok{shannon\_entropy}\NormalTok{(probabilities)}
\NormalTok{    c }\OtherTok{\textless{}{-}} \FunctionTok{js\_complexity}\NormalTok{(probabilities)}
\NormalTok{    h\_values }\OtherTok{\textless{}{-}} \FunctionTok{c}\NormalTok{(h\_values, h)}
\NormalTok{    c\_values }\OtherTok{\textless{}{-}} \FunctionTok{c}\NormalTok{(c\_values, c)}
\NormalTok{  \}}
  
  \FunctionTok{data.frame}\NormalTok{(}\AttributeTok{H =}\NormalTok{ h\_values, }\AttributeTok{C =}\NormalTok{ c\_values)}
\NormalTok{\}}

\CommentTok{\# Calcular padrões ordinais usando pracma}
\NormalTok{patterns }\OtherTok{\textless{}{-}} \FunctionTok{ordinal\_patterns}\NormalTok{(y, }\AttributeTok{emb\_dim =} \DecValTok{3}\NormalTok{)}

\CommentTok{\# Verifique se os padrões ordinais foram gerados corretamente}
\ControlFlowTok{if}\NormalTok{ (}\FunctionTok{all}\NormalTok{(}\FunctionTok{is.na}\NormalTok{(patterns))) \{}
  \FunctionTok{stop}\NormalTok{(}\StringTok{"Os padrões ordinais não foram gerados corretamente."}\NormalTok{)}
\NormalTok{\}}

\CommentTok{\# Visualização dos padrões ordinais}
\FunctionTok{plot}\NormalTok{(patterns, }\AttributeTok{type =} \StringTok{\textquotesingle{}l\textquotesingle{}}\NormalTok{, }\AttributeTok{main =} \StringTok{\textquotesingle{}Padrões Ordinais da Série Temporal\textquotesingle{}}\NormalTok{, }\AttributeTok{xlab =} \StringTok{\textquotesingle{}Índice\textquotesingle{}}\NormalTok{, }\AttributeTok{ylab =} \StringTok{\textquotesingle{}Padrão Ordinal\textquotesingle{}}\NormalTok{)}
\end{Highlighting}
\end{Shaded}

\includegraphics{TrickyTimeSeries_files/figure-latex/unnamed-chunk-15-1.pdf}

\begin{Shaded}
\begin{Highlighting}[]
\CommentTok{\# Inicializa o detector ADWIN}
\NormalTok{adwin }\OtherTok{\textless{}{-}} \FunctionTok{ADWIN}\NormalTok{(}\AttributeTok{delta =} \FloatTok{0.01}\NormalTok{)  }\CommentTok{\# Ajuste delta para um valor maior para maior sensibilidade}

\CommentTok{\# Vetor para armazenar os pontos de detecção de mudança}
\NormalTok{change\_points }\OtherTok{\textless{}{-}} \FunctionTok{numeric}\NormalTok{(}\DecValTok{0}\NormalTok{)}

\CommentTok{\# Processa o fluxo de dados usando padrões ordinais}
\ControlFlowTok{for}\NormalTok{ (i }\ControlFlowTok{in} \DecValTok{1}\SpecialCharTok{:}\FunctionTok{length}\NormalTok{(patterns)) \{}
  \ControlFlowTok{if}\NormalTok{ (adwin}\SpecialCharTok{$}\FunctionTok{update}\NormalTok{(patterns[i])) \{}
\NormalTok{    change\_points }\OtherTok{\textless{}{-}} \FunctionTok{c}\NormalTok{(change\_points, i }\SpecialCharTok{+} \DecValTok{2}\NormalTok{) }\CommentTok{\# Ajuste do índice devido ao embedding dimension}
\NormalTok{  \}}
\NormalTok{\}}

\CommentTok{\# Verificar se há mudanças detectadas}
\FunctionTok{print}\NormalTok{(change\_points)}
\end{Highlighting}
\end{Shaded}

\begin{verbatim}
##  [1]  14  23  50  54  57  67 126 130 139 178 311 314 440 443 513 516 567 570 595
## [20] 599
\end{verbatim}

\begin{Shaded}
\begin{Highlighting}[]
\CommentTok{\# Calcular H e C em janelas deslizantes com padrões ordinais}
\NormalTok{window\_size }\OtherTok{\textless{}{-}} \DecValTok{50}
\NormalTok{hxc\_data }\OtherTok{\textless{}{-}} \FunctionTok{calculate\_hxc\_with\_op}\NormalTok{(y, window\_size, }\DecValTok{3}\NormalTok{)}

\CommentTok{\# Plotar o plano HxC}
\FunctionTok{ggplot}\NormalTok{(hxc\_data, }\FunctionTok{aes}\NormalTok{(}\AttributeTok{x =}\NormalTok{ H, }\AttributeTok{y =}\NormalTok{ C)) }\SpecialCharTok{+}
  \FunctionTok{geom\_point}\NormalTok{(}\AttributeTok{color =} \StringTok{"blue"}\NormalTok{) }\SpecialCharTok{+}
  \FunctionTok{labs}\NormalTok{(}\AttributeTok{title =} \StringTok{"Plano HxC with ADWIN and Ordinal Patterns (TrickyTimeSeries)"}\NormalTok{,}
       \AttributeTok{x =} \StringTok{"Entropia de Shannon (H)"}\NormalTok{, }\AttributeTok{y =} \StringTok{"Complexidade de Jensen{-}Shannon (C)"}\NormalTok{) }\SpecialCharTok{+}
  \FunctionTok{theme\_minimal}\NormalTok{() }\SpecialCharTok{+}
  \FunctionTok{geom\_smooth}\NormalTok{(}\AttributeTok{method =} \StringTok{"lm"}\NormalTok{, }\AttributeTok{color =} \StringTok{"red"}\NormalTok{) }\SpecialCharTok{+}
  \FunctionTok{theme\_bw}\NormalTok{()}
\end{Highlighting}
\end{Shaded}

\begin{verbatim}
## `geom_smooth()` using formula = 'y ~ x'
\end{verbatim}

\includegraphics{TrickyTimeSeries_files/figure-latex/unnamed-chunk-15-2.pdf}

\begin{Shaded}
\begin{Highlighting}[]
\CommentTok{\# Criação do data frame para plotagem}
\NormalTok{Tricky }\OtherTok{\textless{}{-}} \FunctionTok{data.frame}\NormalTok{(}\AttributeTok{x =} \DecValTok{1}\SpecialCharTok{:}\FunctionTok{length}\NormalTok{(y), }\AttributeTok{y =}\NormalTok{ y, }\AttributeTok{change =} \FunctionTok{ifelse}\NormalTok{(}\DecValTok{1}\SpecialCharTok{:}\FunctionTok{length}\NormalTok{(y) }\SpecialCharTok{\%in\%}\NormalTok{ change\_points, }\StringTok{"Change Detected"}\NormalTok{, }\StringTok{"No Change"}\NormalTok{))}

\CommentTok{\# Plotar a série temporal com pontos de mudança}
\FunctionTok{ggplot}\NormalTok{(Tricky, }\FunctionTok{aes}\NormalTok{(}\AttributeTok{x =}\NormalTok{ x, }\AttributeTok{y =}\NormalTok{ y)) }\SpecialCharTok{+}
  \FunctionTok{geom\_line}\NormalTok{() }\SpecialCharTok{+}
  \FunctionTok{geom\_point}\NormalTok{(}\AttributeTok{data =} \FunctionTok{subset}\NormalTok{(Tricky, change }\SpecialCharTok{==} \StringTok{"Change Detected"}\NormalTok{), }\FunctionTok{aes}\NormalTok{(}\AttributeTok{x =}\NormalTok{ x, }\AttributeTok{y =}\NormalTok{ y), }\AttributeTok{color =} \StringTok{"red"}\NormalTok{, }\AttributeTok{size =} \DecValTok{2}\NormalTok{) }\SpecialCharTok{+}
  \FunctionTok{labs}\NormalTok{(}\AttributeTok{title =} \StringTok{"Change Detected with ADWIN and Ordinal Patterns (TrickyTimeSeries)"}\NormalTok{, }\AttributeTok{x =} \StringTok{"Índice"}\NormalTok{, }\AttributeTok{y =} \StringTok{"Valor"}\NormalTok{) }\SpecialCharTok{+}
  \FunctionTok{theme\_minimal}\NormalTok{() }\SpecialCharTok{+}
  \FunctionTok{theme\_tufte}\NormalTok{()}
\end{Highlighting}
\end{Shaded}

\includegraphics{TrickyTimeSeries_files/figure-latex/unnamed-chunk-15-3.pdf}

\begin{Shaded}
\begin{Highlighting}[]
\CommentTok{\# Análise detalhada dos pontos de mudança}
\FunctionTok{print}\NormalTok{(}\StringTok{"Análise dos Pontos de Mudança (Padrões Ordinais):"}\NormalTok{)}
\end{Highlighting}
\end{Shaded}

\begin{verbatim}
## [1] "Análise dos Pontos de Mudança (Padrões Ordinais):"
\end{verbatim}

\begin{Shaded}
\begin{Highlighting}[]
\ControlFlowTok{for}\NormalTok{ (point }\ControlFlowTok{in}\NormalTok{ change\_points) \{}
  \FunctionTok{cat}\NormalTok{(}\StringTok{"Ponto de mudança detectado em:"}\NormalTok{, point, }\StringTok{"}\SpecialCharTok{\textbackslash{}n}\StringTok{"}\NormalTok{)}
  \ControlFlowTok{if}\NormalTok{ (point }\SpecialCharTok{\textgreater{}} \DecValTok{1} \SpecialCharTok{\&}\NormalTok{ point }\SpecialCharTok{\textless{}} \FunctionTok{length}\NormalTok{(y)) \{}
\NormalTok{    before\_change }\OtherTok{\textless{}{-}}\NormalTok{ y[(point}\DecValTok{{-}2}\NormalTok{)}\SpecialCharTok{:}\NormalTok{(point}\DecValTok{{-}1}\NormalTok{)]}
\NormalTok{    after\_change }\OtherTok{\textless{}{-}}\NormalTok{ y[(point}\SpecialCharTok{+}\DecValTok{1}\NormalTok{)}\SpecialCharTok{:}\NormalTok{(point}\SpecialCharTok{+}\DecValTok{2}\NormalTok{)]}
    \FunctionTok{cat}\NormalTok{(}\StringTok{"Valores antes da mudança:"}\NormalTok{, before\_change, }\StringTok{"}\SpecialCharTok{\textbackslash{}n}\StringTok{"}\NormalTok{)}
    \FunctionTok{cat}\NormalTok{(}\StringTok{"Valores após a mudança:"}\NormalTok{, after\_change, }\StringTok{"}\SpecialCharTok{\textbackslash{}n\textbackslash{}n}\StringTok{"}\NormalTok{)}
\NormalTok{  \}}
\NormalTok{\}}
\end{Highlighting}
\end{Shaded}

\begin{verbatim}
## Ponto de mudança detectado em: 14 
## Valores antes da mudança: -0.5783557 -0.2266499 
## Valores após a mudança: -0.7437581 0.5294139 
## 
## Ponto de mudança detectado em: 23 
## Valores antes da mudança: 0.6662211 -0.6471975 
## Valores após a mudança: 0.8545988 -0.5707228 
## 
## Ponto de mudança detectado em: 50 
## Valores antes da mudança: -0.1598391 0.7581457 
## Valores após a mudança: 0.6171515 0.3107404 
## 
## Ponto de mudança detectado em: 54 
## Valores antes da mudança: 0.3107404 -0.5589551 
## Valores após a mudança: -0.8243303 -0.4607779 
## 
## Ponto de mudança detectado em: 57 
## Valores antes da mudança: -0.8243303 -0.4607779 
## Valores após a mudança: -0.2720068 0.3767014 
## 
## Ponto de mudança detectado em: 67 
## Valores antes da mudança: 0.4982655 0.6705493 
## Valores após a mudança: 1.486655 1.846015 
## 
## Ponto de mudança detectado em: 126 
## Valores antes da mudança: 0.5136516 1.115655 
## Valores após a mudança: 0.8943628 0.1921691 
## 
## Ponto de mudança detectado em: 130 
## Valores antes da mudança: 0.1921691 -0.395931 
## Valores após a mudança: 1.247944 0.7347326 
## 
## Ponto de mudança detectado em: 139 
## Valores antes da mudança: -1.349955 0.8698094 
## Valores após a mudança: 0.803461 1.131378 
## 
## Ponto de mudança detectado em: 178 
## Valores antes da mudança: -0.8374811 0.3816841 
## Valores após a mudança: -1.280229 -1.26516 
## 
## Ponto de mudança detectado em: 311 
## Valores antes da mudança: 0.6977619 0.3317987 
## Valores após a mudança: -0.3564954 0.5510765 
## 
## Ponto de mudança detectado em: 314 
## Valores antes da mudança: -0.3564954 0.5510765 
## Valores após a mudança: 1.214612 -0.1934968 
## 
## Ponto de mudança detectado em: 440 
## Valores antes da mudança: -0.6373719 -0.998301 
## Valores após a mudança: -1.106472 0.6514193 
## 
## Ponto de mudança detectado em: 443 
## Valores antes da mudança: -1.106472 0.6514193 
## Valores após a mudança: -0.4992131 1.062957 
## 
## Ponto de mudança detectado em: 513 
## Valores antes da mudança: -0.6367534 -1.10539 
## Valores após a mudança: -1.677551 -0.4645944 
## 
## Ponto de mudança detectado em: 516 
## Valores antes da mudança: -1.677551 -0.4645944 
## Valores após a mudança: -1.357671 -0.8029557 
## 
## Ponto de mudança detectado em: 567 
## Valores antes da mudança: -0.1081061 0.4834069 
## Valores após a mudança: -0.5344492 -0.1977881 
## 
## Ponto de mudança detectado em: 570 
## Valores antes da mudança: -0.5344492 -0.1977881 
## Valores após a mudança: 0.6561524 0.6865173 
## 
## Ponto de mudança detectado em: 595 
## Valores antes da mudança: 0.92638 1.134906 
## Valores após a mudança: 1.083102 1.382588 
## 
## Ponto de mudança detectado em: 599 
## Valores antes da mudança: 1.382588 1.149992 
## Valores após a mudança: -1.111688 NA
\end{verbatim}

\emph{Primeira Imagem}
\includegraphics{attachment:\%20file-sANheuzqX7xVxjBntbpaIBQZ}

\begin{Shaded}
\begin{Highlighting}[]
\CommentTok{\# \# Plotar a série temporal com pontos de mudança}
\CommentTok{\# ggplot(Tricky, aes(x = x, y = y)) +}
\CommentTok{\#   geom\_line() +}
\CommentTok{\#   geom\_point(data = subset(Tricky, change == "Change Detected"), aes(x = x, y = y), color = "red", size = 2) +}
\CommentTok{\#   labs(title = "Change Detected with ADWIN (TrickyTimeSeries)", x = "Índice", y = "Valor") +}
\CommentTok{\#   theme\_minimal() +}
\CommentTok{\#   theme\_tufte()}
\end{Highlighting}
\end{Shaded}

Esta imagem também mostra a série temporal do TrickyTimeSeries, mas
desta vez as mudanças foram detectadas apenas usando o algoritmo ADWIN.
Os pontos vermelhos indicam os locais de detecção de mudança. Comparando
com a segunda imagem, podemos observar que há algumas diferenças nos
pontos de detecção, o que sugere que a combinação com Padrões Ordinais
pode melhorar ou alterar a sensibilidade do ADWIN para detectar certas
mudanças na série temporal.

\emph{Segunda Imagem}
\includegraphics{attachment:\%20file-BOsO2j2JCZOo32Gb06gAxeQA}

\begin{Shaded}
\begin{Highlighting}[]
\CommentTok{\# Plotar a série temporal com pontos de mudança}
\FunctionTok{ggplot}\NormalTok{(Tricky, }\FunctionTok{aes}\NormalTok{(}\AttributeTok{x =}\NormalTok{ x, }\AttributeTok{y =}\NormalTok{ y)) }\SpecialCharTok{+}
  \FunctionTok{geom\_line}\NormalTok{() }\SpecialCharTok{+}
  \FunctionTok{geom\_point}\NormalTok{(}\AttributeTok{data =} \FunctionTok{subset}\NormalTok{(Tricky, change }\SpecialCharTok{==} \StringTok{"Change Detected"}\NormalTok{), }\FunctionTok{aes}\NormalTok{(}\AttributeTok{x =}\NormalTok{ x, }\AttributeTok{y =}\NormalTok{ y), }\AttributeTok{color =} \StringTok{"red"}\NormalTok{, }\AttributeTok{size =} \DecValTok{2}\NormalTok{) }\SpecialCharTok{+}
  \FunctionTok{labs}\NormalTok{(}\AttributeTok{title =} \StringTok{"Change Detected with ADWIN and Ordinal Patterns (TrickyTimeSeries)"}\NormalTok{, }\AttributeTok{x =} \StringTok{"Índice"}\NormalTok{, }\AttributeTok{y =} \StringTok{"Valor"}\NormalTok{) }\SpecialCharTok{+}
  \FunctionTok{theme\_minimal}\NormalTok{() }\SpecialCharTok{+}
  \FunctionTok{theme\_tufte}\NormalTok{()}
\end{Highlighting}
\end{Shaded}

\includegraphics{TrickyTimeSeries_files/figure-latex/unnamed-chunk-17-1.pdf}

Esta imagem mostra a série temporal gerada pelo TrickyTimeSeries, onde
as mudanças foram detectadas usando o algoritmo ADWIN em combinação com
Padrões Ordinais (Ordinal Patterns). Os pontos vermelhos indicam os
locais onde o ADWIN detectou uma mudança. As detecções parecem estar
distribuídas ao longo da série temporal, indicando que o ADWIN é
sensível a várias mudanças ao longo da série.

\emph{Terceira Imagem}
\includegraphics{attachment:\%20file-QoN57pG2gkHD3gwlu37rLRCl}

Esta imagem mostra o plano HxC (Entropia de Shannon x Complexidade de
Jensen-Shannon) calculado para a série temporal
\texttt{TrickyTimeSeries} usando janelas deslizantes. O gráfico indica a
relação entre a entropia (H) e a complexidade (C) da série temporal. A
linha vermelha é uma linha de tendência ajustada usando regressão
linear. A distribuição dos pontos no plano HxC pode fornecer insights
sobre a estrutura e a complexidade da série temporal ao longo do tempo.

\emph{Quarta Imagem}

\begin{Shaded}
\begin{Highlighting}[]
\CommentTok{\# Plotar o plano HxC}
\FunctionTok{ggplot}\NormalTok{(hxc\_data, }\FunctionTok{aes}\NormalTok{(}\AttributeTok{x =}\NormalTok{ H, }\AttributeTok{y =}\NormalTok{ C)) }\SpecialCharTok{+}
  \FunctionTok{geom\_point}\NormalTok{(}\AttributeTok{color =} \StringTok{"blue"}\NormalTok{) }\SpecialCharTok{+}
  \FunctionTok{labs}\NormalTok{(}\AttributeTok{title =} \StringTok{"Plano HxC with ADWIN and Ordinal Patterns (TrickyTimeSeries)"}\NormalTok{,}
       \AttributeTok{x =} \StringTok{"Entropia de Shannon (H)"}\NormalTok{, }\AttributeTok{y =} \StringTok{"Complexidade de Jensen{-}Shannon (C)"}\NormalTok{) }\SpecialCharTok{+}
  \FunctionTok{theme\_minimal}\NormalTok{() }\SpecialCharTok{+}
  \FunctionTok{geom\_smooth}\NormalTok{(}\AttributeTok{method =} \StringTok{"lm"}\NormalTok{, }\AttributeTok{color =} \StringTok{"red"}\NormalTok{) }\SpecialCharTok{+}
  \FunctionTok{theme\_bw}\NormalTok{()}
\end{Highlighting}
\end{Shaded}

\begin{verbatim}
## `geom_smooth()` using formula = 'y ~ x'
\end{verbatim}

\includegraphics{TrickyTimeSeries_files/figure-latex/unnamed-chunk-19-1.pdf}

Esta imagem mostra o plano HxC calculado para a série temporal, desta
vez usando Padrões Ordinais em combinação com ADWIN. A entropia de
Shannon e a complexidade de Jensen-Shannon foram calculadas com base nos
padrões ordinais em janelas deslizantes da série temporal. A linha de
tendência é ajustada de maneira semelhante. Comparando com a terceira
imagem, podemos ver que os padrões ordinais influenciam a distribuição
dos pontos no plano HxC, possivelmente fornecendo uma visão mais
detalhada da complexidade estrutural da série temporal.

\emph{Conclusão}

As imagens mostram que a combinação de ADWIN com Padrões Ordinais pode
detectar mudanças diferentes ao longo da série temporal em comparação
com o uso de ADWIN sozinho. Além disso, a análise do plano HxC com
Padrões Ordinais oferece uma visão alternativa da complexidade da série
temporal, potencialmente revelando características que podem não ser
evidentes apenas com ADWIN. \textbf{Essa abordagem combinada pode ser
mais eficaz para detectar mudanças em séries temporais complexas, onde
as mudanças não são apenas de amplitude, mas também de comportamento
estrutural.}

\begin{center}\rule{0.5\linewidth}{0.5pt}\end{center}

\hypertarget{comparauxe7uxe3o-dos-resultados-com-e-sem-padruxf5es-ordinais}{%
\subsection{Comparação dos Resultados com e sem Padrões
Ordinais}\label{comparauxe7uxe3o-dos-resultados-com-e-sem-padruxf5es-ordinais}}

\hypertarget{distribuiuxe7uxe3o-dos-pontos-de-mudanuxe7a}{%
\subsubsection{Distribuição dos Pontos de
Mudança}\label{distribuiuxe7uxe3o-dos-pontos-de-mudanuxe7a}}

\begin{itemize}
\item
  \textbf{Sem OP}: No gráfico original, os pontos de mudança estão mais
  concentrados em certas áreas, especialmente após a metade da série
  temporal.
\item
  \textbf{Com OP}: No gráfico com OP, os pontos de mudança estão mais
  distribuídos ao longo de toda a série temporal.
\end{itemize}

\hypertarget{sensibilidade-uxe0-mudanuxe7a-1}{%
\subsubsection{Sensibilidade à
Mudança}\label{sensibilidade-uxe0-mudanuxe7a-1}}

\begin{itemize}
\item
  \textbf{Sem OP}: O ADWIN detectou mudanças principalmente em regiões
  onde há variações abruptas na série temporal.
\item
  \textbf{Com OP}: O uso de padrões ordinais parece ter aumentado a
  sensibilidade do ADWIN, permitindo a detecção de mudanças mais sutis
  que podem não ser evidentes apenas pela análise direta dos valores da
  série temporal.
\end{itemize}

\hypertarget{interpretauxe7uxe3o-das-melhorias}{%
\subsection{Interpretação das
Melhorias}\label{interpretauxe7uxe3o-das-melhorias}}

O uso de padrões ordinais de Bandt e Pompe melhorou a detecção de
mudanças de algumas maneiras importantes:

\hypertarget{robustez-uxe0-escala}{%
\subsubsection{Robustez à Escala}\label{robustez-uxe0-escala}}

\begin{itemize}
\tightlist
\item
  Padrões ordinais são baseados na ordem relativa dos valores, tornando
  a detecção de mudanças menos suscetível a grandes variações na
  amplitude dos dados. Isso é útil em séries temporais onde a amplitude
  pode variar consideravelmente.
\end{itemize}

\hypertarget{captura-de-estruturas-subtis}{%
\subsubsection{Captura de Estruturas
Subtis}\label{captura-de-estruturas-subtis}}

\begin{itemize}
\tightlist
\item
  Os padrões ordinais capturam mudanças na estrutura da série temporal
  que podem não ser detectadas por mudanças na média ou variância. Isso
  inclui mudanças no comportamento ordinal dos dados.
\end{itemize}

\hypertarget{detecuxe7uxe3o-de-mudanuxe7as-estruturais}{%
\subsubsection{Detecção de Mudanças
Estruturais}\label{detecuxe7uxe3o-de-mudanuxe7as-estruturais}}

\begin{itemize}
\tightlist
\item
  O ADWIN combinado com padrões ordinais é capaz de detectar mudanças
  estruturais nos dados que não seriam óbvias apenas pela análise dos
  valores dos dados.
\end{itemize}

\hypertarget{ajuste-de-paruxe2metros}{%
\subsubsection{Ajuste de Parâmetros}\label{ajuste-de-paruxe2metros}}

\begin{itemize}
\tightlist
\item
  É interessante experimentar com diferentes valores de \texttt{delta} e
  \texttt{emb\_dim} para otimizar a sensibilidade e precisão da detecção
  de mudanças.
\end{itemize}

\hypertarget{anuxe1lise-comparativa}{%
\subsubsection{Análise Comparativa}\label{anuxe1lise-comparativa}}

\begin{itemize}
\tightlist
\item
  Uma análise comparativa mais detalhada dos pontos de mudança
  detectados com e sem o uso de OP para entender melhor as melhorias
  específicas.
\end{itemize}

\hypertarget{aplicauxe7uxf5es-pruxe1ticas}{%
\subsubsection{Aplicações Práticas}\label{aplicauxe7uxf5es-pruxe1ticas}}

\textbf{- Talvez seja interessante aplicar essa metodologia em séries
temporais de outras áreas para validar a eficácia dos padrões ordinais
em diferentes contextos.}

O uso de padrões ordinais de Bandt e Pompe em conjunto com o ADWIN
melhorou a capacidade de detecção de mudanças, tornando o algoritmo mais
sensível a variações estruturais na série temporal. Isso é
particularmente útil em cenários onde as mudanças não são apenas de
amplitude, mas também de comportamento.

\hypertarget{detecuxe7uxe3o-de-mudanuxe7a-com-adwin-and-op-plano-hxc-trickytimeseries}{%
\subsubsection{Detecção de Mudança com ADWIN and OP Plano HxC
(TrickyTimeSeries)}\label{detecuxe7uxe3o-de-mudanuxe7a-com-adwin-and-op-plano-hxc-trickytimeseries}}

\end{document}
